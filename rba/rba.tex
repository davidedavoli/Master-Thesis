%!TeX spellcheck = en-US
In this chapter, we introduce a novel bounded arithmetic for randomized computation
strongly inspired by Ferreira's PTCA, ~\cite{Ferreira90}.
%
As for PTCA, the terms of our arithmetic describe binary strings by means of two simple operations: concatenations and repetitions. The only syntactical difference between our arithmetic and Ferreira's is the presence of a predicate $\Flip$, which we will discuss later.
%
However, the main difference between our bounded arithmetic and Ferreira's or Buss' ones lies within its semantics: we defined a novel quantitative semantics which associates to each formula a measurable set of functions, which can be used to associate a degree of probability to each formula our bounded arithmetic.
%
This Randomized Bounded Arithmetic (RBA) is based on a new first-order ``probabilistic language'' called $\Lpw$,
and is aimed to capture the class of $\PPT$ functions by means of a slightly
modified notion of $\Sigma^b_i$-representability and the theory $\RS$: a variation on Buss' $S^1_2$ \cite{Buss86}.
%
To this aim, we define the class of \emph{polynomial-time oracle recursive functions}
$\POR$ and a notion of arithmetical representability which extends Buss' and Ferreira's.
Finally, we show that $\POR$ functions are exactly the $\Sigma^b_1$-representable ones.
%
However, we will not examine the details of the proofs, but we will only
outline their high-level structure. This because the main focus of this work
is on my personal contributions to this research, while the proofs of the
aforementioned results have been mainly developed by my collaborators Melissa Antonelli, Ugo Dal Lago, Isabel Oitavem and Paolo Pistone.
%
For the detailed proofs of the results within this section, we invite the reader
to consult \cite{RBA}, an unpublished set of notes, describing exhaustively and
extensively the details of this research.
%
This chapter is structured as follows:

\begin{enumerate}
\itemsep0em
\item In Section~\ref{sec:PORandLpw}, we
define the first-order language $\Lpw$ together with
its semantics and the $\POR$ class of functions.

\item In Section~\ref{sec:TaskA},
we outline the proof that all functions
in $\POR$ are $\Sigma^b_1$-representable
in $\RS$.
The proof is by induction on the
structure of $\POR$ and is inspired
by the
encoding machinery
from~\cite{Buss86,Ferreira90}.

\item In Section~\ref{sec:TaskB}
we outline the proof that all functions which
are $\Sigma^b_1$-representable
in $\RS$ are in $\POR$
by way of realizability techniques
similar to Cook and Urquhart's one~\cite{CookUrquhart}.
\end{enumerate}

These results --- namely Theorem \ref{thm:TaskAB} and \ref{cor:main2} --- prove that $\Sigma^b_1$ representable function of $\RS$ are exactly the $\POR$ functions, as stated by Theorem \ref{thm:TaskAB}.

\begin{theorem}[$\Sigma^b_1$-Representability]
  \label{thm:TaskAB}
  Within $\Lpw$, the $\Sigma^b_1$-representable functions of $\RS$ are exactly the
  $\POR$ functions.
\end{theorem}
\noindent
In Chapter \ref{chap:sfptopor}, this result will be extended to the set of $\PPT$
functions.

\begin{restatable}[$\Sigma^b_1$-Representability of $\PPT$ Functions]{theorem}{pptrepr}
  \label{thm:TaskABC}
  It holds that:
  \[
  \forall G \in \Sigma^b_1.\RS \vdash \forall x \exists ! y. G(x, y) \to \exists f_G \in \PPT. \forall x, y \in \Ss.\mu\left(\llbracket G(x, y) \rrbracket\right)=Pr[f_G(x)=y]
  \]
  \noindent
  and that:
  \[
  \forall f \in \PPT.\exists G_f \in \Sigma^b_1. .\RS \vdash \forall x \exists ! y. G(x, y)\land \forall x, y \in \Ss.\mu\left(\llbracket G(x, y) \rrbracket\right)=Pr[f_G(x)=y].
  \]
\end{restatable}
