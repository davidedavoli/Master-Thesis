% !TEX root = Conjecture.tex
%!TeX spellcheck = en-US

\section{All Functions in $\POR$ are $\Sigma^b_1$-Representable in $\RS$}
\label{sec:TaskA}


In this section we show that each polynomial-time oracle function
is $\Sigma^b_1$-representable in our theory $\RS$.
%
To this end, we extend the standard definition
of $\Sigma^b_1$-representability
so to fit our class of string functions
and the probabilistic word language $\Lpw$.



%%% defn
%%% Sigma^b_1-Representability
\begin{defn}[$\Sigma^b_1$-Representability]\label{df:representability}
A function $f:\Ss^j \times \Os \rightarrow
\Ss$ is \emph{$\Sigma^b_1$-representable} in
$\RS$ if and only if there is
a $\Sigma^b_1$-formula
$G(\vec{x},y)$ in $\Lpw$ such that:
\begin{enumerate}
\itemsep0em
\item $\RS\vdash\forall\vec{x}.\exists y.G(\vec{x},y)$
\item $\RS\vdash \forall \vec{x}.\forall y.\forall z.
\big(G(\vec{x},y)\wedge
G(\vec{x},z) \rightarrow y=z\big)$
\item For all $n_1,\dots, n_j,m\in \Ss$,
and $\omega \in \Os$,
$f(n_1,\dots, n_j,\omega)=m$
if and only if $\omega\in\llbracket G({n_1},
\dots, {n_j},{m})\rrbracket$.
\end{enumerate}
\end{defn}

This definition extends Buss' representability condition,
adding a constraint which links the formula's
qualitative semantics to the additional functional parameter of the $\POR$ function algebra.
%
This constraint --- the third one --- is the characterizing feature of this definition. Indeed, it captures the main peculiarity of probabilistic computation: evaluation is a stochastic process so the semantics of a program is a function mapping an input to a probability distribution which reflects all the possible outcomes of the computation.
%
The main result of this section is Theorem \ref{theorem1}. We do not show an entire and comprehensive proof of the result, because I did not directly work on it. However, we will show an exhaustive sketch of the proof given in \cite{RBA}.
%
The proof relies on a
well-known result by Parikh.

%%%
\begin{prop}[{``Parikh''~\cite{Parikh71}}]\label{prop:Parikh}
Let $F(\vec{x},y)$ be a bounded formula in $\Lpw$
such that $\RS\vdash(\forall \vec{x})(\exists y)F(\vec{x},y)$.
Then, there is a term $t$ such that
$\RS \vdash \forall \vec{x}.\exists y\preceq t(\vec{x}).F(\vec{x},y)$.
\end{prop}
\noindent
Actually, Parikh's theorem is usually presented
in the context of Buss' bounded theories, stating that, if $B$ is a first-order formula with variables interpreted over Natural Numbers,
then if $S^i_2 \vdash \forall \vec{x}.\exists y.
B$, there is a term $t(\vec{x})$
such that  $S^i_2\vdash\forall\vec{x}.\exists y\leq
t(\vec{x}).B(\vec{x},y)$,~\cite{Buss86,Buss98}.
However, due to~\cite{FerreiraOitavem},
Buss' \emph{syntactic} proof holds for
Ferreira's
$\Sigma^b_1$-NIA~\cite{Ferreira88} as well. Finally,
the same result holds for
$\RS$, because the set of its axioms does not contain
any specific rule concerning $\Flip(\cdot)$
and so is defined exactly upon the same axioms
of Ferreira's $\Sigma^b_1$-NIA~\cite{Ferreira90}.


%%% THEOREM
%%% theorem1
\begin{theorem}\label{theorem1}
Every $f\in \POR$ is $\Sigma^b_1$-representable in
$\RS$.
\end{theorem}
% Proof Sketch
\begin{proof}[Proof Sketch]
The proof is by induction on the
structure of functions in $\POR$.

%% Basic function


\emph{Base case.}
Each basic function
is $\Sigma^b_1$-representable in $\RS$.

\begin{itemize}
%% f=E
\item
$f=E$ is $\Sigma^b_1$-represented
in $\RS$ by the formula:
$$
G_E(x,y) := x = x \wedge y=\epsilon.
$$

\begin{enumerate}
\item Given $x$, it suffices to take $y=1$ as a witness of the existential quantifier.
\item Uniqueness is a consequence of identity's transitivity.

\item The quantitative claim comes is a consequence of Definition \ref{def:quantsem}.

\end{enumerate}

All the base case are similar to this one, apart from $\query$, which is shown below:




%% f = query
\bigskip
$f=\query$ is
$\Sigma^b_1$-represented in $\RS$
by the formula:
$$
G_{\query}(x,y) := \big(\Flip(x) \wedge y=\one\big) \vee
\big(\neg\Flip(x) \wedge y=\zero\big).
$$
Notice that this proof relies on the fact that every $f\in \POR$
is capable of invoking exactly \emph{one} oracle.
\begin{enumerate}
% Existence
\item Existence is proved by cases: if $\RS\vdash\Flip(x)$,
let $y=\one$.
By the reflexivity of identity
$\RS\vdash\one=\one$ holds,
so also $\RS\vdash \Flip(x) \wedge \one=\one$.
In this case we conclude that:
$$
\RS
\vdash \exists y.((\Flip(x)
\wedge y=\one) \vee
(\neg\Flip(x) \wedge y=\zero)).
$$
%
If $\RS\vdash \neg \Flip(x)$, the proof is analogous.
%% Uniqueness
\item Uniqueness is established relying on the
transitivity of identity.

%% Semantics
\item Finally, we need to show that for every
$n,m\in \Ss$ and $\omega^*\in \Os$,
$\query(n,\omega^*)=m$
if and only if
$\omega^* \in \llbracket G_{\query}({n},{m})\rrbracket$.
%$\omega^* \in \big\llbracket  \big(\Flip({n}) \wedge {m}=\one\big) \vee \big(\neg\Flip({n}) \wedge {m}=\zero\big)\big\rrbracket$.

Suppose $m=\oone$.
Then it holds that $\query(n,\omega^*)=\oone$,
which is equivalent to $\omega^*(n)=\oone$, thus:
\small
\begin{align*}
\big\llbracket\big(\Flip({n}) \wedge
{m}=\one\big) \vee \big(\neg\Flip({n})
\wedge {m}=\zero\big)\big\rrbracket
&=
\llbracket \Flip({n}) \wedge {m}=\one\rrbracket \cup
\llbracket \neg\Flip({n}) \wedge
{m} = \zero\rrbracket \\
%
&= \big(\llbracket \Flip({n})\rrbracket \cap
\llbracket \one =\one\rrbracket \big)
\cup \big(\llbracket \neg\Flip({n})\rrbracket
\cap \llbracket \one =\zero\rrbracket\big) \\
%
&=  \big(\llbracket \Flip({n})\rrbracket
\cap \Os \big) \cup (\llbracket \neg \Flip({n})
\rrbracket \cap \emptyset \big) \\
%
&= \llbracket \Flip({n})\rrbracket \\
%
&= \{\omega \in\Os \ | \ \omega(n)=\oone\}.
\end{align*}
\normalsize
Clearly, $\omega^*\in \big\llbracket\big(\Flip({n})
\wedge {m}=\one\big)
\vee \big(\neg\Flip({n}) \wedge
{m}=\zero\big)\big\rrbracket$.
The case $m=\zzero$ and the opposite direction
are proved similarly.
\end{enumerate}

\end{itemize}

%%% Inductive case.
\emph{Inductive case.}
The discussion of this cases is out of the scope of this
work. However, The proofs of claims 1 and 2 follow the same approach
described by Ferreira in \cite{Ferreira88, Ferreira90}.
In particular this can be done because, from a \emph{syntactical}
point of view, $\RS$ axioms and Ferreira's are exactly the same. This allows us to reuse Parikh's Proposition \cite{Parikh71} (rewritten in Proposition \ref{prop:Parikh}) to fit the induction hypothesis into the $\Sigma^b_1$ representability condition of the claims. Indeed, the induction hypothesis does not pose any bound over $y$'s size. But according to Proposition \ref{prop:Parikh} this can be done. So, for instance, the case of composition, the $\Sigma^b_1$ formula is:
$$
G(x,y) := \exists z_1\preceq t_{h_1}(\vec{x}).
\dots \exists z_k\preceq t_{h_k}(\vec{x}).
\big(G_{h_1}(\vec{x},z_1) \wedge \dots
\wedge G_{h_k}(\vec{x},z_k) \wedge
G_g(z_1,\dots, z_k,y)\big).
$$

where the existence of $h_1,\ldots,h_k$ is a consequence of Parikh's Proposition.

\end{proof}
