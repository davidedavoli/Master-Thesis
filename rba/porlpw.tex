%!TeX spellcheck = en-US
\section{The $\POR$ Function Algebra and $\RS$ Theory}
\label{sec:PORandLpw}

In this section, we collect the definitions of the
$\POR$ (Section \ref{sec:POR}) function algebra and the system of axioms $\RS$,
(Section \ref{sec:S13}).\footnote{
We would like to precise that for brevity we call $\RS$ both the system of axioms
and the theory they induce. However, to be proper, we should refer the theory using a different name.
} Before doing so, for clarity's sake,
we define the notational conventions we are adopting in this whole work.
This is done in Section \ref{sec:notation}.


\subsection{Notational preliminaries}\label{sec:notation}

\begin{defn}[Sets]~
  \begin{itemize}
  \item We call $\Bool=\{\zzero,\oone\}$ the set of bits.
  \item $\Ss=\Bool^*$
  is the set of binary strings of finite
  length.
  \item $\Os=\Bool^\Ss$
    indicates the set of functions from $\Nat$ to $\Bool$.
  \end{itemize}
\end{defn}

Notice that given $\omega\in \Bool$ and $x \in \Ss$,
$\omega(x)$ denotes \emph{one}
specific bit $b \in \Bool$,
the so-called $x$-th bit of $\omega$.
Meta-variables $\omega',\omega'',\dots$
are used to denote the elements $\Os$.

For every $x,y \in \Ss$,
we will use $x \subseteq y$
to express that $x$ is an \emph{initial} or
\emph{prefix substring} of $y$.
% Formally:
%
%
% \begin{defn}[String length]
%   Let $|\cdot|: \Ss \longrightarrow \Nat$ denote the
%   length-map of a string,~i.e.
%   \begin{align*}
%     |\eepsilon| &:=0\\
%     |\sigma \bool| &:=1+ |\sigma|\\
% \end{align*}
% \end{defn}
%
Within $\Ss$, we define two binary operations
(determining a ring) which allow us to define
strings on top of other strings.

\begin{defn}[String concatenation]
  Given two strings $x,y \in \Ss$,
  we denote with
  $x\cconc y$ (which will always be
  abbreviated as simply $xy$)
  the concatenation of $x$ and $y$.
\end{defn}

\begin{defn}[Binary Product]
  Given two strings $x,y \in \Ss$,
  we define their binary product,
  denoted $x\bm\times y$, the string
  obtained by concatenating
  $x$ with itself for $|y|$-times. Formally:
  \begin{align*}
    x \times \eepsilon &:= \eepsilon\\
    x \times y\bool &:= (x\times y)\conc x.
  \end{align*}
\end{defn}

\subsection{The Class $\POR$}\label{sec:POR}

The $\POR$ class of function is strongly inspired by Ferreira's PTCA
\cite{Ferreira90}, which itself is a Cobham style function algebra \cite{Cobham1965}.
%
The main difference between $\POR$ and Ferreira's PTCA is that $\POR$
functions carry an additional functional argument $\omega: \Ss \longrightarrow \Bool$
which is intended to be used as a source of random bits. To do so, a novel function
is added to the set of base functions, i.e. the query function $Q(x, \omega)= \omega(x)$.

%% defn
%% The Class POR
\begin{defn}[The Class $\POR$]
The \emph{class $\POR$} is the smallest class of
functions from
$\Ss^n\times \Os$ to $\Ss$, containing:
\begin{itemize}
\itemsep0em
%
\item The \emph{empty} function $E(x,\omega)=\eepsilon$;
%
\item The \emph{projection} functions $P^n_i(x_1,\dots, x_n,\omega)=x_i$, for
$n\in \Nat$ and $1\leq i\leq n$;
%
\item The \emph{word-successor} $\Sf_{\bbool}(x,\omega)=x\bbool$, for every $\bbool\in\Bool$;
%
\item The \emph{conditional} function
\begin{align*}
\Cf(\eepsilon, y, z_\zzero,z_\oone,\omega) &= y \\
\Cf(x\bbool, y,z_\zzero, z_\oone,\omega) &= z_\bbool,
\end{align*}
where $\bbool\in \Bool$.
%
\item The \emph{query} function $\query(x,\omega)=\omega(x)$;
\end{itemize}
and closed under:
\begin{itemize}
%
\item \emph{Composition}, such that $f$
is defined from $g,h_1,\dots, h_k$
as
$$
f(\vec{x},\omega)=g(h_1(\vec{x},\omega), \dots,
h_k(\vec{x},\omega),\omega).
$$
%
\item \emph{Bounded recursion on notation},
such that $f$ is defined from
$g,h_0,h_1$ as
\begin{align*}
f(\vec{x},\eepsilon, \omega) &= g(\vec{x},\omega); \\
f(\vec{x}, y\zzero,\omega) &= h_0\big(\vec{x},y,f(\vec{x},y,\omega),\omega\big)|_{t(\vec{x},y)}; \\
f(\vec{x},y\oone,\omega) &= h_1\big(\vec{x},y,f(\vec{x},y,\omega),\omega\big)|_{t(\vec{x},y)},
\end{align*}
\noindent
where $t$
is defined from
$\eepsilon,\zzero,\oone,\cconc,
\times$ by explicit definition, which means that $t$ can be obtained by a finite term using composing the functions $\conc$ and $\times$ on the constants $\eepsilon$, $\zzero$, $\oone$ and the variables $\vec x$ and $y$.\footnote{This language is identical to the one of $\Lpw$ terms, Definition \ref{def:lpwterms}.}
\end{itemize}
\end{defn}









% \begin{remark}
% {
% Notice that Ferreira's characterization~\cite{Ferreira90},
% not only does not include the
% query function $\query$, but also the
% conditional is not used.
% Instead, it contains the ``substring-conditional'' function:
% $$
% \emph{S}(x,y,\omega) = \begin{cases}
% \oone \ \ \ &\text{if } x \subseteq y \\
% \zzero \ \ \ &\text{otherwise}
% \end{cases}
% $$
% Nevertheless,
% we can define it due by bounded recursion.
% First, let $f_{\Tail}(x,\omega)$ be defined as follows:
% \begin{align*}
% \emph{Tail}(\eepsilon,\omega) &= \eepsilon \\
% %
% \emph{Tail}(x\bbool,\omega) &= x|_x.
% \end{align*}
% %
% %
% Then, let $Eq(x,y,\omega)$ be:
% \begin{align*}
% \emph{Eq}(x,\eepsilon,\omega) &= \Cf(x,\oone,\zero,\zero,\omega) \\
% %
% \emph{Eq}(x,y\zzero,\omega) &= \Cf(x, \zero,
% \emph{Eq}(\emph{Tail}(x), y,\omega), \zero, \omega)|_{\one} \\
% %
% \emph{Eq}(x,y\oone,\omega) &= \Cf(x, \zero, \zero,
% \emph{Eq}(\emph{Tail}(x),y,\omega),\omega)|_{\one}.
% \end{align*}
% Finally, we can define $\emph{S}(x,y,\omega)$ as:
% \begin{align*}
% \emph{S}(x,\eepsilon,\omega) &= \Cf(x,\oone,\zzero,\zzero,
% \omega) \\
% %
% \emph{S}(x, y\zzero, \omega) &=
% \Cf\big(x,\oone, \Cf \big(\emph{Eq}(x,y\zzero,\omega),
% \emph{S}(x,y,\omega), \oone,\oone,\omega\big),
% \emph{S}(x,y,\omega),\omega\big) \\
% %
% \emph{S}(x,y\oone,\omega) &=
% \Cf\big(x, \oone,
% \emph{S}(x,y,\omega),
%  \Cf\big(\emph{Eq}(x,y\oone,\omega),
% \emph{S}(x,y,\omega),\oone,\oone,\omega)\big),\omega\big)
% \end{align*}}
% \end{remark}
% %
% %
% \noindent
% Actually even the conditional function $\Cf$
% could be defined by bounded recursion
% as follows:
% \begin{align*}
% \Cf(\eT, y, z_0, z_1,\omega) &= y \\
% %
% \Cf(x\zzero, y, z_0, z_1,\omega) &= z_0|_{z_0} \\
% %
% \Cf(x\oone, y, z_0, z_1, \omega) &= z_1|_{z_1}.
% \end{align*}
% but we will take it as a primitive function of
% $\POR$ to make the the realizability interpretation
% of Section~\ref{sec:TaskB} better readable.







































%%%% SUBSECTION
%%%% THE THEORY \RS
%%%% sec:S13
\subsection{The Theory $\RS$}\label{sec:S13}

The theory $\RS$ is a bounded arithmetic inspired by Ferreira's \cite{Ferreira90}.
This first-order theory relies on the following components:

\begin{itemize}
  \item A first-order language $\Lpw$ with equality whose terms represent strings, introduced in Section \ref{subsub:lpw}.
  \item A quantitative semantics $\llbracket \cdot\rrbracket$ associating a measurable set to each formula, which is described in Section \ref{subsub:semantics}.
  \item A set of axioms, and a derivation system, discussed in Section \ref{subsub:S13}.
\end{itemize}



%%% PARAGRAPH
%%% The Language L_PW
\subsubsection{The Language $\Lpw$.}\label{subsub:lpw}
The language $\Lpw$ is the
first-order language
with equality defined in~\cite{FerreiraOitavem},
augmented by a predicate symbol
$\Flip(\cdot)$, as described below:


%%% defn
%%% Terms
\begin{defn}[Terms]
  \label{def:lpwterms}
Let $x,y,\dots$ denote variables.
Terms are defined by the following grammar:
$$
t,s ::= x \midd \epsilon \midd \zero \midd \one
\midd t \conc s \midd t\times s.
$$
\end{defn}


%%% Notation
\begin{notation}
The symbol $\conc$ is usually omitted:
$t\conc s$ is abbreviated as $ts$.
\end{notation}


%%% defn
%%% Formul\ae{}
\begin{defn}[Formul\ae{}]
Let $x,y,\dots$ denote variables and
$t,s,\dots$ terms.
Formul\ae{} are defined by the following
grammar:
$$
F, G ::= \Flip(t) \midd t=s \midd t\subseteq s
\midd \neg F \midd F\wedge G \midd F\vee G
\midd F\rightarrow G \midd \exists x.F \midd
\forall x.F.
$$
\end{defn}

As a syntactical facilitation, we define some notations which do not extend the
language's expressive power, yet they enable us to write more concise and clear formul\ae{}.

%% Notation
\begin{notation}[Exponentiation]~
  \begin{itemize}
    \item For every term $t$ of $\Lpw$, the abbreviation
    $\one^t$ stands for the term $\one\times t$.
    \item Given two terms of $\Lpw$ $t,s$,
    the formula $t\preceq s$ is syntactic
    sugar for $\one^t\subseteq \one^s$,
    meaning that
    the length of $t$ is
    less than or equal to that of $s$.
    \item If
      $t,r,s$ are strings, the
      abbreviation $t|_r=s$
      denotes the following formula:
      $$
      (\one^r\subseteq \one^t
      \wedge s\subseteq t \wedge \one^r=\one^s)
      \vee (\one^t\subseteq \one^r \wedge s=t)
      $$
      saying that $s$ is the \emph{truncation}
      of $t$ at the length of $r$.
  \end{itemize}
\end{notation}

As for Buss' arithmetic, even our arithmetic relies on bounded quantification.
This is done with the aim of keeping the size of the terms within the formul\ae{}
under control and then, to keep the functions' time complexity within specific classes.
%
To this aim, Buss introduces two different flavors of quantification, namely
bounded quantification and
sharply bounded quantification (Definition \ref{def:bussboundedquantifiers});
within our framework, they respectively correspond to
 bounded quantification and subword quantification.

%%% Notation
\begin{notation}[Bounded Quantification]
In $\Lpw$, \emph{bounded
quantification} is quantification
in the form $\forall x\preceq t.F$,
which abbreviates
$\forall x.
\one^x\subseteq \one^t\rightarrow F$,
or
$\exists x\preceq t.F$.
%which denotes  $(\exists x)(\one^x\subseteq \one^t\wedge F)$, where the formula $t\preceq s$ is syntactic sugar for $\one^t\subseteq \one^s$.
%
\end{notation}
This form of quantification, similarly to Buss' bounded quantification $Q x \le k$\footnote{With $k \in \Nat$ and $Q\in\{\forall, \exists\}$}, ranges over $O(2^{|k|})$ values of $x$.

\begin{notation}[Subword quantification]
\emph{Subword quantification}
is quantification in the form
$\forall x\subseteq^* t$ and
$\exists x\subseteq^* t$,
such that $\forall x\subseteq^* t.F$
and $\exists x\subseteq^* t.F$
abbreviate (resp.)
$\forall x.(\exists w \subseteq t.(wx\subseteq t)
\rightarrow F$
and $\exists x.(\exists w\subseteq t.(wx
\subseteq t) \wedge F)$.
%where the formula $t\subseteq^* s$ is  syntactic sugar for $(\exists r\subseteq s)(rt\subseteq^* s)$.
%
\end{notation}
\noindent
As for Buss' sharp quantification, $\forall x \le |k|$, allows the variable $x$ to range over a set of values whose size is linear in $k$.
%
For readability's sake, in the following,
we also abbreviate the so-called
\emph{initial subword quantifications}
$\forall x.x\subseteq t \rightarrow F$
as $\forall x\subseteq t.F$
and $\exists x.x\subseteq t \wedge F$
as $\exists x\subseteq t.F$.






%%% SUBSECTION
%%% Semantics for Formul\ae{} in L_PW
%%% sec:semantics
\subsubsection{Semantics for Formul\ae{} in $\Lpw$}\label{subsub:semantics}

Upon $\Lpw$, we can define different semantics: for the aims of this work,
we want our semantics to capture measure-related concerns about $\Lpw$'s
formul\ae{}. To this end,
we introduce the alternative,
\emph{quantitative}
semantics for $\Lpw$-terms and formul\ae{} inspired by~\cite{ADLP21}.
Terms are interpreted in
a standard way, so mapping constants to canonical strings and
functions to members of $\Ss^{\Ss\times \Ss}$.
This is done in Definition~\ref{df:terms} below.
On the other hand, the semantics for formul\ae{}
is inherently
quantitative,
as any formula is associated with a (measurable)
set, $\llbracket F\rrbracket \in \sigma(\mathscr{C})$.
%
Intuitively, the \emph{quantitative} semantics of $\Lpw$ associates to
a formula $F$ the set of characteristic functions $\omega \in \sigma(\mathscr{C})$
such that, if these functions are employed as $\Flip$'s interpretations for
the \emph{qualitative}
semantics of $\Lpw$, then $F$ is valid. In order to convey this intuition,
we first introduce the standard \emph{qualitative} semantics then
the \emph{quantitative} one and, finally, in Remark \ref{rem:quantqual} we then show their relation.






%%% defn
%%% Interpretation of Terms
\begin{defn}[Interpretation for Terms]\label{df:terms}
An environment $\xi:\mathcal{G}\mapsto \Ss$,
where $\mathcal{G}$ is the set of term variables,
is a mapping that assigns to each variable
a string.
Given a term $t$ in $\Lpw$ and an environment
$\xi$,
the \emph{interpretation of $t$ in $\xi$}
is the string $\llbracket t\rrbracket_\xi \in \Ss$
inductively defined as follows:

\begin{minipage}{\linewidth}
\begin{minipage}[t]{0.4\linewidth}
\begin{align*}
\llbracket \epsilon\rrbracket_\xi &:=\eepsilon \\
\llbracket \zero\rrbracket_\xi &:=\zzero \\
\llbracket \one \rrbracket_\xi &:=\oone
\end{align*}
\end{minipage}
\hfill
\begin{minipage}[t]{0.6\linewidth}
\begin{align*}
\llbracket x\rrbracket_\xi &:= \xi(x) \in \Ss \\
\llbracket t\conc s\rrbracket_\xi &:=\llbracket t\rrbracket_\xi
 \llbracket s\rrbracket_\xi \\
\llbracket t\times s\rrbracket_\xi &:=\llbracket t\rrbracket_\xi
\times \llbracket s\rrbracket_\xi.
\end{align*}
\end{minipage}
\end{minipage}
\end{defn}

The standard, \emph{qualitative}
model for terms and formul\ae{}
of $\Lpw$ consists in
$\mathscr{W} = (\Ss, \cconc, \times)$.
In this case, logical operators are interpreted
in the canonical way and
$\Flip(\cdot)$
is treated as a standard, unary predicate
of first-order logic, which is interpreted
as a subset of $\Ss$.

%%% defn
%%% Qualitative Semantics for L_PW-Formul\ae{}
\begin{defn}[Qualitative Semantics for $\Lpw$-Formul\ae{}]
  \label{def:qualsem}
Given a formula $F$ in $\Lpw$
and an interpretation $\rho
=(\xi,\omega^{\mathtt{FLIP}})$, where $\xi :\mathcal{G}\rightarrow
\Ss$ and $\omega^{\mathtt{FLIP}} \in \Os$,
the \emph{interpretation of $F$ in $\rho$},
$\llbracket F\rrbracket_\rho$,
is inductively defined as follows:

\begin{minipage}{\linewidth}
\begin{minipage}[t]{0.4\linewidth}
\begin{align*}
\llbracket \Flip(t)\rrbracket_\rho &:=
\begin{cases}
1 \ \ \ &\text{if } \omega^{\mathtt{FLIP}}(\llbracket t\rrbracket_\xi)
=\oone \\
0 \ \ \ &\text{otherwise}
\end{cases} \\
%
\llbracket t=s\rrbracket_{\rho} &:=
\begin{cases}
1 \ \ \ &\text{if } \llbracket t\rrbracket_{\xi}
= \llbracket s\rrbracket_\xi \\
0 \ \ \ &\text{otherwise} \\
\end{cases} \\
%
\llbracket t\subseteq s\rrbracket_\rho
&:=
\begin{cases}
1 \ \ \ &\text{if } \llbracket t\rrbracket_\xi \ssubseteq
\llbracket s\rrbracket_\xi \\
0 \ \ \ &\text{otherwise}
\end{cases}
\end{align*}
\end{minipage}
\hfill
\begin{minipage}[t]{0.5\linewidth}
\begin{align*}
\llbracket \neg G\rrbracket_\rho &:= 1 -
\llbracket G\rrbracket_\rho \\
%
\llbracket G\wedge H\rrbracket_\rho &:=
min\{\llbracket G\rrbracket_\rho, \llbracket H\rrbracket_\rho\} \\
%
\llbracket G\vee H\rrbracket_\rho &:=
max\{\llbracket G\rrbracket_\rho, \llbracket H\rrbracket_\rho\} \\
%
\llbracket G\rightarrow H\rrbracket_\rho
&:= max\{(1-\llbracket G\rrbracket_\rho), \llbracket H\rrbracket_\rho\} \\
%
\llbracket \forall x.G\rrbracket_\rho &:= min\{\llbracket G
\rrbracket_{(\xi\{x\leftarrow s\}, \omega^{\mathtt{FLIP}}\})} \ | \ s\in \Ss\} \\
%
\llbracket \exists x.G\rrbracket_\rho &:=
max\{\llbracket G\rrbracket_{(\xi\{x\leftarrow s\}, \omega^{\mathtt{FLIP}}\})} \ | \ s \in \Ss\}.
\end{align*}
\end{minipage}
\end{minipage}
\end{defn}


The model $\mathscr{W}$
can be extended to a probability space
by considering as the underlying
sample space $\Os$.
There are standard ways of building
a well-defined $\sigma$-algebra and
a probability space over $\Os$.

\begin{defn}
  \label{def:cylsigmaalgebra}
  For every countable set $A$, each $K$ a finite subset of $A$
  and $H\subseteq \Bool^K$, each subset of $\Bool^A$ of the form:
  $$
  \mathsf{C}(H) = \{\omega \in \Bool^A\ | \ \omega|_K \subseteq H\},
  $$
  are called \emph{cylinders} over $A$,~\cite{Billingsley}.\footnote{To be precise, these objects
  are a slight variation of {Billingsley}'s \emph{cylinders}, which
  are subsets of $\{\zero, \one\}^\Nat$.}
\end{defn}

Let $\mathscr{C}$
denote the set of all
cylinders and the $\sigma$-algebra generated by cylinders over $\Os$.
The smallest $\sigma$-algebra
including $\mathscr{C}$,
which is Borel's,
is indicated as $\sigma(\mathscr{C})$.
There is a natural way of defining a probability
measure $\mu$
on $\mathscr{C}$, the canonical way is described in Definition \ref{def:mu}.

\begin{defn}[Cylinder measure, \cite{Billingsley}]
  \label{def:mu}
  Be $S \subseteq \{\zero, \one\}^A$ for a countable $A$,
  $K$ a finite subset of $A$
  and $H\subseteq \Bool^K$,
  (in this context $A=\Ss$, but later $A$ we will tale in exam even the case $A=\Nat$)
  such that:
  $$
    S=\{\omega \in \{\zero, \one\}^A\, |\, \omega|_K \in H\}
  $$
  then, we define $\mu(S)$ as follows:
  $$
  \mu(S)= \frac {|H|}{2^{|K|}}.
  $$
\end{defn}

Without proving that Definition \ref{def:mu} is a measure over the $\sigma$-field
$\sigma(\mathscr C)$, we still cannot introduce any quantitative semantics for
$\Lpw$.
These results are in Chapter \ref{app}:
in Remark \ref{rem:mufun}, we prove that $\mu$ is a function, while in
Remark \ref{rem:mumeas}, we show that $\mu$ is a measure function on $\sigma(\mathscr C)$.


%Definition \ref{def:mu} is well-founded because of the definition of $\sigma$-algebra, indeed $\overline {\mathsf{C}_b}={\mathsf{C}_{\overline b}} $.
Due to the existence of $\mu$, the canonical model $\mathscr{W}=(\Ss,\cconc,\times)$
can be generalized to
$\mathscr{P}=(\Os, \cconc,\times, \sigma(\mathscr{C}),
\mu_\mathscr{C})$.
As anticipated, this new semantics has a concrete application
based of probability experiments. Indeed,
when interpreting
sequences in $\Os$
as the outcome of Bernoulli's process,
the set of sequences such that
the $n$-th
coin flip's result is
$\oone$ (for any fixed $n\in \Nat$)
is assigned measure
$\frac{1}{2}$,
meaning that each random bit is uniformly
distributed and independent of the
others. Namely:
$$
\llbracket \Flip(t)\rrbracket_\xi := \mathsf C(\{\llbracket t\rrbracket_\xi\mapsto \one\}).
$$


%%% defn
%%% Quantitative Semantics
\begin{defn}[Quantitative Semantics]\label{def:quantsem}
Given a formula
$F$ and an environment $\xi:\mathcal{G}\rightarrow
\Ss$, where $\mathcal{G}$
is the set of term variables,
the \emph{interpretation of F in $\xi$},
$\llbracket F\rrbracket_\xi$,
is the (measurable) set of sequences
%\in \sigma(\mathscr{C})$
inductively defined as follows:

\begin{minipage}{\linewidth}
\begin{minipage}[t]{0.4\linewidth}
\begin{align*}
\llbracket \Flip(t)\rrbracket_{\xi} &:= \{\omega \ | \
\omega(\llbracket t\rrbracket_{\xi}) = \oone\} \\
%\mathsf{C}_{X^1_{|\llbracket t\rrbracket_\xi|}} \\
\llbracket t=s\rrbracket_\xi &:= \begin{cases}
\Os \ \ &\text{if } \llbracket t\rrbracket_\xi = \llbracket s\rrbracket_\xi \\
\emptyset \ \ &\text{otherwise}
\end{cases} \\
\llbracket t\subseteq s\rrbracket_{\xi} &:=
\begin{cases}
\Os \ \ \ &\text{if } \llbracket t\rrbracket_\xi
\ssubseteq \llbracket s\rrbracket_\xi \\
\emptyset \ \ \ &\text{otherwise}
\end{cases}
\end{align*}
\end{minipage}
\hfill
\begin{minipage}[t]{0.5\linewidth}
\begin{align*}
\llbracket \neg G\rrbracket_\xi &:=
\Os\setminus\llbracket G\rrbracket_\xi \\
\llbracket G\vee H\rrbracket_\xi &:= \llbracket G\rrbracket_\xi \cup \llbracket H\rrbracket_\xi \\
%
\llbracket G\wedge H\rrbracket_\xi &:=
\llbracket G\rrbracket_\xi \cap \llbracket H\rrbracket_\xi \\
%
\llbracket G\rightarrow H\rrbracket_\xi &:=
(\Os\setminus\llbracket G\rrbracket_\xi)
\cup \llbracket H\rrbracket_\xi \\
%
\llbracket \exists x.G\rrbracket_\xi &:=
\bigcup_{i\in\Ss} \llbracket G\rrbracket_{\xi\{x\leftarrow i\}} \\
%
\llbracket \forall x.G\rrbracket_\xi &:=
\bigcap_{i\in \Ss}\llbracket G\rrbracket_{\xi\{x\leftarrow i\}}.
\end{align*}
\end{minipage}
\end{minipage}
%where $|\llbracket t\rrbracket_\xi|$ denotes the length of $\llbracket t\rrbracket_\xi\in \Ss$.
\end{defn}
\noindent
The semantics is well-defined since
the sets $\llbracket \Flip(t)\rrbracket_\xi$,
$\llbracket t=s\rrbracket_\xi$
and $\llbracket t\subseteq s\rrbracket_\xi$
are measurable
and measurability is preserved by all
the logical operators.










%%% NOTATION
\begin{notation}
For readability's sake, in the following part of this work, we may abbreviate
the former interpretation as $\llbracket \cdot
\rrbracket_\omega$
and the quantitative interpretation
$\llbracket \cdot\rrbracket_\xi$
simply as $\llbracket \cdot\rrbracket$.
\end{notation}

As we mentioned before, the \emph{quantitative} and the \emph{qualitative} semantics are strictly related one with each other:


\begin{remark}
\label{rem:quantqual}
For each $\Lpw$ formula $F$, environment $\xi$, function $\omega \in \Os$ and each interpretation $\rho=(\omega, \xi)$, let $\llbracket \cdot \rrbracket_\xi$ the \emph{quantitative} semantics for $\Lpw$ described in Definition \ref{def:quantsem} and $\llbracket \cdot \rrbracket_{\rho}$ the \emph{qualitative} semantics for $\Lpw$ described in Definition \ref{def:qualsem},
then it holds that:
$$
\llbracket F\big\rrbracket_{(\xi, \omega)} = 1 \leftrightarrow
\omega \in \llbracket F \rrbracket_\xi.
$$
\end{remark}
\begin{proof}
  The proof goes by induction on $F$'s syntax. The only base case is $\Flip$'s, which is a direct consequence of Definitions \ref{def:quantsem} and \ref{def:qualsem}. All the other cases are trivial consequences of the induction hypothesis(es) and the definitions of the two semantics.
\end{proof}
%%%% PARAGRAPH
%%%% THE THEORY \RS
\subsubsection{The Theory $\RS$.}
\label{subsub:S13}

As we anticipated in the previous sections, the theory $\RS$ includes
the axioms by~\cite{Ferreira90} as
expressed in $\Lpw$ using any derivation system for classical first-order logic, such as, for example, sequent calculus or natural deduction.
%
As for Buss' Bounded Arithmetic, our characterization relies on syntactically defined formul\ae{}. In particular, the $\POR$ class will be captured by particular $\Sigma^b_1$ formul\ae{}.


%%% Sigma^b_1-Formula
\begin{defn}[$\Sigma^b_1$-Formula]\label{df:Sigmab1}
A $\Sigma^b_0$-formula  is a subword quantified formula,
i.e. a formula belonging to the smallest
class of $\Lpw$ containing atomic
formul\ae{} and closed under Boolean operations
and subword quantifications.
A \emph{$\Sigma^b_1$-formula}
in $\Lpw$ is a formula of the form
$\exists x.\big(x\preceq t(\vec{z})\wedge
F(\vec{z},x)\big)$,
where $F$ is a subword quantified formula. For simplicity, we will call $\Sigma^b_1$ the class
containing all and only the $\Sigma^b_1$-formul\ae{}.
\end{defn}
% \noindent
% Every string $s\in \Ss$
% can be seen as a term $\overline{s}$
% of $\Lpw$, such that
% $\overline{\eepsilon}=\epsilon,
% \overline{s\zzero} = \overline{s}\conc\zero$
% and $\overline{s\oone}=\overline{s}\conc\one$,~e.g.
% $\overline{\zzero\zzero\oone}
% = \zero\conc\zero\conc\one$.


%\begin{notation}[{$s$-word}]\label{sword}
%Given a string $s\in \Ss$, let us inductively  define the corresponding {\emph{s-word}},  denoted $\overline{s} \in \Lpw$, as follows:
%\begin{align*}
%\overline{\eepsilon} &:= \epsilon \\
%\overline{s\cconc\zzero} &:= \overline{s}\zero \\
%\overline{s\cconc\oone} &:= \overline{s}\one.
%\end{align*}
%\end{notation}






% Theory \RS
\begin{defn}[Theory $\RS$]
  \label{def:S13}
The theory $\RS$ is defined by
axioms belonging to two classes:
\begin{itemize}
\itemsep0em
\item \emph{Basic} axioms:
\begin{enumerate}
\itemsep0em
\item $x\epsilon=x$;
\item $x(y\zero)=(xy)\zero$;
\item $x(y\one)=(xy)\one$;
\item $x\times \epsilon=\epsilon$;
\item $x\times y\zero=(x\times y)x$;
\item $x\times y\one =(x\times y)x$;
\item $x\subseteq \epsilon \leftrightarrow
x=\epsilon$;
\item $x\subseteq y\zero \leftrightarrow
x\subseteq y \vee x=y\zero$;
\item $x\subseteq y\one \leftrightarrow
x\subseteq y \vee x=y\one$;
\item $x\zero=y\zero \rightarrow x=y$;
\item $x\one=y\one \rightarrow x=y$;
\item $x\zero\neq y\one$;
\item $x\zero\neq \epsilon$;
\item $x\one \neq \epsilon$.
\end{enumerate}

\item Axiom scheme for \emph{induction on
notation}:
$$
B(\epsilon) \wedge \forall x.\big(B(x)
\rightarrow B(x\zero)\wedge B(x\one)\big)
\rightarrow \forall x.B(x),
$$
where $B$ is a $\Sigma^b_1$-formula
in $\Lpw$.
\end{itemize}
And closed under the rules of standard first-order classical logic.
\end{defn}
\noindent

% An \emph{extended $\Sigma^b_1$-formula}
% is any formula of $\Lpw$
% that can be constructed in a finite
% number of steps
% by starting with subword quantifications and bounded
% existential quantifications.
%
% \begin{prop}[\cite{Ferreira90}]
% In $\RS$ any extended $\Sigma^b_1$-formula
% is logically equivalent to a $\Sigma^b_1$-formula.\footnote{
% Actually,
% Ferreira proves this result for his theory
% $\Sigma^b_1$-NIA~\cite[pp. 148-149]{Ferreira88},
% but it clearly holds for $\RS$ as well.}
% \end{prop}
