%!TeX spellcheck = en-US
\section{From $\RS$ Representability to $\POR$}
\label{sec:TaskB}
% !TEX root = Conjecture.tex


In this section, we show
that if a function is
$\Sigma^b_1$-representable
in $\RS$,
then it is in $\POR$, as stated by Theorem \ref{thm:taskB}


\begin{theorem}\label{thm:taskB}
Let $\RS \vdash \forall x.\exists y\preceq t.A(x,y)$,
where $A$ is a $\Sigma^b_1$-formula
with only $x,y$ free.
For any function $f:\Ss\times \Os \rightarrow \Ss$,
if $\forall x.\exists y\preceq t. A(x,y)$
represents $f$ so that:
\begin{enumerate}
\itemsep0em
\item $\RS \vdash \forall x.\exists !y.A(x,y)$
\item $\llbracket A(s_1,s_2)\rrbracket =
\{\omega \ | \ f(s_1,\omega)=s_2\}$,
\end{enumerate}
then $f\in \POR$.
\end{theorem}


The proof is based on the work by Cook and Urquhart's
for the system IPV$^\omega$~\cite{CookUrquhart}
and is structured as follows:
\begin{enumerate}
\itemsep0em
\item
First, in Section~\ref{sec:PORo},
we define
a basic equational theory
$\PORo$
for a simply typed $\lambda$-calculus
enriched with primitives corresponding
to functions of $\POR$.

\item
Then, in Section~\ref{sec:IPORo},
we we show that for each $\Sigma^b_1$-representable function
there is a $\PORo$ term which computes that function.
To do so, we define a
first-order \emph{intuitionist} theory
$I\PORo$, which extends
$\PORo$ with usual predicate calculus. Then,
we show that from any derivation of
$\forall x.\exists y.A(x,y)$,
where $A$ is a $\Sigma^b_0$-formula,
one can extract a $\lambda$-term
$\tTerm$ of $\PORo$
such that $\forall x.A(x,\tTerm x)$
is provable {in $I\PORo$}.
This allows us to conclude
that every function which is $\Sigma^b_1$-representable
in $I\RS$ is in $\POR$. Finally,
we extend this result to classical
$\RS$.
\end{enumerate}

As for Section \ref{sec:TaskA}, we will not get into the technical
details of the proof, as they can be consulted in \cite{RBA}.
























































































%%%%%% SUBSECTION
%%%%%% The system POR\lambda
\subsection{The System $\PORo$}\label{sec:PORo}
$\PORo$ is an equational theory
for a simply typed $\lambda$-calculus
augmented typed constants for $\POR$-like primitive functions.
















%%%%%%%% SUBSUBSECTION
%%%%%%%% The SYNTAX OF POR\lambda
\subsubsection{The $\PORo$ Theory}

$\PORo$ is an equational theory whose terms are ordinary
simply-typed $\lambda$-terms. Together with ordinary $\lambda$-terms,
the language of $\PORo$ contains constants
corresponding to $\POR$-like primitives (Definition \ref{def:termsPORo}).
Thus, the derivation system of $\PORo$ is
induced by a set of equational-axioms defining the reduction rules of the base functions of $\PORo$, $\lambda$-terms and a set derivation rules describing the
behavior of the identity predicate.


%% Terms of POR\lambda
\begin{defn}[Terms of $\PORo$]\label{def:termsPORo}
\emph{Terms of $\PORo$}
are standard, simply typed
$\lambda$-terms plus the
constants from the signature below:
\begin{align*}
\zT, \oT, \eT &: \sT \\
\cdotT &: \sT \arrowT \sT \arrowT \sT \\
\Tail &: \sT \arrowT \sT \\
\Trunc &: \sT \arrowT \sT \arrowT \sT \\
\Cond &: \sT \arrowT \sT \arrowT \sT \arrowT \sT \arrowT \sT \\
\flipcoin &: \sT \arrowT \sT \\
\Rec &: \sT \arrowT (\sT \arrowT \sT \arrowT \sT)
\arrowT (\sT \arrowT \sT \arrowT \sT) \arrowT
(\sT \arrowT \sT) \arrowT \sT \arrowT \sT.
\end{align*}
\end{defn}
%
%
\noindent
{In addition to the constants of Definition \ref{def:termsPORo},
we employ a set of axioms
endowing specific equational behaviors to these symbols.
These equations ground a $\POR$-like function algebra
in which:
\begin{itemize}
  \item $\Tail(x)$ computes the string
  obtained by deleting
  {the rightmost of $x$}; thus it can be defined by the axioms:
  \begin{align*}
    \Tail(\eepsilon) &= \eepsilon\\
    \Tail(x b) &= b.
  \end{align*}
  \item $\Trunc(x,y)$
  outputs the \emph{left} prefix
  $x$ obtained truncating $x$ at the length of $y$;
  this corresponds to the $\Lpw$ (and $\POR$'s truncating operator $\cdot|_\cdot$).
  \item $\Cond(x,y,z,w)$ corresponds to $\POR$'s $C$: indeed, it
  computes the function that
  yields $y$
  when $x=\eepsilon$,
  $z$ when $x=x'\zzero$, and
  $w$ when $x=x'\oone$;
  \item $\flipcoin(x)$ is a random bit
  generator, corresponding to $\POR$'s $Q$.
  \item $\Rec$
  is the operator for bounded recursion
  on notation.
\end{itemize}

Notice that with respect to $\POR$, we are not defining a composition term constant, this because
$\lambda$-calculus is itself compositional.
%
For readability's sake, we will use some syntactical conventions:

\begin{notation}
We will usually denote $x\cdotT y$ simply
as $xy$.
Moreover, to enhance readability,
let $\TT$ be any constant
$\Tail, \Trunc, \Cond, \flipcoin, \Rec$
of arity $n$,
we indicate $\TT \uT_1\dots \uT_n$
as $\TT(\uT_1,\dots, \uT_n)$.
\end{notation}


$\PORo$ is reminiscent of {PV$^\omega$}
by
Cook and Urquhart~\cite{CookUrquhart};
however an important variation is
the constant
$\flipcoin$,
deonting a function
which
randomly generates either
{$\zzero$} or
{$\oone$}
depending on its input string.
As for $\RS$, our
deduction system does not have $\flipcoin$ specific rules,
because the behavior of this primitive
will be only given by means of a set of axioms
$T_\omega$, depending on a specific $\omega \in \Os$, such that:
$$
\flipcoin(t)\in T_\omega \Leftrightarrow \omega(t)=\one
$$

\begin{defn}[$\flipcoin$-Defining Axioms]
  \label{def:flipaxs}
  Given a function $\omega \in \Os$, we denote with
  $T_\omega$ the set of flip-defining axioms:
  \begin{align*}
    \flipcoin(\zT) &= \omega(\zero)\\
    \flipcoin(\oT) &= \omega(\one)\\
    \flipcoin(\zT\zT) &= \omega(\zero\zero)\\
    \ldots\quad\quad &= \ldots\ .
  \end{align*}
\end{defn}


% \noindent
% We now introduce some abbreviations
% for composed functions:
% \begin{itemize}
% \itemsep0em
%
% % B
% \item $\BT(x):=\Cond(x,\eT,\zT,\oT)$
% indicates the function that
% computes the last digit of $x$,~i.e. coerces $x$ to a Boolean value.
%
% % BNeg
% \item $\BNeg(x):=\Cond(x,\eT,\oT,\zT)$
% indicates the function that computes
% the Boolean negation of $\BT(x)$.
%
% % BOr
% \item $\BOr(x,y):=\Cond(\BT(x), \BT(y), \BT(y),\oT)$
% indicates the function that
% coerces $x,y$ to Booleans and then performs
% the OR-operation.
%
% % BAnd
% \item $\BAnd(x,y) := \Cond\big(\BT(x), \eT, \zT,
% \BT(y)\big)$ indicates the function that coerces
% $x,y$ to Booleans and then performs the AND-operation.
%
% % Eps
% \item $\Eps(x):= \Cond(x,\oT,\zT,\zT)$
% indicates the characteristic function of the
% predicate ``$x=\epsilon$''.
%
% % Bool
% \item $\boolT(x):=\BAnd\big(\Eps(\Tail(x)),\BNeg(\Eps(x))\big)$
% indicates the characteristic function of the predicate
% ``$x= \zero \vee x=
% \one$''.
%
%  % Zero
%  \item $\Zero(x) := \Cond\big(\boolT(x), \zT, \Cond(x,\zT,\zT,\oT),\zT\big)$
%  indicates the characteristic
%  function of the predicate  ``$x=\zero$''.
%
%  % Conc
%  \item $\Conc(x,y)$ indicates the concatenation
%  function
%  defined by the equations below:
%  \begin{align*}
%  \Conc(x,\eT) &:= x \\
%  \Conc(x,y\zT) &:= \Conc(x,y)\zT \\
%  \Conc(x,y\oT) &:= \Conc(x,y)\oT.
%  \end{align*}
%
%  % Eq
%  \item $\Eq(x,y)$ indicates
%  the characteristic function
%  of the predicate  ``$x=y$''
%  and is defined by double recursion by
%  the equations below:
%  \begin{align*}
%  \Eq(\eT,\eT) &:= \oT \\
%  \Eq(\eT, y\zT) = \Eq(\eT, y\oT) &:=\zT \\
%  \\
%  \Eq(x\zT, \eT) &:= \zT \\
%  \Eq(x\zT,y\zT) &:= \Eq(x,y) \\
%  \Eq(x\zT,y\oT) &:= \zT \\
%  \\
%  \Eq(x\oT,\eT) &:= \zT \\
%  \Eq(x\oT,y\zT) &:= \zT \\
%  \Eq(x\oT,y\oT) &:= \Eq(x,y).
%  \end{align*}
%
%  % Times
%  \item $\Times(x,y)$ indicates the function for
%  self-concatenation, $x,y\mapsto x \times y$,
%  and is defined by the equations below:
%  \begin{align*}
%  \Times(x,\eT) &:= \eT \\
%  \Times(x,y\bT) &:= \Conc(\Times(x,y), x),
%  \end{align*}
%  where $\bT\in \{\zT,\oT\}$.
%
%
%
%  % Sub
%  \item $\Sub(x,y)$ indicates {the initial-substring
%  function $x,y\mapsto \emph{S}(x,y)$,}
%  and is defined by bounded recursion as follows:
%  \begin{align*}
%  \Sub(x,\eT) &:= \Eps(x) \\
%  \Sub(x,y\zT) &:= \BOr\big(\Sub(x,y),
%  \Eq(x,y\zT)\big) \\
%  \Sub(x,y\oT) &:= \BOr\big(\Sub(x,y),
%  \Eq(x,y\oT)\big).
%  \end{align*}
% \end{itemize}

















% of $\PORo$.
%
































%%% SUBSECTION
%%% Relating POR and POR\lambda
\subsubsection{Relating $\POR$ and $\PORo$}
\label{subsub:relating}

In this section, we assess that all the functions
within $\POR$ are representable by a term $\tTerm\in \PORo$.
This is done in Theorem \ref{theorem:provRepr} by induction
on the syntactical syntax of the function.
%
To do so, we define an encoding of
strings within $\PORo$ (Definition \ref{def:stringencporo}) and notion of representability
of $\POR$ functions in $\PORo$
(Definition \ref{def:provableRepresentability}).


\begin{defn}
  \label{def:stringencporo}
  For any string $s\in \Ss$, let
  $\ooverline{s} : \sT$
  denote the term of $\PORo$
  corresponding to it,~i.e.:
  %
  \begin{align*}
  \ooverline{\eepsilon}&:=\eT\\
  \ooverline{s\zzero}&:=\ooverline{s}\zT\\
  \ooverline{s\oone}&:=\ooverline{s}\oT.
  \end{align*}
\end{defn}


%%% definition
%%% df:flipcoin
\begin{defn}[Provable Representability]\label{def:provableRepresentability}
For every $f:\Ss^j \times \Os \longrightarrow \Ss$,
we say that a term $\tTerm :\sT \arrowT \dots \arrowT \sT$
of $\PORo$
\emph{provably represents} $f$
if and only if for all strings $s_1,\dots, s_j,s
\in \Ss$,
and $\omega \in \Os$,
$$
f(s_1,\dots, s_n,\omega) = s \ \ \ \Leftrightarrow \ \ \
T_\omega \vdash_{\PORo} \tTerm \ooverline{s_1} \dots
\ooverline{s_j} = \ooverline{s}.
$$
with $T_\omega$ being a $\flipcoin$-defining set of axioms, as described in Definition \ref{def:flipaxs}
\end{defn}
%
%
\noindent
The representability of base functions is almost trivial.
Take, for example, the most peculiar of $\PORo$ base constants:
$\flipcoin$. Its representability is shown in the example below.


%% Example
\begin{ex}\label{ex:flipcoin}
The term $\flipcoin:\sT \arrowT \sT$
provably represents
the query function $\query(x,\omega)=\omega(x)$
of $\POR$,
since for any $s\in \Ss$ and $\omega\in \Os$,
$$
\flipcoin(\ooverline{s})=\ooverline{\omega(s)}\vdash_{\PORo}
\flipcoin(\ooverline{s}) =\ooverline{\query(s,\omega)}.
$$
\end{ex}

\noindent
This observation can be generalized to cover all the base cases
and even the inductive ones,
which are composition and bounded recursion on notation.
This is stated by the first claim of Theorem \ref{theorem:provRepr}.
The second claim guarantees the inverse direction.
An exhaustive proof of this result is out of the scope of this work.
For that reason, we will only take in exam some exemplificative cases of the proof. For further details, see \cite{RBA}.



% Let us show that some of the terms described above
% provably represent the intended functions.
% Let $\func{\Tail}(s,\omega)$ indicate the string obtained by chopping
% the first digit of
% $s$ (with $E(s,\omega)=\eepsilon$) and
% $\func{\Trunc}(s_1,s_2,\omega)=s_1|_{s_2}$ %and
%$$
%\func{cond}(s_1,s_2,s_3,s_4) =
%\begin{cases}
%s_2 \ \ \ &\text{if } s_1 = \eepsilon \\
%s_3 \ \ \ &\text{if } s_1=s' \zzero \\
%s_4 \ \ \ &\text{if } s_1 =s' \oone.
%\end{cases}
%$$






%%% LEMMA
% \begin{lemma}
% {The terms
% $\Tail,\Trunc,\Cond$ provably represent
% the functions $\func{\Tail}, \func{\Trunc}$
% and $\Cf$, respectively.}
% \end{lemma}
%
% \begin{proof}[Proof Sketch]
% For $\Tail$ and $\Cond$, the claim
% follows immediately from the defining
% axioms of the corresponding constants.
% For example, if $s_1=s_2{\zzero}$,
%  then
% $\func{\Tail}(s_1,\omega)=s_2$ {and
% $\ooverline{s_1}=\ooverline{s_2\zzero}=
% \ooverline{s_2}\zT$}.
% Using the defining axioms of $\Tail$:
% $$
% \vdash_{\PORo} \Tail(\ooverline{s_1}) = \Tail(\ooverline{s_2}\zT)
% = \ooverline{s_2}.
% $$
% For $\Trunc$ by double induction
% on two strings
% $s_1,s_2\in \Ss$ we conclude that:
% $$
% \vdash_{\PORo} \Trunc(\ooverline{s_1}, \ooverline{s_2}) =
% \ooverline{s_1 |_{s_2}}.
% $$
% \end{proof}






%% Theorem
%% provably representable
\begin{theorem}\label{theorem:provRepr}
\begin{enumerate}~
\itemsep0em
\item Any function $f\in \POR$ is provably represented
by a term $\tTerm \in \PORo$.
\item For any term $\tTerm \in \PORo$,
there is a function $f\in \POR$ such that
$f$ is provably represented by $\tTerm$.
\end{enumerate}
\end{theorem}

%%% COROLLARY
\begin{cor}
For any function $f: \Ss^j \times \Os \rightarrow \Ss$,
$f\in \POR$ if and only if $f$ is provably represented
by some term $\tTerm :\sT \arrowT \dots \arrowT \sT
\in \PORo$.
\end{cor}















































































%%%%% SUBSECTION
%%%%% THE THEORY IPOR\lambda
\subsection{Realizability of $I\RS$}\label{sec:IPORo}

In this section, we show that for each $I\RS$-representable function
it is possible to extract term $\tTerm$ of $\PORo$.
The theory $I\RS$ is
obtained from the same axioms of $\Lpw$, using an
intuitionist derivation system instead of a
classical logic proof system.
%
To prove this result, we extend $\PORo$'s language to include
first-order quantifiers and propositional connectives,
even $\PORo$'s derivation system is extended with an
induction system and an intuitionist predicate calculus.
The theory thus obtained is called $I\PORo$.
%
The correspondence between $I\PORo$ and $I\RS$ relies on
a proof of realizability \cite{Buss98, CookUrquhart}: by induction on the
proof that $I\RS \vdash F$ we show that we can build an
$I\PORo$ term $\tTerm$ realizing $F$, similarly, by induction on
the fact that $\tTerm$ realizes $F$, we can show that $F$ provable under
$I\RS$'s axioms. As a consequence of the Definition of realizability
given in \cite{RBA},
we get that, if $I\RS\vdash \forall x.\exists ! y F(x, y)$, then the
term $\tTerm$ realizing that formula is such that  $F(x, \tTerm x)$.
This proves the following Theorem.

%%% Corollary
  \begin{theorem}\label{cor:PORTerm}
Let $\forall x.\exists y.A(x,y)$
be a closed theorem of $I\PORo$,
where $A$ is a $\Sigma^b_1$-formula.
Then, there exists a closed term $\tTerm:\sT \arrowT \sT$
of $\PORo$ such that:
$$
\vdash_{I\PORo} \forall x.A(x,\tTerm x).
$$
\end{theorem}












Now, we have all the ingredients to prove that
if a function is $\Sigma^b_1$-representable
in $I\RS$, in the sense of Definition~\ref{df:Sigmab1},
then
it is in $\POR$.





%%% COROLLARY
\begin{cor}\label{cor:taskBmain1}
%Let $(\forall x)(\exists y)A(x,y)$ be a closed theorem of $I\RS$, where $A$ is a $\Sigma^b_1$-formula.
%
For any function $f:\Os \times \Ss \rightarrow \Ss$,
if there is a closed formula $\Sigma^b_1$-formula
$A(x,y)$ in $\Lpw$ such that:
%
%
\begin{enumerate}
\itemsep0em
\item $I\RS \vdash \forall x.\exists !y.A (x,y)$
%
\item $\llbracket A(\overline{\overline{s_1}},\overline{\overline{s_2}})\rrbracket
=\{\omega \ | \ f(\omega,s_1)=s_2\}$,
%
\end{enumerate}
then $f\in\POR$.
\end{cor}

\begin{proof}
Since
$\vdash_{I\RS} \forall x.\exists ! y.A(x,y)$,
also $\vdash_{I\POR} \forall x.\exists! y.A (x,y)$.
From $\vdash_{I\PORo} \forall x.\exists y.A(x,y)$
we deduce $\vdash_{I\PORo} \forall x.A(x,\mathtt{g}x)$
for some closed term
$\mathtt{g}:\sT \arrowT \sT$ of $\PORo$, by
Theorem~\ref{cor:PORTerm} and,
%
by Theorem~\ref{theorem:provRepr},
there is a function $g\in \POR$ such that
for any $\omega \in \Os, s_1,s_2\in \Ss$,
$$
T_\omega \vdash_{I\PORo} A(\ooverline{s_1},\ooverline{s_2})
\ \Leftrightarrow \ g(s_1,\omega)=s_2.
$$
From this we conclude:
\begin{align*}
g(s_1,\omega)=s_2 \ & \Leftrightarrow \ T_\omega \vdash_{I\PORo}
A(\ooverline{s_1},\ooverline{s_2}) \\
&\stackrel{~}{\Leftrightarrow} \
\omega \in \llbracket A(\ooverline{s_1},
\ooverline{s_2})\rrbracket \\
& \Leftrightarrow \ f(s_1,\omega)=s_2.
\end{align*}
So, $f=g$ and, thus,
$f\in \POR$.
\end{proof}

To conclude this the main claim of this chapter,
we show that
%is its extension from intuitionistic
%$I\RS$ to classical $\RS$, showing that
any function
which is $\Sigma^b_1$-representable
in $\RS$ is also $\Sigma^b_1$-representable
in $I\PORo$. Once this result has been accomplished,
Corollary~\ref{cor:taskBmain1}
will allow us to obtain the result we are looking for,
asserting that all the functions which are
also $\Sigma^b_1$-representable
in $I\RS$ can be represented $I\PORo$, too. This representability
result is the claim of Proposition \ref{prop20}.
%
%(and the equivalent generalization
%of $I\PORo$ due to the principle \MP).
Then, it is possible to show that the realizability
interpretation extends to such
$I\PORo$ + \EM{}, so that
for any of its closed theorems
$\forall x.\exists y\preceq \tTerm.A(x,y)$,
with $A$ in $\Sigma^b_1$,
there is a closed term $\tTerm:\sT \arrowT
\sT$ of $\PORo$ such that
$\vdash_{I\POR} \forall x.A(x,\tTerm x)$.
%
%
%
%
%
% \subsubsection{From $I\PORo$ to
% $I\PORo$ + \MP.}
% Let $\EM$ be the Excluded-Middle schema,
% $A\vee \neg A$
% and \emph{Markov's principle} be defined as follows.
% %%% DEFN
% %%% MARKOV's PRINCIPLE
% %\begin{defn}[Markov's Principle]
% %Let \emph{Markov's principle} be the schema below:
% \begin{align*}
% \neg\neg(\exists x)A \rightarrow (\exists x)A, \tag{Markov}
% \end{align*}
% where $A$ is a $\Sigma^b_1$-formula.
% %\end{defn}
%
%
%
%
%
%
%
%
% %%% Proposition
% %%% prop:MP
% \begin{prop}\label{prop:MarkovEM}
% For any $\Sigma^b_1$-formula
% $A$, if $\vdash_{I\PORo+\EM} A$,
% then  $\vdash_{I\PORo + \MP} A$.
% \end{prop}
%
% %
% %
% %
% %
% \noindent
% So, we basically need to show that the
% realizability interpretation defined
% in Section~\ref{sec:realizability} extends
% to $I\PORo$ + \MP, that is for any of
% its closed theorems $(\forall x)(\exists y\preceq \tTerm)
% A(x,y)$, with $A$ in $\Sigma^b_1$,
% there is a closed term $\tTerm:\sT \arrowT \sT$
% of $\PORo$ such that
% $\vdash_{I\PORo} (\forall x)A(x,\tTerm x)$.
%
%
%
%
%
%
%
%
%
%
%
%
%
%
%
%
% \paragraph{From $I\PORo$ to $(I\PORo)^*$}
%
%
% Let us assume given a subjective encoding
% $\enc:(\sT \arrowT \sT)\arrowT \sT$
% {in $I\PORo$ of first-order unary functions as strings,
% together with a ``decoding'' function
% $\app:\sT \arrowT \sT \arrowT \sT$ satisfying}:
% $$
% \vdash_{I\PORo} \app(\enc \fTerm,x)= \fTerm x.
% $$
% Moreover,
% let
% $$
% x*y := \enc\big(\lambda z.\BAnd\big(\app(x,z),\app(y,z)\big)\big)
% $$
% and
% $$
% T(x) := (\exists y)\big(\BT(\app(x,y))=\zT\big).
% $$
% There is a \emph{meet semi-lattice} structure
% on the set of terms of type $\sT$
% defined by
% $
% \tTerm \sqsubseteq \uT
% $
% iff
% $
% \vdash_{I\PORo} T(\uT)\rightarrow
% T(\tTerm)
% $
% with top element
% $\topO = \enc(\lambda x.\oT)$
% and meet given by $x*y$.
% %
% %
% Indeed, from $T(x*\oT) \leftrightarrow T(x)$,
% $x\sqsubseteq \topO$ follows.
% Moreover, from
% $\BT(\app(x,\uT))=\zT$, we obtain
% $\BT\big(\app(x*y,\uT)\big)=
% \BAnd\big(\app(x,\uT),\app(y,\uT)\big)=\zT$,
% whence $T(x)\rightarrow T(x*y)$,~i.e.
% $x*y\sqsubseteq x$.
% One can similarly prove $x*y\sqsubseteq y$.
% Finally, from $T(x)\rightarrow T(v)$
% and $T(y)\rightarrow T(v)$,
% we deduce $T(x*y)\rightarrow T(v)$,
% by observing that
% $\vdash_{I\PORo} T(x*y)\rightarrow T(y)$.
% %
% Notice that the formula $T(x)$ is \emph{not} in
% $\Sigma^b_1$, as its existential
% quantifier is not bounded.
%
%
%
%
%
% %%% DEFINITION
% \begin{defn}
% For any formula $A$ {of $I\PORo$}
% and
% fresh variable $x$,
% we define formul\ae{}
% $x\Vdash A$ inductively, as follows:
% \begin{align*}
% x \Vdash A &:= A\vee T(x) \ \ \ \ \ \ \ \ \  (A \ atomic) \\
% %
% x \Vdash B\wedge C &:= x \Vdash B \wedge x
% \Vdash C \\
% %
% x\Vdash B\vee C &:= x \Vdash B\vee x \Vdash C \\
% %
% %x \Vdash \neg A &:= (\forall y)\big(y\Vdash A \rightarrow T(x*y)\big) \\
% %
% x\Vdash B \rightarrow C &:=
% (\forall y)\big(y\Vdash B \rightarrow x*y \Vdash C\big) \\
% %
% x \Vdash (\exists y)B &:=
% (\exists y)x \Vdash B \\
% %
% x \Vdash (\forall y)B &:=
% (\forall y)x \Vdash B.
% \end{align*}
% \end{defn}
% %
% %
% \noindent
% %Then, the following lemmas are proved as in~\cite{CoquandHofmann}.
% %
% The following Lemma~\ref{lemma21} is
% then
% established by induction on the structure
% of formul\ae{} in $I\PORo$.
%
%
%
%
% % Lemma
% \begin{lemma}\label{lemma21}
% If $A$ is provable in $I\PORo$ without using
% $\NP$-induction,
% then $x\Vdash A$ is provable in $I\PORo$.
% \end{lemma}
% %\begin{proof}
% %Immediate by induction.
% %\end{proof}
%
%
%
%
% % Lemma
% \begin{lemma}\label{lemma22}
% Let $A=(\exists x\preceq \tTerm)B$,
% where $B$ is a $\Sigma^b_0$-formula.
% Then, there exists a term
% $\uT : \sT$, with $FV(\uT_A)=FV(B)$,
% such that:
% $$
% \vdash_{I\PORo} A \leftrightarrow T(\uT_A).
% $$
% \end{lemma}
% \begin{proof}
% Since $B(x)$ is a $\Sigma^b_0$-formula,
% for all terms $\vTerm :\sT$,
% $\vdash_{I\PORo} B(x) \leftrightarrow \tTerm_{x\preceq
% \tTerm \wedge B}(x)=\zT$
% (where $t_{x\preceq t\wedge B}$
% has the free variables of $t$ and $B$).
% %
% Let $C(x)$ be a $\Sigma^b_0$-formula,
% one can show by induction on its structure
% that
% for all term $\vTerm:\sT$, $\tTerm_{C(\vTerm)}=\tTerm_C(\vTerm)$.
% %
% Then,
% $$\vdash_{I\PORo} A\leftrightarrow
% (\exists x)\tTerm_{x\preceq \tTerm
% \wedge B}(x)=\zT \leftrightarrow
% (\exists x)T(\enc(\lambda x.\tTerm_{x\preceq \tTerm \wedge B}(x))).
% $$
% So, we let $\uT_A=\enc(\lambda x.t_{x\preceq \tTerm \wedge B}(x))$.
% \end{proof}
% %
% %
% %
% %
% %
% %
% %
% %% Lemma
% %% Lemma 8
% %\begin{lemma}
% \noindent
% From which we obtain the following three
% properties:
% \begin{itemize}
% \itemsep0em
% \item[i.] $\vdash_{I\PORo} (x\Vdash A) \leftrightarrow
% (A\vee T(x))$
% \item[ii.] $\vdash_{I\PORo} (x\Vdash \neg A) \leftrightarrow
% (A\rightarrow T(x))$
% \item[iii.] $\vdash_{I\PORo} (x\Vdash \neg\neg A) \leftrightarrow
% (A\vee T(x))$.
% \end{itemize}
% where $A$ is a $\Sigma^b_1$-formula.
% %\end{lemma}
%
%
% %%% COROLLARY
% %%% MARKOV's PRINCIPLE
% \begin{cor}[Markov's Principle]\label{cor:Markov}
% If $A$ is a $\Sigma^b_1$-formula,
% then
% $$
% \vdash_{I\PORo} x\Vdash \neg\neg A \rightarrow A.
% $$
% \end{cor}
%
%
%
%
%
%
%
%
%
%
%
%
%
%
%
%
%
%
%
%
%
% Then, to define the extension $(I\PORo)^*$
% of $I\PORo$, we start by formally introducing
% \PIND.
%
% \begin{defn}[\PIND]
% Let \PIND(A) indicate the formula:
% $$
% \big(A(\eT) \wedge \big((\forall x)(A(x) \rightarrow
% A(x\zT)) \wedge (\forall x)(A(x) \rightarrow A(x\oT))\big)
% \rightarrow (\forall x)A(x).
% $$
% \end{defn}
% %
% %
% \noindent
% Observe that  if
% $A(x)$ is a formula of the form
% $(\exists y\preceq \tTerm)\uT=\vT$, then
% the formula
% $z\Vdash$ \PIND$(A)$
% is of the form
% \PIND$\big(A(x) \vee T(z)\big)$,
% which is \emph{not}
% an instance of the $\NP$-induction
% schema (as the formula
% $T(z)=(\exists x)\BT\big(\app(z,x)\big)=\zT$
% is not bounded).
% %
% %
% %
% %
% %
% %
% \begin{defn}[The Theory $(I\PORo)^*$]
% Let ($I\PORo$)* indicate the theory
% extending $I\PORo$
% with all instances of the induction
% schema \PIND$\big(A(x)\vee B\big)$,
% where $A(x)$ is of the form
% $(\exists y\preceq \tTerm)\uT=\mathsf{v}$,
% and $B$ is an arbitrary formula
% with $x\not\in FV(B)$.
% \end{defn}
% %
% %
% \noindent
% {From the discussion above} we deduce:
%
%
% %%% PROPOSITION
% \begin{prop}
% For any $\Sigma^b_1$-formula
% $A$, if $\vdash_{I\PORo} A$,
% then $\vdash_{(I\PORo)^*} x\Vdash A$.
% \end{prop}
% %
% %
% %
% \noindent
% Then, we can extend the realizability
% interpretation of Section~\ref{sec:realizability}
% to $(I\PORo)$ by simply constructing a realizer
% for \PIND$(A(x) \vee B)$.
%
% %% Lemma
% \begin{lemma}\label{lemma26}
% Let $A(x)=(\exists y\preceq \tTerm)\uT=\zT$
% and $B$ be any formula not containing
% free occurrences of $x$.
% Then, there exist terms $\termO$
% such that:
% $$
% \vdash_{I\POR} \termO \realize \emph{\PIND}(A(x)
% \vee B).
% $$
% \end{lemma}
%
% %\begin{proof}
% %Let $D(x)=A(x)\vee B$.
% %It suffices to construct a realizer of:
% %$$
% %\big(D(\eT) \wedge (\forall x)(A(x) \rightarrow D(x\zT)) \wedge (\forall x)(A(x) \rightarrow D(x\oT))\big) \rightarrow (\forall x)D(x).
% %$$
% %Let us assume that:
% %\begin{itemize}
%
% %\item $\varO_1 \realize D(\eT)$, that is $\varO_1=x_1^0x_1^1\varO^2_1$, where
% %$$
% %\big(x^0_1 = \zT \wedge x^1_1 \preceq \tTerm[\eT/x] \wedge \uT[\eT/x, x^1_1/y] = \zT\big) \vee \big(x^0_1 = \oT \wedge \varO^2_1 \realize B\big).
% %$$
%
%
% %%
% %\item $\varO_2\realize (\forall x)(A(x) \rightarrow D(x\zT))$, where $\varO_2 = x_2^0x_2^1\varO^2_2$ and under the assumption $y\preceq t$ and $u=\zT$,
% %$$
% %\big(x^0_2 xy=\zT \wedge x^1_2 xy \preceq \tTerm [y\zT/x] \wedge \uT[y\zT/x, x^1_2 xy/y] = \zT\big) \vee \big(x^0_2 xy=\oT \wedge \varO^2_2 xy \realize B\big).
% %$$
%
%
%
% %%%
% %\item $\varO_3 \realize (\forall x)(A(x) \to D(x\oT))$, where $\varO_3=x^0_3x^1_3\varO_3^2$ and under the assumption $y\preceq \tTerm$ and $\uT=\zT$,
% %$$
% %\big(x^0_3xy = \zT \wedge x^1_3xy \preceq \tTerm[y\oT/x] \wedge \uT[y\oT/x, x^1_3xy/y] = \zT\big) \vee \big(x^0_3xy=\oT \wedge \varO^2_3xy \realize B\big).
% %$$
% %\end{itemize}
% %
% %
% %
% %We will construct a term $\uT:\sT\to\sT$ satisfying the following specifications:
% %\begin{itemize}
% %\itemsep0em
%
% %% a.
% %\item[a.] $\vdash_{I\RS} \Eps(\uT x)=\zT$
%
%
% %% b.
% %\item[b.] If $\BT(\uT x)=\zT$, then $\Tail(\uTerm x)\realize A(x)$
%
%
% %% c.
% %\item[c.] If $\BT(\uT x)=\oT$, then
% %$$
% %x^0_1 = \oT \vee \big( y_0 \preceq \tTerm\{x_0/y\} \wedge \uT\{x_0, y_0/x\} = \zT \wedge \BOr(\varO^2_2 x_0y_0, \varO^2_3 x_0y_0)=\oT\big)
% %$$
% %where $x_0=\Tail(\uT x)$ and $y_0=\Tail(x_0)$.
% %\end{itemize}
% %Notice that the three conditions above can all be specified by way of an equation of the form $v[\uTerm,x]=\zT$.
% %
% %Indeed, if such a term exists, then we can define  $\termO=\tTerm_1\tTerm_2\tTerm_3$ realizing $A(x)\vee B$ by
% %\begin{align*}
% %\tTerm_1x &:= \BT(\uT x) \\
% %\tTerm_2x &:= \Tail(\uT x) \\
% %\tTerm_3x &:= \Cond\big(x^0_1,\eT, \varO^2_1, \BOr(\varO^2_2 x_0 y_0, \varO^2_3 x_0y_0)\big).
% %\end{align*}
% %We can define $\uT$ by bounded recursion as follows:
% %\begin{align*}
% %\uT\eT &:= \Cond(x^0_1, \eT, x^1_1 \zT, \oT) \\
% %
% %\uT(x\zT) &:= \Cond\big(\uT x, \eT, \Cond(x_2^0 x(\Tail(\uT x)), \eT, x^1_2 x(\Tail(\uT x))\oT), \Cond(x^0_1,\eT,\eT,\oT)\oT\big) \\
% %
% %\uT(x\oT) &:= \Cond\big(\uT x, \eT, \Cond(x^0_2x(\Tail(\uT x)), \eT, x^1_2 x(\Tail(\uT x))\zT, \varO^2_2 x(\Tail(\uT x))\oT), \Cond(x^0_1, \eT, \eT, \oT)\oT\big).
% %\end{align*}
% %
% %
% %We show that $\Gamma\vdash_{I\RS} \vT[\uT, x]=\zT$, where $\Gamma=\varO_1 \realize  D(\eT), \varO_2 \realize (\forall x)(A(x) \rightarrow D(x\zT)), \varO_3 \realize (\forall x)(A(x) \rightarrow D(x\oT))$, by $\NP$-induction:
% %
% %
% %
% %
% %\begin{itemize}
% %\itemsep0em
%
%
%
%
% %
% %\item $\vT(\uT, \eT) =\zT$:
% %\begin{itemize}
% %\itemsep0em
% %\item[a.] From $\varO_1\realize D(\eT)$ it can be deduced that $x^0_1 = \zT \vee x^0_1 = \oT$.
% %Using this fact, together with $\Eps(\uT\eT)=\Eps(\Cond(x^0_1, \eT, x^1_1\zT,\oT)=\zT$, the claim is proved due to the defining axioms of $\Cond$.
%
% %\item[b.] $\BT(\uT \eT)=\zT \rightarrow \uT \eT =x^1_1 \zT$, whence $\Tail(\uT \eT)=x^1_1$ which, by hypothesis, realizes $A(\eT)$.
%
% %\item[c.] $\BT(\uT\eT)=\oT \rightarrow x^0_1=\oT$.
% %\end{itemize}
%
%
%
%
%
%
% %\item $\vT[\uT, x]=\zT \rightarrow \vT[\uT,x\zT]=\zT$.
% %Suppose $\vT[\uT,x]=\zT$ holds, i.e. a.-c. hold for $x$:
% %\begin{itemize}
%
%
% %\item[a.] From $x^0_1=\zT \vee x^0_1=\oT$ and the fact that $x^0_1 =\zT \vdash_{I\PORo} \BT(\uT(x\zT)) =\zT$ and $x^0_1 =\oT \vdash_{I\PORo} \BT(\uT(x\zT))=\oT$, we deduce $\BT(\uT\rm(x\zT))=\zT \vee \BT(\uT(x\zT))=\oT$, which implies $\Eps(\uT(x\zT))=\zT$.
%
% %
% % b.
% %\item[b.] $\BT(\uT\zT))=\zT$ implies $\uT x=\zT$ together with
% %$$
% %\Tail(\uT(x\zT))=\Cond(x^0_2x (\Tail(\uT x)), \eT, x^1_2 x(\Tail(\uT x))\zT, \varO^2_1 x(\Tail(\uT x))\oT) x^0_2(\Tail(\uT x))=\zT.
% %$$
% %From $\vT[\uT, x]$ and $\uT x=\zT$, we deduce $\Tail (\uT x)\realize A(x)$ and, thus, by $\varO_2\realize (\forall x)(A(x) \rightarrow D(x\zT))$, also $\varO_2x(\Tail(\uT x)) \realize D(x\zT)$.
% %From $x^0_2 x (\Tail (\uT x))=\zT$, we conclude  $\Tail(\uT(x\zT))= x^1_2 x(\Tail(\uTerm x))\realize A(x\zT)$.
%
%
%
% %
% % c.
% %\item[c.] If $\BT(\uT(x\zT))=\oT$, there are two possible cases:
% %\begin{itemize}
% %\item $\BT(\uT x)=\oT$ and $x^0_1=\oT$.
%
%
% %\item $\BT(\uT x)=\zT$ and $x^0_2x(\Tail(\uT x))=\oT$.
% %Then, by $\vT[\uT, x]$, b. we deduce that $\Tail(\uT x)\realize A(x)$.
% %So, by $\varO_2 \realize (\forall x)(A(x) \rightarrow D(x\oT))$, also $\varO_2x(\Tail(\uT x)) \realize D(x\zT)$.
% %Moreover, from $x^0_2 x(\Tail(\uT x))=\oT$, we conclude $\Tail (\uT(x\zT))= \varO^2_2 x(\Tail(\uT x)) \realize B$.
% %\end{itemize}
% %\end{itemize}
%
%
%
% %\item $\vT[\uT, x]=\zT \to \vT[\uT, x\oT]=\zT$.
% %This is proved similarly to the previous case.
% %\end{itemize}
%
% %\end{proof}
% %
% %
% %
% \noindent
% So, by Theorem~\ref{theorem:realSound},
% we obtain immediately, for any
% $\Sigma^b_1$-formula $A$ and
% formula $B$,
% with$x\not\in FV(A)$,
% $$
% \vdash_{I\PORo} \emph{\PIND}(A(x) \vee B).
% $$
%
%
%
%
%
%
%
%
%
%
%
% %%%% PROPOSITION
% %\begin{prop}
% %For any $\Sigma^b_1$-formula $A$ and formula $B$, with $x\not\in FV(A)$, $\vdash_{I\PORo} \PIND(A(x)\vee B)$.
% %\end{prop}
%
%
%
% \begin{cor}[$\forall\NP$-Conservativity
% of $I\PORo$ + \EM \ over $I\PORo$]
% {
% Let $A$ be a $\Sigma^b_1$-formula,
% if $\vdash_{I\PORo+\EM} (\forall x)(\exists y\preceq \tTerm)A(x,y)$
% then $\vdash_{I\PORo} (\forall x)(\exists y\preceq \tTerm)A(x,y)$.}
% \end{cor}
%
%
%
%
%
%
% \paragraph{Concluding the Proof.}
%
% We can finally conclude proving
% Proposition~\ref{prop20}.
%
%
In doing so, we first introduce the theory
$I\PORo$ + \EM, where \EM{} is the Excluded Middle rule:
$$
\vdash_{I\PORo + \EM} A \lor \lnot A.
$$
Thus, we prove that a representability theorem
in the style of Theorem \ref{cor:PORTerm} relating this
class to $I\PORo$, this is done in Proposition \ref{prop20}.


%% Proposition
\begin{prop}\label{prop20}
Let $\forall x.\exists y\preceq \tTerm. A(x,y)$
be a closed theorem of $I\PORo$ + \EM,
where $A$ is a $\Sigma^b_1$-formula.
Then, there exists a closed term
$\tTerm:\sT\arrowT \sT$ of $\PORo$ such that:
$$
\vdash_{I\PORo} \forall x.A(x,\tTerm x).
$$
\end{prop}
\noindent
Even in this case, a comprehensive and exhaustive proof is out of the scope of this chapter. The reader who is interested in further details can consult \cite{RBA}.


% \begin{proof}
% If $I\PORo$ + \MP \ proves
% $(\forall x)(\exists y)A(x,y)$, then
% by Parikh's Proposition~\ref{prop:Parikh},
% it also proves $(\exists y\preceq \tTerm)A(x,y)$
% and $(I\PORo)^*$
% proves $z\Vdash (\exists y\preceq \tTerm)A(x,y)$.
% Let $B:=(\exists y\preceq \tTerm)A(x,y)$.
% By taking $\mathsf{z}=\uT_C$,
% using Lemma~\ref{lemma22},
% we deduce $\vdash_{(I\PORo)^*}B$
% and thus, by Lemma~\ref{lemma21} and
% Lemma~\ref{lemma26},
% we deduce that there exist
% $\termO,\termT$
% such that
% $\vdash_{I\PORo} \termO,\termT\realize
% B$,
% which implies
% $\vdash_{I\PORo} A(x,\termO x)$.
% Thus,
% $\vdash_{I\PORo} (\forall x)(A(x), \termO x)$.
% \end{proof}


Then, we can observe than, since $I\PORo$ + \EM{} contains
$I\RS$ plus the Excluded Middle law, then all the theorems of
$\RS$ are in  $I\PORo$ + \EM, too.
From this we conclude our proof arguing
as for Corollary~\ref{cor:taskBmain1}.
%
%
%
%\newpage
%
%
%%% THEOREM
%\begin{theorem}
%{Let $\vdash_{I\POR + \emph{\EM}}
%(\forall x)(\exists y\preceq \tTerm)A(x,y)$,
%where
%$A$ is a $\Sigma^b_1$-formula with
%only $x,y$ free.}
%Then, there is a closed term
%$\tTerm:\sT \arrowT \sT$ of $\PORo$ such that:
%$$
%\vdash_{I\PORo} (\forall x)A(x,\tTerm x).
%$$
%\end{theorem}
%\noindent
%From which we deduce the following
%Corollary~\ref{cor:main2}, by arguing as
%for Corollary~\ref{cor:taskBmain1}.



%% COROLLARY
\begin{cor}\label{cor:main2}
Let $\RS \vdash \forall x.\exists y\preceq t.A(x,y)$,
where $A$ is a $\Sigma^b_1$-formula
with only $x,y$ free.
For any function $f:\Ss\times \Os \rightarrow \Ss$,
if $\forall x.\exists y\preceq t. A(x,y)$
represents $f$ so that:
\begin{enumerate}
\itemsep0em
\item $\RS \vdash \forall x.\exists !y.A(x,y)$
\item $\llbracket A(\ooverline{s_1},\ooverline{s_2})\rrbracket =
\{\omega \ | \ f(s_1,\omega)=s_2\}$,
\end{enumerate}
then $f\in \POR$.
\end{cor}
%\noindent
%This proof is obtained by adapting the method from~\cite{CoquardHofmann}.
\noindent
As we mentioned in the introduction of this chapter, Corollary \ref{cor:main2}, together with Theorem \ref{theorem1}, completes the proof of Theorem \ref{thm:TaskAB}.
