

\newtheorem{defn}{Definition}
\newtheorem{conj}{Conjecture}
\newtheorem{remark}{Remark}
\newtheorem{notation}{Notation}
\newtheorem{prop}{Proposition}
\newtheorem{ex}{Example}
\newtheorem{lemma}{Lemma}
\newtheorem{cor}{Corollary}
\newtheorem{theorem}{Theorem}


\newcommand{\map}{\longrightarrow}


\newcommand{\res}{\mathit{res}}
\newcommand{\les}{\mathit{les}}

\newcommand{\reverse}{\mathit{reverse}}
\newcommand{\ras}{\mathit{ras}}
\newcommand{\las}{\mathit{las}}
\newcommand{\rrs}{\mathit{rrs}}
\newcommand{\lrs}{\mathit{lrs}}

\newcommand{\rel}{\mathit{rel}}
\newcommand{\lel}{\mathit{lel}}
\newcommand{\ral}{\mathit{ral}}
\newcommand{\lal}{\mathit{lal}}
\newcommand{\rrl}{\mathit{rrl}}
\newcommand{\lrl}{\mathit{lrl}}
\newcommand{\srrl}{\mathit{srrl}}
\newcommand{\simulate}{\mathit{sim}}
\newcommand{\concat}{\mathit{concat}}
\newcommand{\Succ}{\Sf}
\newcommand{\pd}{\mathit{pd}}
\newcommand{\Cf}{\mathit{C}}
\newcommand{\op}{\mathit{op}}
\newcommand{\mult}{\mathit{mult}}
\newcommand{\sa}{\mathit{sa}}
\newcommand{\Lpw}{\mathcal L}
\newcommand{\rmsep}{\mathit{rmsep}}
\newcommand{\apply}{\mathit{apply}}
\newcommand{\dectape}{\mathit{dectape}}






\newcommand{\PPT}{\mathcal{PPT}}
\newcommand{\SFP}{\mathcal{SFP}}

\newcommand{\POR}{\mathcal{POR}}
\newcommand{\PTCA}{\mathbf{PTCA}}
\newcommand{\Ss}{\mathbb{S}}
\newcommand{\Os}{\mathbb{O}}
\newcommand{\Bool}{\mathbb{B}}
\newcommand{\Nat}{\mathbb{N}}
\newcommand{\Qs}{\mathbb{Q}}
\newcommand{\Vv}{\mathbb{V}}
\newcommand{\Ll}{\mathbb{L}}
\newcommand{\Tt}{\mathbb{T}}
\newcommand{\POLY}{\mathsf{POLY}}

\newcommand{\zero}{\mathbf{0}}
\newcommand{\one}{\mathbf{1}}

\newcommand{\eepsilon}{\epsilon}
\newcommand{\bool}{\mathbf{b}}

\newcommand{\blank}{\circledast}
\newcommand{\odd}{\mathit{odd}}
\newcommand{\even}{\mathit{even}}
\newcommand{\eq}{\mathit{eq}}

\newcommand{\Cyl}{Cyl}

\newcommand{\PCyl}{Pcyl}
\newcommand{\NCyl}{Ncyl}
\newcommand{\cyl}{cyl}
\newcommand{\CylP}{\Cyl P}

\newcommand{\rt}{\mathbf{rt}}

\newcommand{\meas}{\nu}

\newcommand{\sig}{\m{\sigma}}
\newcommand{\stmreach}{\lvert\triangleright}
\newcommand{\stmtrans}{\Vdash}





\newcommand{\longv}[1]{}




\newcommand{\m}[1]{\textcolor{blue}{#1}}
\newcommand{\g}[1]{\textcolor{teal}{#1}}





%%% NEW
\newcommand{\fcob}{\mathcal{PF}_\Ss}
\newcommand{\polyF}{\mathcal{PTF}_{\Ss}}

\newcommand{\cob}[1]{{#1}_{\mathcal{D}}}



\newcommand{\ovverline}[2]{{\underline {#1}}_{#2}}
\newcommand{\porif}{f_{\mathit{if}}}
\newcommand{\Sf}{S}
\newcommand{\listenc}[2]{{\langle #1\rangle}^n_\Ll}


\newcommand{\doubling}{doub}
\newcommand{\halving}{halv}

\newcommand{\tmstep}{\vdash}
\newcommand{\tmreach}{\triangleright}


\newcommand{\dyad}{\emph{dyad}}
\newcommand{\dy}{\emph{dy}}

\newcommand{\ext}{ext}

\addbibresource{bib.bib}

\title{Task C}
\author{}
\date{} % delete this line to display the current date

%%% BEGIN DOCUMENT
\begin{document}

\maketitle
\tableofcontents

%\section{Task C}



%%% CLASS PPT
\begin{defn}[Class $\PPT$]
The \emph{class $\PPT$} is the class of functions
from $\Ss$ to $\mathbb D(\Ss)$
which are computable by a PTM in a polynomial
number of steps.
\end{defn}
%
%
\noindent
We show that for each $f\in \PPT$,
the probability that $f(x)=y$ is equal to the measure
of the set of $\omega$ in the
corresponding function of $\POR$,
returning $y$ on input $x$, and vice-versa.
%
Formally,
%
% Conjecture
\begin{conj}\label{conj1}
\begin{enumerate}
\itemsep0em
\item[i)] For each $f\in \PPT$, there is $g \in \POR$
such that:
$$
\forall x, y. \emph{Pr}\big(f(x)=y\big) = \mu\big(\{
\omega \in \Os \ | \ g(x,\omega)=y\}\big).
$$

\item[ii)] For each $g\in \POR$, there is an $f\in \PPT$
such that:
$$
\mu\big(\omega \in \Os \ | \ g(x,\omega) = y\}\big)
= \emph{Pr}\big( f(x) =y \big).
$$
\end{enumerate}
\end{conj}
%
% Proof Sketch
\begin{proof}[Proof Sketch]
First, we introduce an auxiliary class of functions,
$\SFP$, which is strongly related to $\PPT$,
but based on a source of randomness close
to that of $\POR$, Section~\ref{sec:SFP}.
%
Then, we actually relate the class $\SFP$
with $\POR$, Section~\ref{sec:SFPtoPOR},
and vice-versa, Section~\ref{sec:PORtoSFP}.
\end{proof}













%%%%%% SECTION
%%%%%% THE CLASS SFP
\section{The Class $\SFP$}\label{sec:SFP}
In the perspective of Conjecture~\ref{conj1},
probabilistic Turing machines (PTM, for short)
are not a suitable computational model.
But they are anyway needed to prove our claim,
because the definition of the $\PPT$ class
relies on these machines. Moreover,
$\BPP$ itself is defined on top of these machines.



%%% DEFINITION PTM
\begin{defn}[Probabilistic Turing Machines \cite{AroraBarak}]
A \emph{probabilistic Turing machine}
is a Turing machine with two transition functions: $\delta_0,\delta_1$.
%
Given an input $x$, the PTM chooses at each step
to apply, with probability $\frac{1}{2}$,
either $\delta_0$ or $\delta_1$.
%
The choice is independent from all the previous ones.
\end{defn}
%
%
%
\noindent
Indeed, the source of randomness in a PTM is ``implicit'',
as coming from a uniform distribution of
probabilities, and the machine
outputs distributions of probabilities over strings,
(not strings).
%
On the other hand, in $\POR$,
randomness is enucleated by the oracle function,
$Q$
%$\omega : \Ss \map \Bool$, and
%its functions output strings, not probability distributions.
returning strings (not probability distributions).


To bridge this gap %between $\PPT$ and $\POR$,
we introduce the \emph{class $\SFP$}
of functions which are computable by
stream machines in polynomial-time.
%
This class is based on
a better-fitting computational model,
namely \emph{stream Turing machines}
(STM, for short).
%
In STMs, the source of randomness
consists in an ``explicit'', read-only
random tape.
%
More precisely, these machines can be defined
as ordinary
TMs with $k+1$ tapes,
with one of them designated as a read-only \emph{oracle},
which stores an infinite stream of characters to
be read from left to right.
%
These machines are a valuable intermediate
model, linking $\POR$ and $\PPT$.
%
On the one hand, they use the oracle tape as an \emph{explicit}
source or randomness,
that can be represented
as a function $\eta:\Nat \map \Bool$,
and the behavior of which
is consistent with that of
the query functions and $\omega$s in $\POR$.
%
On the other hand, STMs share many features with
multi-tape TMs.\footnote{Conversely,
we can see
a TM as special STM, which basically ``ignores'' the
oracle tape.}
%
Thus, many well-known results
about the latter model still hold
%
Intuitively, the class of functions computed by STM
is almost equivalent to that of the ones computable
by PTMs.



%%% DEFN Stream Turing Machine
\begin{defn}[Stream Turing Machine]\label{df:streamMachine}
A \emph{stream Turing machine} is a quadruple
$M:= \langle \m{\Qs}, \m{q_0}, \m{\Sigma}, \m{\delta} \rangle$, where:
\begin{itemize}
\itemsep0em
\item $\m{\Qs}$ is a finite set of states ranged over by
$\m{q_i}$ and similar meta-variables.
%
\item $\m{q_0} \in \m{\Qs}$ is an initial state
%
\item $\m{\Sigma}$ is a finite set of characters
ranged over by $\m{c_i}$ \emph{et simila}.
%
\item $\m{\delta}: \m{\hat{\Sigma}}
\times \m{\Qs} \times \m{{\hat{\Sigma}}} \times
\Bool
\map \m{\hat{\Sigma}} \times \m{\Qs} \times \m{\hat{\Sigma}} \times \{\m{L},\m{R}\}$
is a transition function describing the new configuration
reached by the machine,
\end{itemize}
where  \m{$L$} and \m{$R$} are two fixed and distinct symbols,
e.g.~\m{$\zero$} and \m{$\one$},
$\m{\hat{\Sigma}}=\m{\Sigma} \cup \{\m{\blank}\}$
and \m{$\blank$} represents the
\emph{blank character}, such that $\m{\blank} \not \in
\m{\Sigma}$.
\end{defn}
%
\noindent
Without loss of generality,
we can assume
$\m{\Sigma}=\{\m{\zero}, \m{\one}\}$,
and define the canonical STM as follows:




%%% DEFN
%%% Canonical STM
\begin{defn}[Canonical STM]\label{df:canonicalSTM}
A \emph{canonical stream machine} is an STM,
such that
$\m{\Sigma}=\{\m{\zero}, \m{\one}\}$,
%$\m{L}=\m{\zero}$, and $\m{R}=\m{\one}$.
\end{defn}


%%% Notation
%\begin{notation}
%In the following, let us use $\m{c}_{|\m{\Sigma}|+1}$ to denote the blank character $\m{\blank}$.
%\end{notation}



The \emph{configuration} of a ordinary TM is
a tuple which keeps track of the current state
and some strings representing the state of the
machine tape(s).
%
The portion of the oracle tape
which has not been
queried yet is represented by means of a function
$\m{\eta} : \Nat \map \Bool$.


%%% DEFINITION
%%% CONFIGURATION
\begin{defn}[Configuration of STM]\label{df:STMConfiguration}
The \emph{configuration of an STM}
is a quadruple $\langle \m{\sigma},
\m{q}, \m{\tau}, \m{\eta}\rangle$,
where:
\begin{itemize}
\itemsep0em
%
\item $\m{\sigma} \in \hat{\Sigma}^*$
is the portion of the work tape on the left of the head
%
\item $\m{q}\in A$ is the current state of the machine
%
\item $\m{\tau} \in \hat{\Sigma}$ is the portion of the
work tape on the right of the head
%
\item $\m{\eta} \in \Bool^\Nat$ is the portion
of the oracle tape that has not been read yet.
\end{itemize}
\end{defn}
%
%
\noindent
At each step, the machine queries a new value
on the oracle tape.
%
Shifting on the work tape is naturally defined by
pre-fixing and post-fixing characters to strings,
so formalized by a shifting operation
between the function $\m{\eta}$
and a string $\m{\sigma}$.



%%% DEFINITION
%%% Shifting Operation
\begin{defn}[Shifting Operation]\label{df:shifting}
Given $n\in \Nat$, $\m{\sigma} \in \Ss$, and
$\m{\eta}:\Nat \map \Bool$,
we define the \emph{shifting of $\m{\eta}$
by %prefix
$\m{\sigma}$},
$\m{\sigma \eta}(\cdot)$,
by induction on the structure of $\m{\sigma}$:
\begin{align*}
(\m{\eepsilon} \m{\eta})(n) &:= \m{\eta}(n) \\
%
(\m{\bool \tau}) \m{\eta}(n) &:= \begin{cases}
\m{\bool} \ \ \ &\text{if } n=0 \\
(\m{\tau \eta})(n-1) \ \ \ &\text{otherwise.}
\end{cases}
\end{align*}
\end{defn}


The dynamics of STMs is defined
as predictable by extending the notion of standard
transition function in the natural way.


%%% DEFINITION
%%% STM Transition Function
\begin{defn}[STM Transition Function]\label{df:STMTransition}
Given an STM, $M=\langle
\m{\Qs}, \m{q}, \m{\Sigma}, \m{\delta}\rangle$,
we define the \emph{partial transition function}
$\vdash_{\delta} \hat{\Sigma}^* \times
Q \times \hat{\Sigma}^* \times
\Bool^\Nat \map
\hat{\Sigma}^* \times
Q \times \hat{\Sigma}^*
\times \Bool^\Nat$
between two configurations as:
%
\begin{align*}
\langle \m{\sigma}, \m{q}, \m{c\tau}, \m{\zero\eta}\rangle
%
\stmtrans_{\delta}
%
\langle \m{\sigma c'}, \m{q'}, \m{\tau}, \m{\eta}\rangle
%
\ \ \ \ \ &\text{if} \ \m{\delta}(\m{q},\m{c},\m{\zero}) =
\langle \m{q'}, \m{c'}, \m{R}\rangle \\
%
%
%
\langle \m{\sigma c_0},
\m{q}, \m{c\tau},
\m{\zero\eta \rangle}
\stmtrans_{\delta}
\langle \m{\sigma},
\m{q'}, \m{c_0c_1'\tau},
\m{\eta}\rangle
%
\ \ \ \ \ &\text{if} \ \m{\delta}(\m{q}, \m{c_1},
\m{\zero}) = \langle
\m{q'}, \m{c_1'}, \m{L}\rangle \\
%
%
%
\langle \m{\sigma}, \m{q}, \m{c\tau}, \m{\one\eta}\rangle
%
\stmtrans_{\delta}
\langle \m{\sigma c'}, \m{q'},
\m{\tau}, \m{\eta}\rangle
%
\ \ \ \ \ &\text{if } \m{\delta}(\m{q}, \m{c}, \m{\one})
= \langle \m{q'}, \m{c'}, \m{R}\rangle \\
%
\langle \m{\sigma c_0}, \m{q}, \m{c_1\tau}, \m{\one \eta}\rangle
\stmtrans_{\delta}
\langle \m{\sigma}, \m{q'}, \m{c_0c_1' \tau},
\m{\eta}\rangle
%
\ \ \ \ \ &\text{if }
\m{\delta}(\m{q},\m{c_1},\m{\one}) = \langle
\m{q'},\m{c_1'},\m{L}\rangle.
%
\end{align*}
%
\end{defn}
%
%
%
\noindent
The configuration reached by the machine
after $n$ steps of computation is obtained
by composing $n$ times
its reachability function, defined below.


%%% DEFINITION
%%% REACHABILITY FUNCTION
\begin{defn}[STM Reachability Function]\label{df:STMReachability}
Given a STM,
$M=\langle \m{\Qs}, \m{q_0}, \m{\Sigma}, \m{\delta}\rangle$,
$\{\stmreach^n_M\}_n$ is the smallest
family of relations such that:
%
%
%
\begin{align*}
\langle \m{\sigma}, \m{q}, \m{\tau}, \m{\eta}\rangle
&\stmreach^0_M
\langle \m{\sigma}, \m{q}, \m{\tau}, \m{\eta}\rangle \\
%
%
\big(\langle \m{\sigma}, \m{q}, \m{\tau}, \m{\eta}\rangle
\stmreach^n_M
\langle \m{\sigma'}, \m{q'}, \m{\tau'},\m{\eta'}\rangle\big)
%
\wedge
%
\big(\langle \m{\sigma'}, \m{q'}, \m{\tau'}, \m{\eta'}\rangle
%
&\stmtrans_\delta
%
\langle \m{\sigma''}, \m{q''}, \m{\tau''}, \m{\eta''}\rangle\big)
\rightarrow
\big(\langle \m{\sigma}, \m{q}, \m{\tau}, \m{\eta}\rangle
\stmreach^{n+1}_M
\langle \m{\sigma''}, \m{q''}, \m{\tau''},\m{\eta''}\rangle\big)
\end{align*}
\end{defn}
%
%
%
%
\noindent
Without loss of generality,
we assume STMs not to use final states:
computation is regarded as concluded
whenever the current configuration does
not define the transition function.\footnote{Indeed,
in all these cases,
we could imagine to add a final state $q_F$
and a transition to it.}


\begin{prop}\label{prop}
For any STM,
$M=\langle \m{\Qs}, \m{q_0}, \m{\Sigma},\m{\delta}\rangle$
and $n\in \Nat$,
$\stmreach^n_M$ is a function.
\end{prop}













\begin{notation}[Final Configuration]
Given an STM,
$M=\langle \m{\Qs}, \m{q_0}, \m{\Sigma}, \m{\delta}\rangle$,
and a configuration
$\langle \m{\sigma}, \m{q}, \m{\tau}, \m{\eta}\rangle$,
we write
$\langle \m{\sigma}, \m{q}, \m{\tau}, \m{\eta}\rangle \not\stmtrans_{\delta}$
when there are no $\m{\sigma'}, \m{q'}, \m{\tau'}, \m{\eta'}$
such that
$\langle \m{\sigma}, \m{q}, \m{\tau}, \m{\eta}\rangle
\stmtrans_{\delta} \langle \m{\sigma'}, \m{q'},
\m{\tau'}, \m{\eta'}\rangle$.
\end{notation}
%
%
%
\noindent
Finally, let us introduce the notion of function computable by (poly-time) STMs.





%%% COMPUTATION of STM
\begin{defn}[STM Computation]\label{df:STMcomputation}
Given an STM,
$M = \langle \m{\Qs}, \m{q_0}, \m{\Sigma}, \m{\delta}\rangle$,
strings $\sigma,\gamma \in \Ss$,
and $\eta:\Nat \map \Bool$,
we say that
\emph{$f_M$ computes $\m{\gamma}$ on input $\m{\sigma}$
and oracle tape $\m{\eta}$},
$f_M (\m{\sigma},\m{\eta})=\m{\gamma}$,
when there are a number $n\in \Nat$,
a string $\m{\tau}\in \Ss, \m{q'} \in \Qs$,
and a function
$\m{\psi} : \Nat \map \Bool$ such that:
$$
\langle \m{\eepsilon}, \m{q_0}, \m{\sigma}, \m{\eta}\rangle
\ \stmreach^n_M \
\langle \m{\gamma'}, \m{q'}, \m{\tau}, \m{\psi}\rangle
\not\vdash_{\delta},
$$
%for some $\m{\tau}, \m{q'$} and $\m\psi$ and
$\m{\gamma}$ being
the longest suffix of $\m{\gamma'}$ not including
$\m{\blank}$.
\end{defn}




\begin{defn}[poly-time Stream Machine]\label{df:PSTM}
A \emph{poly-time stream machine}
is an STM,
$M= \langle \m{\Qs}, \m{q_0}, \m{\Sigma}, \m{\delta}\rangle$
such that,
$$
\exists p\in \textsf{POLY}.
\forall \m{\sigma} \in \Ss,
\eta \in \Bool^\Nat.
\exists n \leq p(|\m{\sigma}|)
\big(\langle \m{\eepsilon},
\m{q_0}, \m{\sigma}, \m{\eta} \rangle
\stmreach^n_{M}
\langle \m{\gamma}, \m{q'}, \m{\tau}, \m{\psi}\rangle \not\vdash_{\delta}.
$$
\end{defn}
%
%
\noindent
We define the class of functions which
are computable by poly-time STMs.


%%% The Class SFP
\begin{defn}[The Class $\SFP$]\label{df:SFP}
$$
\SFP := \{ f \in \Ss \times \Bool^\Nat \ | \
f = f_{M}\},
$$
for some canonical polynomial STM, $M$.
\end{defn}
%
%
\noindent



\begin{remark}
All definitions given so far concern single-input,
single-tape machines only.
%
We could naturally generalize them as
multi-input and multi-tape machines
in the standard way.
\end{remark}





Finally, the notion of initial prefix of an
oracle tape is also crucial:
we will show that a polynomially long prefix
of the oracle tape is sufficient to determine
the value of the function.


%%%% Prerfix of Oracle Tape
\begin{defn}[Prefix of Oracle Tape]
For each function $\eta : \Nat \map \Bool$,
$\sigma \in \Ss$ and  $n,i\in \Nat$ such that
$i<n$,
$$
\eta_n = \sigma \ \ \ \Leftrightarrow \ \ \eta(i)=
{\sigma(i)}.
$$
\end{defn}
%
%
\noindent


%%% LEMMA
% \begin{lemma}[Identity of Prefix]\label{lemma:idPrefix}
% For any number $n\in \Nat$ and functions
% $\eta, \eta': \Nat \map \Bool$ the $n$-long
% prefix of which are
% the same, i.e.~$\eta_n=\eta'_n$,
% strings $\m{\sigma},\m{\tau}\in\Ss$,
% and state
% $\m{q}\in \Qs$,
% $$
% \langle \m{\sigma},
% \m{q}, \m{\tau}, \m{\eta}\rangle \stmreach^n_\delta
% \langle \m{\sigma'}, \m{q'}, \m{\tau'}, \m{\xi}\rangle
% \ \ \ \Leftrightarrow
% \ \ \ \langle\m{\sigma}, \m{q}, \m{\tau}, \m{\eta'}\rangle
% \stmreach^n_\delta
% \langle \m{\sigma'}, \m{q'}, \m{\tau'}, \m{\chi}\rangle.
% $$
% for some $\xi, \chi \in \Ss$.
% \end{lemma}


%\begin{cor}\label{cor:idPrefix1}
%For any $n\in \Nat, \eta, \eta': \Nat \map \Bool$, such that $\eta_n=\eta'_n$, then for any $\m{\sigma},\m{\tau}\in\Ss$, $\m{q}\in \Qs$, there are some $\xi,\chi$ such that:
%$$
%\langle \m{\sigma}, \m{q}, \m{\tau}, \m{\eta}\rangle \overline{\triangleright}^n_\delta \langle \m{\sigma'}, \m{q'}, \m{\tau'}, \m{\xi}\rangle \ \ \ \Leftrightarrow \ \ \ \langle\m{\sigma}, \m{q}, \m{\tau}, \m{\eta'}\rangle \overline{\triangleright}^n_\delta \langle \m{\sigma'}, \m{q'}, \m{\tau'}, \m{\chi}\rangle.
%$$
%\end{cor}


%\begin{cor}\label{cor:idPrefix2}
%For any $n\in \Nat, \eta, \eta': \Nat \map \Bool$, such that $\eta_n=\eta'_n$, then for any $\m{\sigma},\m{\tau}\in\Ss$, $\m{q}\in \Qs$, there are some $\xi,\chi$ such that:
%$$
%\langle \m{\sigma}, \m{q}, \m{\tau}, \m{\eta}\rangle \overline{\underline{\triangleright}}^n_\delta \langle \m{\sigma'}, \m{q'}, \m{\tau'}, \m{\xi}\rangle \ \ \ \Leftrightarrow \ \ \ \langle\m{\sigma}, \m{q}, \m{\tau}, \m{\eta'}\rangle \overline{\underline{\triangleright}}^n_\delta \langle \m{\sigma'}, \m{q'}, \m{\tau'}, \m{\chi}\rangle.
%$$
%\end{cor}
































































%%%%%%%%
Before going on, notice that $\POR$ and $\SFP$
are two inherently different sets:
\begin{align*}
\POR \subseteq \bigcup_{i\in \Nat} \Ss^i \times
\Bool^\Ss \map \Ss \\
%
\SFP \subseteq \bigcup_{i\in \Nat} \Ss^i \times
\Bool^\Nat \map \Ss
\end{align*}
%
Thus, some effort is required
to relate these two classes
%
and, actually, %as we shall see,
the correspondence obtained in this
context is weaker than the one between
$\Sigma^b_1$-formul\ae{} representable in
$\RS$ and $\POR$, from
Section~\ref{taskA} and~\ref{taskB}.
%
Differently from formul\ae{} of $\Lpw$
and functions in $\POR$,
which have access to their source of randomness
in almost the same way,
we cannot define a precise identity between $\SFP$
and $\POR$.
%
We could relate sets of (oracle) functions,
\emph{which are not the same}, by considering
their measure,
and
obtain a weaker (but strong enough) result,
providing encodings between the two classes,
which preserve measures of the random sources
associated to the given input and output.


\begin{theorem}\label{thmTaskC}
\begin{itemize}

\item[i)] For each $f \in \SFP$, there is a $g \in \POR$ such that for every $x,y\in \Ss:$
$$
\mu\big(\{\eta \in \Bool^\Nat \ | \  f(x,\eta) = y\}\big) = \mu\big(\{\omega \in \Os \ | \ g(x,\omega)=y\}\big).
$$

\item[ii)] For each $g\in \POR$, there is an  $f \in \SFP$ such that for every $x,y\in \Ss:$
$$
\mu\big(\{\omega \in \Os \ | \ g(x,\omega) = y\}\big) = \mu\big(\{\eta \in \Bool^\Nat \ | \  f(x, \eta) = y\}\big).
$$
\end{itemize}
\end{theorem}

In the two following sections, we will address the two reductions
separately, namely: in Section \ref{sec:SFPtoPOR}, we will show
that $\SFP$ can be reduced to $\POR$ and in Section
 \ref{sec:PORtoSFP} we will show the converse reduction.

 Finally, in Section \ref{SFPandPPT}, we will show that a similar
 relation holds between $\SFP$ and $\PPT$, too.











































































































































































































































































%%%% SECTION
%%%% FROM SFP to POR
%\newpage
\section{From $\SFP$ to $\POR$}\label{sec:SFPtoPOR}
%
%

In this section, we address the first of the two claims in
Theorem \ref{thmTaskC}:

%%% LEMMA
%%% lemma:taskC1
\begin{lemma}\label{lemma:taskC1}
For any $f\in \SFP$, there is a $g \in \POR$
such that for every $x,y\in \Ss$,
$$
\mu\big(\{\eta \in \Bool^\Nat \ | \ f(x, \eta)=y \}\big)
= \mu\big(\{\omega \in \Os \ | \ g(x,\omega) =y\}\big).
$$
\end{lemma}

When reducing $\SFP$ to $\POR$, we can either build
a direct encoding or not. A direct encoding would require
an on-line management of the probabilistic choices made by
the $\POR$ function and the $\SFP$ function.
This would make the proof cumbersome and would require
quite some effort.
Intuitively, we would need to simulate any function computed
by an STM with the oracle tape associated with
$\m{\eta}$
by means of a
$\POR$-function $g$,
querying a fresh \emph{coordinate} of its
oracle $\omega$ at each simulate step,
in turn emulating
the corresponding value written
on $\m{\eta}$.
From a technical viewpoint, providing such
apparently-natural correspondence between
$\m{\eta}$ and $\omega$ (i.e.~linking functions
in $\Os$ and in $\Bool^\Nat$)
is not trivial.
%
For this reason, we prefer to separate the probabilistic concerns by
the computational one, ending up with a easier and clearer proof,
paying the cost of introducing another intermediate formalism.

This new formalism, called ``\emhp{Finite} Stream Turing Machines'',
is defined as a variation of the STM formalism in which the oracle
tape contains a finite-length string, instead of a finite stream of
characters.

On top of the ``\emhp{Finite} Stream Turing Machines'', we define a
class of functions which can be computed in a polynomial time with
respect to their first input. This class of functions, $\polyF$, is
strongly related to $\SFP$.



%%% proof sketch
Before giving the details of the proof of Lemma \ref{lemma:taskC1},
we would like to outline the main steps of the proof we give,
in order to facilitate the reader in understanding the following
parts of this section.
Given an arbitrary $f \in \SFP$ with time bound $p \in \POLY$,
we define a function $h:\Ss \times \Ss \map \Ss$
such that
$$
f(x,\eta) = h\big(x, \eta_{p(|x|)}\big),
$$
where $h$ is something very close to an ordinary \emph{poly-time} function%
\footnote{The main difference lies in the fact that ordinary poly-time function should take one input only.}; in particular, $f \in \polyF$.
To prove this, we pass
through the corresponding class of machines, as done in
Section~\ref{sec:H}.
%
Then, we define a function $h' : \Ss \times \Ss \times
\Os \map \Ss$ such that, for any $x,y\in \Ss$ and $\omega \in \Bool^\Ss$:
$$
h'(x,y,\omega) = h(x,y)
$$
and we prove that $h'\in \POR^-$.
%
Finally, in Section~\ref{sec:E},
we define a function
$e: \Ss \times \Os \map \Ss \in \POR$
 to mimic the prefix extractor, namely we
define $e$ such that its output have
\emph{the same distribution} of all possible $\eta$'s prefixes,
but taking a functional argument with different signature
to do so we need to deal with a bijection between $\Ss$ and $\Nat$,
which ensures us that for each $\eta \in \Bool^\Nat$
there is an $\omega\in \Bool^\Ss$, such that
any prefix of $\eta$ is an output of
$e(y, \omega)$ for a specific $y$.
Proving $e \in\POR$. Since $\POR$ is closed under composition and
$\POR \setminus \{Q\} \subseteq \POR$, we get the claim. In particular, taking

$$
g(x, \omega) := h'(x, e(x, \omega), \omega)
$$

% Notice that this result entails, as a
% corollary, the fact that Ferreira's PTCA
% is complete with respect to poly-time computation.
% %
% Nevertheless, in order for our proof to be self-contained
% we formally establish this result in
% Section~\ref{sec:G}.
%

We will structure the proof as follows:

\begin{itemize}
  \item In Section \ref{}, we will show that for each $f \in \SFP$
  with time bound $p \in \POLY$ there is a $\polyF$ $h:\Ss \map \Ss$
  such that
  $$
  f(x,\eta) = h\big( x, \eta_{p(|x|)}\big),
  $$
  \item In Section \ref{}, we introduce $\POR^-$
  a subset of $\POR$ which is complete
  with respect to $\polyF$ functions\footnote{And, in particular with to poly-time functions}. This entails that there is a function
  $h' \in \POR$ such that:
  $$
  \forall x, y, \omega. h(x, y)=h'(x, y, \omega)
  $$
  \item Then, in Section \ref{}, we show that $e \in \POR$.
  \item In Section \ref{}, we join all the results in order to prove the claim.
\end{itemize}












%%%%%% PRELIMINARY NOTIONS
\subsection{Preliminary Notions}\label{sec:preliminary}
The proof of Lemma~\ref{lemma:taskC1},
relies on the introduction of some auxiliary notions.
In particular, we need all the necessary instruments to
show that the function $h$ introduced in the
proof of Lemma~\ref{lemma:taskC1} is actually poly-time.
%
To do so, we define:

\begin{itemize}
\item The class $\polyF$, defined in Section \ref{sec:poly-timeTM}.
\item A subset of $\POR$ which is expressive enough to capture the class $\polyF$,
 defined in Section \ref{sub:por-o}.
\end{itemize}
%
%Although some of them are almost standard,  we briefly present them formally.









%%%%%%%% SUBSECTION
%%%%%%%% poly-time TURING MACHINE
\subsubsection{(poly-time) Finite Stream Turing Machine}\label{sec:poly-timeTM}


The Finite Stream Turing Machine (FSTM, for short)
are a blending between
$\SFP$ machines, introduced in Section
\ref{} and ordinary Turing Machines.
Basically, they are ordinary Stream Machines, but
with the difference that the second tape can
contain a finite stream of values.\footnote{Some
definitions are omitted but can be found
in Appendix~\cite{app}}
%
\begin{defn}[Finite Stream Turing Machine]\label{df:FSTuringMachine}
A \emph{Finite Stream Turing machine} is a quadruple,
$M = \langle \m{\Qs}, \m{q_0}, \m{\Sigma}, \m{\delta}\rangle$,
where:
\begin{itemize}
\itemsep0em
%
\item $\m{\Qs}$ is a finite set of states ranged over by the meta-variables
$\m{q_i}$
%
\item $\m{q_0}\in \m{\Qs}$ is the initial state
%
\item $\m{\Sigma}$ is a finite set of characters
ranged over by the $\m{c_i}$ meta-variables.
%
\item $\m{\delta} : \m{\hat{\Sigma}} \times
\m{\Qs} \times \m{\hat{\Sigma}} \times \m{\hat{\Sigma}} \map \m{\hat{\Sigma}}
\times \m{\Qs} \times \m{\hat{\Sigma}} \times
\{\m{L}, \m{R}\}$
is a transition function describing the new
configuration reached by the machine.
\end{itemize}
where $\m{L}$ and $\m{R}$
are two fixed constants,
e.g.~$\m{\zero}, \m{\one}$,
$\m{\hat{\Sigma}} = \m{\Sigma} \cup \{\m{\blank}\}$
and $\m{\blank}$ represents the blank character
such that $\m{\blank} \not \in \m{\Sigma}$.
\end{defn}
%
%
%
\noindent
%Without loss of generality, we can assume $\m{\Sigma}=\{\m{\zero}, \m{\one}\}$, and define the canonical TM as follows:
%
%\begin{defn}[Canonical TM]\label{df:canonicalSTM}
A \emph{canonical Finite Stream Turing Machine} is a FSTM,
such that
$\m{\Sigma}=\{\m{\zero}, \m{\one}\}$,
$\m{L}=\m{\zero}$, and $\m{R}=\m{\one}$.
%\end{defn}
%
%
Configurations are defined in the standard way.

\begin{defn}[FSTM Configuration]\label{df:TMConfiguration}
The \emph{configuration of a FSTM} is a 4-tuple
$\langle \m{\sigma}, \m{q}, \m{\tau},\m{\xi}\rangle$, where
\begin{itemize}
\itemsep0em

\item $\m{\sigma} \in \m{\hat{\Sigma}^*}$ is the portion of the work tape on the left of the head

\item $\m{q} \in \m{\Qs}$ is the current state of the machine

\item $\m{\tau} \in \m{\hat{\Sigma}^*}$ is the portion
of the work tape on the right of the head.

\item $\m{\sigma} \in \m{\hat{\Sigma}^*}$ is the portion
of the secondary tape on the right of the head.

\end{itemize}
\end{defn}


Due to the definition of FSTM configuration, it is possible to
define the FSTM's transition function.






\begin{defn}[FSTM Reachability Function]\label{df:TMReachability}
Given a TM,
$M=\langle \m{\Qs}, \m{q_0}, \m{\Sigma}, \m{\delta}\rangle$,
$\{\tmreach^n_M\}_n$ is the smallest
family of relations such that:
%
%
%
\begin{align*}
\langle \m{\sigma}, \m{q}, \m{\tau}, \m{\xi}\rangle
&\tmreach^0_M
\langle \m{\sigma}, \m{q}, \m{\tau}, \m{\xi}\rangle \\
%
%
\big(\langle \m{\sigma}, \m{q}, \m{\tau}, \m{\xi}\rangle
\tmreach^n_M
\langle \m{\sigma'}, \m{q'}, \m{\tau'}, \m{\xi'}\rangle\big)
%
\wedge
%
\big(\langle \m{\sigma'}, \m{q'}, \m{\tau'}, \m{\xi''}\rangle
%
&\tmstep_{\m{\delta}}
%
\langle \m{\sigma''}, \m{q''}, \m{\tau''}, \m{\xi''}\rangle\big)
\rightarrow
\big(\langle \m{\sigma}, \m{q}, \m{\tau}, \m{\xi}\rangle
\tmreach^{n+1}_M
\langle \m{\sigma''}, \m{q''}, \m{\tau''}, \m{\xi''}\rangle\big)
\end{align*}
\end{defn}
\noindent
Without loss of generality,
we assume TMs not to use final states:
computation is regarded as concluded
whenever the current configuration does
not define the transition function.\footnote{Indeed,
in all these cases,
we could imagine to add a final state $q_F$
and a transition to it.}




\begin{prop}
For any FSTM, $M\langle \m{\Qs}, \m{q_0}, \m{\Sigma},
\m{\delta}\rangle$ and $n\in \Nat$,
$\tmreach^n_M$ is a function.
\end{prop}

\begin{notation}[Final Configuration]
Given a FSTM,
$M=\langle \m{\Qs}, \m{q_0}, \m{\Sigma}, \m{\delta}\rangle$,
and a configuration
$\langle \m{\sigma}, \m{q}, \m{\tau}, \m{\xi}\rangle$,
we write
$\langle \m{\sigma}, \m{q}, \m{\tau}, \m{\xi}\rangle \not\tmstep_{\m{\delta}}$
when there are no $\m{\sigma'}, \m{q'}, \m{\tau'}, \m{\xi'}$
such that
$\langle \m{\sigma}, \m{q}, \m{\tau}, \m{\xi}\rangle
\tmstep_{\m{\delta}} \langle \m{\sigma'}, \m{q'},
\m{\tau'}, \m{\xi'}\rangle$.
\end{notation}
%
%
%

%
%Transition function is again as predictable and denoted as  $\tmstep_{\delta}$.


%%% DEFINITION
\begin{defn}[FSTM Transition Function]\label{df:FSTMTransition}
Given an STM, $M=\langle
\m{\Qs}, \m{q}, \m{\Sigma}, \m{\delta}\rangle$,
we define the \emph{partial transition function}
$\tmstep_{\delta} \m{\hat{\Sigma}^*} \times
\m{\Qs} \times
\m{\hat{\Sigma}^*}\times
\m{\hat{\Sigma}^*}
\map
\m{\hat{\Sigma}^*} \times
\m{\Qs} \times
\m{\hat{\Sigma}^*}$
between two configurations as:
%
\begin{align*}
\langle \m{\sigma}, \m{q}, \m{c\tau}, \m{d} \rangle
%
\tmstep_{\m{\delta}}
%
\langle \m{\sigma c'}, \m{q'}, \m{\tau}\rangle
%
\ \ \ \ \ &\text{if} \ \m{\delta}(\m{q},\m{c}, \m{d}) =
\langle \m{q'}, \m{c'}, \m{R}\rangle \\
%
%
%
\langle \m{\sigma c_0},
\m{q}, \m{c\tau}, \m{d}\rangle
\tmstep_{\m{\delta}}
\langle \m{\sigma},
\m{q'}, \m{c_0c_1'\tau}
\rangle
%
\ \ \ \ \ &\text{if} \ \m{\delta}(\m{q}, \m{c_1}, \m{d}) = \langle
\m{q'}, \m{c_1'}, \m{L}\rangle \\
%
%
%
\langle \m{\sigma}, \m{q}, \m{c\tau}, \m{d}\rangle
%
\tmstep_{\m{\delta}}
\langle \m{\sigma c'}, \m{q'},
\m{\tau}\rangle
%
\ \ \ \ \ &\text{if } \m{\delta}(\m{q}, \m{c}, \m{d})
= \langle \m{q'}, \m{c'}, \m{R}\rangle \\
%
\langle \m{\sigma c_0}, \m{q}, \m{c_1\tau}, \m{d}\rangle
\tmstep_{\m{\delta}}
\langle \m{\sigma}, \m{q'}, \m{c_0c_1' \tau}, \m{d}\rangle
%
\ \ \ \ \ &\text{if }
\m{\delta}(\m{q},\m{c_1}, \m{d}) = \langle
\m{q'},\m{c_1'},\m{L}\rangle.
%
\end{align*}
%
\end{defn}
%
\noindent
As for STM, the configuration reached by the machine
$M$ after $n$ steps of computation is obtained
by composing $n$ times
its reachability function, denoted by $\tmreach^n_M$.




%%% poly-time TM
\begin{defn}[poly-time Finite Stream Turing Machine]\label{df:polyTM}
A \emph{poly-time Finite Stream Turing machine}
is a TM, $M=\langle \m{\Qs}, \m{q_0}, \m{\Sigma},
\m{\delta}\rangle$
such that,
$$
\exists p \in \mathsf{POLY}. \forall \m{\sigma}, \m{\tau} \in \m{\Ss}.
\exists n \leq p(|\m{\sigma}|)
\big(\langle \m{\eepsilon}, \m{q_0}, \m{\sigma}, \m{\tau}
\rangle \tmreach^n_M
\langle \m{\gamma}, \m{q'}, \m{\sigma'}, \m{\tau'}\rangle
\not \tmstep_\delta
$$
\end{defn}

%
% \begin{remark}
%   All these definitions scale naturally to multi-tape Turing machines, if we define
%   the configuration of a $k$-taped TM as a tuple:
%   $$
%     \langle \sigma_1, \ldots, \sigma_k, q, \tau_1, \ldots, \tau_k\rangle
%   $$
%   Where the $\sigma_i$ are the strings on the left of each tape's head, and
%   the $\tau_i$ are the strings on the right of the heads.
% \end{remark}

%%% COMPUTATION of STM
\begin{defn}[FSTM Computation]\label{df:FSTMcomputation}
Given a  FSTM,
$M = \langle \m{\Qs}, \m{q_0}, \m{\Sigma}, \m{\delta}\rangle$,
three strings $\sigma, \tau, \gamma \in \Ss$,
we say that
$f_M$ computes $\m{\gamma}$ on input $\m{\sigma, \tau}$
$f_M (\m{\sigma},\m{\tau})=\m{\gamma}$,
when there are a natural number $n\in \Nat$,
a string $\m{\xi}\in \Ss, \m{q'} \in \Qs$,
such that
$$
\langle \eepsilon, q, \sigma, \tau\rangle \tmreach^n_M
\langle \gamma',  q, \sigma', \tau'\rangle\
$$
%for some $\m{\tau}, \m{q'$} and $\m\psi$ and
with $\m{\gamma}$ being
the longest suffix of $\m{\gamma}$ not including
$\m{\blank}$.
\end{defn}




We are now able to define the class
$\polyF$, namely the class of functions
which are computable by poly-time FSTMs.



%%% Class ?
\begin{defn}[The Class $\polyF$]\label{df:poly-time}
$$
\polyF := \{f \in \Ss \times \Ss \map \Ss \ | \ f=f_M\}
$$
for some poly-time TM $M$.
\end{defn}


The class $\polyF$, basically, is a generalization to $k$-taped Turing Machines
of the class PF.









%%%%%%% SUBSUBSECTION
\subsubsection{The Class $\POR^-$}
\label{sub:por-o}

%
Furthermore, we present the class
$\POR^-$ which is defined as
$\POR$ except for the
absence of the query function $Q$.
%
This class is useful because it will be trivially shown being
sound and complete with respect
to Ferreira's PTCA \cite{}
(Remarks \ref{remarkdafare} and \ref{remarkdafare2}).
%
This entails that, showing that $\POR^-$ is complete
with respect to $\polyF$ function
and that $\polyF$ functions are complete with respect to
\emph{ordinary poly-time functions} (Lemma \ref{lemma:polyFcompleteness})
yields, as a corollary, the proof that
Ferreira's PTCA contains the class of poly-time
computable functions.
%
The completeness of $\POR^-$ with
respect to FSTM function is proved in Section \ref{sec:G}.



\begin{defn}[The Class $\POR^-$]
The \emph{class $\POR^-$}
is the smallest class of functions
$\Ss^n\times \Os\map \Ss$ containing:
\begin{itemize}
\itemsep0em

\item the empty function $E(x,\omega)=\eepsilon$

\item the projection function
$P^{n}_i(x_1,\dots, x_n,\omega)=x_i$

\item the word-successor $S_\bool(x,\omega) = x\bool$,
for every $\bool \in \Bool$

\item the conditional function
\begin{align*}
C(\eepsilon, y, z_\zero, z_\one,\omega) &= y \\
C(x\bool, y, z_\zero, z_\one,\omega) &= z_\bool.
\end{align*}
where $\bool \in \Bool$.




\end{itemize}
and closed under:
\begin{itemize}
\item composition, such that $f$ is defined from
$g,h_1,\dots, h_k$ as:
$$
f(\vec{x}) = g\big(
h_1(\vec{x},\omega), \dots, h_k(\vec{x},\omega),\omega\big)
$$


\item bounded recursion, such that $f$ is defined from
$g, h_1, h_2$ as:
\begin{align*}
f(\vec{x},\eepsilon,\omega) &:= g(\vec{x},\omega) \\
%
f(\vec{x}, y_1\zero,\omega) &:= h_1
\big(\vec{x}, y_1,
 f(\vec{x},y_1,\omega),\omega\big)|_{t(\vec{x},y)} \\
%
f(\vec{x}, y_2,\omega\one) &:=
h_2\big(\vec{x}, y,
f(\vec{x}, y_2,\omega),\omega\big)|_{t(\vec{x},y_2)}
\end{align*}
where $t$ is defined from $\eepsilon, \zero, \one,
\frown, \times$ by explicit definition.
\end{itemize}
\end{defn}


%\begin{defn}[The Class $\POR^-$]
%The class $\POR^-$ is  the class of functions $\Ss^n\times \Os \map \Ss$ containing $E, P^n_i, S_{\bool},  C$, and closed under oracle composition and bounded-recursion.
%\end{defn}











































































































%\newpage
%%%%%% ENCODING

\subsection{From $\SFP$ to $\polyF$}\label{sec:H}
First of all we show that for any $f\in \SFP$,
the corresponding $\polyF$ function
$h:\Ss \times \Ss\map \Ss$ can be constructed.
%
The core idea consists in passing through
the associated machines.
%
According to Definition~\ref{df:SFP},
there is a poly-time STM $M$, such that $f=f_M$.
%
We want to construct the corresponding
\emph{poly-time} FSTM $N$, which,
given as second argument polynomial prefix of $\eta$,
behaves exactly like $M$.
%
In particular, $N$ is constructed basing on a
two-taped TM $N$, which
simply inherits the transition function of $M$
%
For each computation,
$N$ performs a number of steps which
is \emph{exactly the same number of those performed} by $M$,
so also this standard TM must be \emph{poly-time}.



%%% LEMMA
%%% lemma:poly-time
\begin{lemma}\label{lemma:SFPtopolyF}
For each $f\in \SFP$ with time-bound $p\in \POLY$,
there is an $h \in \polyF$ such that
for any $\eta\in \Bool^\Nat$ and $x, y\in \Ss$,
$$
f(x,\eta) = h(x, \eta_{p(|x|)}).
$$
\end{lemma}



%%% Proof of Lemma
\begin{proof}
Assume that $f \in \SFP$.
%
By Definition~\ref{df:SFP},
there is a poly-time STM, $M =\langle \m{\Qs},
\m{q_0}, \m{\Sigma}, \m{\delta}\rangle$,
such that $f=f_M$.
%
Let us define a (standard) TM $N$
which, given a polynomial prefix of $\eta$,
behaves like $M$.
%
The Definition of $N$ is identical to the definition of $M$.
Formally,
for any $k\in \Nat$ and some $\sigma,\tau , y' \in \Ss$
$$
\langle \m{\eepsilon}, \m{q_0'}, \m{x}, \m{y} \rangle
\triangleright^k_{\delta'} \langle \m{\sigma},
\m{q}, \m{\tau}, \m{y'}\rangle
\ \ \ \Leftrightarrow \ \ \
\langle \g{\eepsilon}, \g{q_0'}, \g{x}, \g{\eta}\rangle
\stmreach^k_\delta
\langle \g{\sigma}, \g{q}, \g{\tau}, \g{y'\eta}\rangle.
$$
Moreover,
$N$ requires a number of steps which is exactly equal to
the number of steps required by $M$, and thus
is in $\polyF$, too.
%
We conclude the proof defining $h=f_{N}$.
\end{proof}













































































%\newpage
%%%%%%%
%%%%%%%
%%%%%%% FROM PTF TO POR
\subsection{From $\polyF$ to $\POR^-$}\label{sec:G}

In this section we will show that all the function which are in $\polyF$
can be represeted in $\POR$, too. This can formalized as follows.

%%% LEMMA
\begin{restatable}{lemma}{polyFtoCob}
\label{lemma:polyFtoCob}
  For any $f\in \polyF$ and $x\in \Ss$,
  there is a $g\in \POR^-$,
  such that
  $
  \forall x, y, \omega. f(x, y)=g(x, y,\omega).
  $
\end{restatable}

\begin{proof}[Proof Sketch]
  Any configuration of a Finite Stream Turing Machine
  can be encoded into strings, thus we
  can pass the value which is an encoding of the initial machines configuration
  to a function which emulates the execution of the machine which computes $f$ ($M_f$) for
  a  polynomial number of steps.
  When $M$ reaches a final configuration, it is sufficient to extract
  from the final configuration the longest portion of the primary tape
  which is on the left and free form $\circledast$ characters.
\end{proof}

In order to prove the lemma above, we did an important technical work, which can
be found in Section \ref{sub:encoding} of the Appendix. Now we will show that
$\POR^-$ benefits of three important properties:

\begin{itemize}
  \item There is a function $\apply \in \POR^-$ which receiving as input
  the encoding of a FSTM configuration $c$, and the encoding of a
  FSTM transition function $\delta$,
  it computes the encoding configuration obtained
  applying the function $\vdash_\delta$ on $c$.
  \item For each $f \in \POR^-$ and for each $x, y \in \Ss$ if there is a
  term $t(x)$ in $\Lpw$
  which bounds $f(x, \omega)$ for each $\omega$, then there is a function which
  applies $|y|$ times $f$ on its own output.
  \item There is a function $\dectape$ which extracts
  the machine's output from any encoded final configuration of a FSTM machine.
\end{itemize}

\subsubsection{The function $\apply$}
\label{subsub:apply}

To define $\apply$, we show that given the encoding of a finite function
(for details, see Corollary \ref{cor:deltarepr} in the Appendix), can be
\emph{interpreted} by means of a $\POR^-$ function,
the \emph{total function simulator}\footnote{We call it \emph{total} because,
since $\POR^-$ is total, it defines a default value to be returned when the
simulated function is not defined on the queried input.}


\begin{defn}[Total Function Simulator]
We define the \emph{total function simulator}
$sim(\cdot, \cdot, \omega)$ as follows:
\begin{align*}
sim'(y,x,\eepsilon,\omega) &:= \eepsilon \\
%
sim'(y,x,z\bool, \omega) &:= \mathit{if}\big(\pi_2(\pi(y,z\bool,\omega),
\omega), sim'(y,x,z,\omega), eq(\pi_1(\pi(y,z\bool,\omega),
\omega), x,\omega), \omega\big) \\
\\
sim(x,y,\omega) &:= sim'\big(y,x,\pi_0(y,\omega),\omega\big).
\end{align*}
\end{defn}

The total function simulator is defined by induction on the number of elements
in the encoding of the simulated function is defined
(the number of pairs in the function's graph). At step $i$ the unction extracts the
$i$-th projection from the function's graph and compares its first element with
the queried value. If they correspond, then it returns the second projection of
the pair, otherwise is procedes recursively.
%
The formal proof of the correctness of this function is in the Appendix,
Lemma \ref{lemma:simcorr}.
%
It holds that there is a function $\apply$ in $\POR$ which,
given a transition function $\delta$, simulates the function $\vdash_\delta$
on the encodings of machine's configuration, formally:

\begin{lemma}
  \label{lemma:applycorr}
  The function $\apply$ is such that for any FSTM $M$ with transition function
  $\delta$, said $x_\delta$ the encoding of $\delta$ described in Corollary
  \ref{cor:deltarepr}, and $h(c)$ the encoding of a configuration $c$ as
  described in Definition \ref{def:canencs},
  \begin{align*}
    \forall \omega \in \Os.\vdash_\delta(c)=d \to \apply(x_\delta, h(c), \omega) = h(d)\\
    \forall \omega \in \Os. \left(c \not\vdash_\delta\right ) \to \apply(x_\delta, h(c), \omega) = h(c)
  \end{align*}
\end{lemma}
%
\noindent
The formal proof of the statement of Lemma \ref{lemma:applycorr}, together
with the definition of the function $\apply$ are quite cumbersome, for this
reason they are given in the Appendix.
%
\subsubsection{Power Function}
\label{subsub:pf}

In this section we show that for each function in $\POR^-$, if it has an output
size which is bounded by a term in $\Lpw$, then its $n$-th power is in $\POR^-$.
%
Thanks to this result, we will be able to compute the polynomial transitive
closure of a function in $\POR^-$, in order to compute the final configuration
of any FSTM, given its $\delta$ function and its input.

\begin{lemma}
  \label{lemma:saPOR}
  For each $f : \Ss^{k+1} \times \Os \longrightarrow \Ss \in \POR$
  If there is a term $t \in \Lpw$ such that
  $\forall x, \vec z, \omega. f(x, \vec z, \omega)|_t = f(x, \vec z, \omega)$
  then there is also a function $\sa_{f, t} : \Ss^{k+2} \times \Os \longrightarrow \Ss$
  such that
  $$
  \forall n \in \Nat. \forall x \in \Ss,  \omega \in \Os.
  sa_{f, t}(x, \ovverline n \Nat, \vec z, \omega) =
  \underbrace{f(f(f(x, \vec z,  \omega), \vec z, \omega), \ldots)}_{n\text{ times}}.
  $$
\end{lemma}

For sake of readability, we will proof this result giving the definition of the
$\sa_{\cdot, \cdot}$ funtion schema only in the Appendix, the details
are in Proof \ref{proof:saPOR}.
%
The function $\sa_{\cdot,\cdot}$ allows us to compute
the transitive closure of the function $\apply$, but to do so,
we must show that the growth of the terms computed by $\apply$
is under control. This is due to the fact that the only iterative
mechanis of $\POR^-$ is the bounded recursion on notation, which
requires that, at each step, the function outputs are bounded in size.
%
So, the self-application schema must pass through bounded recursion on notation
and then, requires size bounds. For this reason, we show that the term-growth
of the size growth of the output of the function $\apply$ is at mosst constant.

\begin{lemma}[$\apply$ Size growth]
  \label{lemma:applysize}
  For all $x_c\in \Ss$ being the encoding of a FSTM configuration
  and for each encoding of a transition function $\delta$
  in a string $x_\delta \in \Ss$ and for each $\omega\in \Os$, there is
  a $k \in \Nat$ sich that
  $\apply(x_c, x_\delta, \omega)|_{x_c\one^k}$
\end{lemma}

The specific value of the size bound expressed in the lemma above
is not particularly meaningful, and it is
mainly due to the encodings we decided to adopt, but it is important that
fixed a FSTM machine we can always find such $k$.
A technical proof of this result is in the Appendix.








%%%%% REPRESENTING TRANSITION IN COB
\subsubsection{Representing transition in $\POR^-$}

In this section, we show that it is possible to define a function
$\dectape \in \POR^-$ which takes in input the encoding of a tape and
$\omega \in \Os$, and returns the longest suffix of the tape on the left of the
head without any occurrence of $\circledast$. To do so, we need to do some
technical work. First we introduce
some auxiliary functions $\in \POR^-$:
the difference function $\mathit{diff}$ returning the difference
between two numbers,
the list-projector function $\pi_n$
taking (the encoding of) a list
and a number $n$ as its input and returning
the $n$-th element of the list,
and the right-remover function $\rrs$.\footnote{Formal
definitions are presented in Appendix~\ref{}.}
Moreover, $eq: \Ss^2 \times \Os \map \Ss$
is a function in $\POR^-$.
%
%
%
%
%
%
While decoding the final configuration of a
canonical TM,
we need to extract the longest
sequence of bits on the immediate
left of the head.
%
To do so, we introduce an auxiliary
function $\rho:\Ss^2\times \Os \map \Ss$
that is supposed to take the encoding of a tape
as its input and return the $y$-th right character
of the tape.
Formally,
%
$$
\rho(x,y,\omega) = \pi\big(x,
diff(\pi_0(x,\omega), y,\omega),\omega\big).
$$
Then, we define a \emph{decoding}
function so that at each step,
due to $eq$,
it checks whether the character obtained from
$\rho$ is the encoding of $\circledast$ defined in the Appendix, i.e.
$\one\one\one$.
If so, it returns $\eepsilon$.
%







%%%%%% DECTAPE
\begin{defn}[$\dectape$ Function]
Let $dec :\Ss\times \Ss \times \Os
\map \Ss$ be an auxiliary function defined
as follows:
\begin{align*}
dec(x,\eepsilon,\omega) &= \eepsilon \\
dec(x, y\bool,\omega) &= \mathit{if}\big(dec(x,y,\omega)
\rho (x,y\bool,\omega), \eepsilon,
\lnot eq(\rho(x,y\bool,\omega),\omega), \one\one\one,\omega)\big)
\end{align*}
with
$$
\rho(x,y,\omega) = \pi\big(x,
diff(\pi_0(x,\omega), y,\omega),\omega\big).
$$
We define the function \emph{dectape}:
$\Ss  \times \Os \map \Ss$
as follows:
$$
dectape(x,\omega) = dec\big(x,rrs(\pi_0(x,\omega),\omega),
\omega\big)
$$
\end{defn}
%
%
\noindent
The function $\dectape$ returns a string which is the
longest suffix not including $\circledast$
of the tape encoded with $\ovverline \cdot \Tt$ and stored in $x$.
By Definition~\ref{def:TM}, this is precisely
the value computed by the machine.
A formal proof of this statement
is given in the Appendix, the details are in Lemma \ref{lemma:dectape}

\begin{lemma}\label{lemma:dectape}
{The function $\mathit{dectape}\in \POR^-$
is such that if $\underline{\cdot}_\Tt$
is the encoding for tapes above,
then for any $\sigma\in \{\zero,\one,\blank\}$
$\omega \in \Os$}
$$
f(\underline{\sigma}_\Tt, \omega) = \tau
$$
and $\tau$ is the longest suffix of $\sigma$ without $\blank$.
\end{lemma}


Let us also define a function \emph{size}
which allows us to compute the
encoding of the size of a string throughout
$\underline{\cdot}_\Nat$
\begin{defn}
The function $\mathit{size} : \Ss\times \Os \map \Ss$
is defined as follows:
\begin{align*}
\mathit{size}(\eepsilon, \omega) &= \one \\
\mathit{size}(x\bool,\omega) &= \mathit{size}(x,\omega)\one|_{x\one\one}.
\end{align*}
\end{defn}

The correctness of the function $\mathit{size}$, with respect to the
encoding we are adopting --- namely $n \mapsto\one^{n +1}$ ---
can be shown by induction.













%%%%% CONCLUDING THE PROOF
\subsubsection{\emph{Concluding the Proof}}
It is now possible to show that, for any
$\SFP$-function there is
a function in $\POR^-$,
which behaves exactly like the  former one.
%
The conclusion is the \emph{composition} of three main ingredients:
\begin{itemize}
  \item Machine steps can be simulated by means of the function $\apply$.
  \item The function $\apply$ can be self-applied a polynomial numbers of times.
  \item We can extract the value computed by the machine by its final configuration.
\end{itemize}

These results allow us to give a proof to Lemma \ref{lemma:polyFtoCob}.



%%% LEMMA
\polyFtoCob*
% \begin{lemma}\label{lemma:polyFtoCob}
% {For any $f\in \polyF$ and $x, y\in \Ss$,
% there is a $g\in \POR^-$,
% such that
% $
% \forall \omega \in \Os. f(x, y)=g(x, y, \omega).
% $}
% \end{lemma}
%% PROOF
\begin{proof}
%
By definition of $\polyF$,
there is a FSTM computing $f$ on a machine $M$ with a polynomial
time-bound, $p$, and a transition function $\delta$.
%
Let us consider a function $g$
defined as follows:
$$
g(x, y, \omega) = dectape(\pi_{1}(sa_{\apply, t_M}(x_\delta,
\langle \ovverline \circledast \Tt, \underline{0}_\Nat, \ovverline x\Tt, \ovverline y\Tt\rangle_\Ll,
\underline p(\mathit{size}(x, \omega), \omega), \omega), \omega), \omega)
$$
where $p$ is the $\POR^-$-function which
computes the encoding of the value of the polynomial $p$.\footnote{
This function is in $\POR^-$, because we have shown that this class
contains all the polynomials.}
This function is correct as the self-application of $\apply$ for $p(|x|)$
times returns in the machine configuration, as a consequence of Lemma \ref{lemma:applycorr}.
Finally, \emph{dectape}
extracts the longest suffix free from blank characters
which is on the left of the head
in the encoding of the tape reached at the end of the computation.
This is a consequence of the correctness of the projector $\pi_1$
with respect to the encoding of lists (for details, see the Appendix)
and of the correctness of $\dectape$ (Lemma \ref{lemma:dectape}).
As required by Definition~\ref{def:TM}, this is exactly $f(x, y)$.
\end{proof}
%
%
%
%
%
\noindent
As a consequence,
%%% COROLLARY
\begin{cor}
  \label{cor:sfptopor-}
For each $f\in \SFP$ and polynomial
time-bound $p\in \POLY$,
there is a function $g\in \POR^-$ such that
for any $\eta : \Nat \map \Bool$, $\omega : \Nat \map \Bool$ and $x\in\Ss$,
$$
f(x,\eta) = g\big(x,\eta_{p(|x|)}, \omega\big).
$$
\end{cor}

%%% PROOF
\begin{proof}
Assume $f\in \SFP$ and $y = \eta_{p(|x|)}$.
By Lemma~\ref{lemma:SFPtopolyF},
there is a function $h\in \polyF$ such that,
for any $\eta :\Nat \map \Bool$ and
$x\in \Ss$,
$$
f(x,\eta) = h\big(x,\eta_{p(|x|)}\big).
$$
%
%
Moreover,
due to Lemma~\ref{lemma:polyFtoCob},
there is also a $g\in \POR^-$ such that
%
$$
g(x,y, \omega) = h\big( x, y \big).
$$
%
%
Then, the desired function is $g$.
\end{proof}




























































































%\newpage
\subsection{Extractor Function in $\POR$}\label{sec:E}
Finally, we construct an extractor function $e(x, \omega)$ in $\POR$
we introduced in the sketch of the proof of Lemma \ref{lemma:taskC1}.
%
This function extracts $|x|+1$ bits from $\omega$ and places them in a
and concatenates them in its output.
%%%%% PRELIMINARIES
%\paragraph{\emph{Preliminaries.}}
%
%
As said, for any polynomial $p$, $x\in \Ss$ and $\omega\in\Os$,
there is $t\in \POR^-\subseteq \POR$, %(or $\in \polyF$)
such that
$|t(x,\omega)|=p(|x|)$.
Then, there is also a $t \in \POR$ such that
for any $\vec{x} \in \Ss$ and $\omega \in \Os$,
$|t(\vec{x},\omega)|= p(|\vec{x}|)$.
%
Moreover, it is possible to associate each natural
number $n$ to its dyadic representation.
Intuitively, for each natural number $n\in \Nat$,
if $n_2$ is the encoding of $n$ in base 2,
the dyadic representation of $m$ is $(m+1)_2$ without is leftmost bit.
%
The intuition is at the basis of our definition
of an ``extractor'' function in $\POR$.











First, we define a function $bin:\Ss\times \Os\map \Ss$
such that, if $x$ has length $k$,
then for every $\omega\in \Os$,
$bin(x,\omega)$ is the binary encoding of $k$.
%\footnote{For
%readability's sake, we define functions in a semi-formal way.
%The formal definition of their representability in $\POR$
%can be found in Appendix~\ref{??}.}





\begin{defn}
Let us define an auxiliary function
$binsucc:\Ss\times \Os\map \Ss$,
\begin{align*}
binsucc(\eepsilon,\omega) &= {\one} \\
binsucc(x\zero,\omega) &= x\one|_{x\zero\zero} \\
binsucc(x\one,\omega) &= binsucc(x,
\omega)\zero|_{x\zero\zero}.
\end{align*}
Then, $bin:\Ss\times \Os \map \Ss$ is
defined as:
\begin{align*}
bin(\eepsilon,\omega) &= {\zero} \\
bin(x\bool,\omega) &= binsucc\big(bin(x,\omega),\omega\big)|_{x\bool}
\end{align*}
\end{defn}
%
On top of the definition of the binary encoding of a number,
we define the dyadic encoding of it:
\begin{defn}
  We call $\mathit{dy}: \Ss \times \Os \longrightarrow \Ss$ the function in $\POR^-$ such that
  $\forall n \in \Nat, \omega \in \Os. \mathit{dy}(\ovverline n \Nat, \omega)$ is the diadic encoding of $n$.
  Namely:
  \begin{align*}
    \mathit{dy}(x, \omega):= \lrs(bin(x, \omega), \omega), \omega
  \end{align*}
\end{defn}
%
In the Appendix (Lemma \ref{lemma:dyadbij})
we prove that for any $\omega$,
the function $\mathit{dy}(\ovverline n \Nat, \omega)$
is a bijection between a a natural number $n$ and a string
--- its dyadic representation.
This result will be crucial proving Lemma \ref{lemma:taskC1}, because it relates
throughout a bijection the set $\Bool^\Nat$ to $\Bool^\Ss$.
%
%
%
%
%
%
%
\noindent
We then define a function $e \in \POR$,
which takes a string and an oracle
as its input and returns a finite string obtained picking $|x|$ bytes from it,
choosing exactly the coordinates of $\omega$ corresponding to the dyadic encoding
of the first $|x|$ natural numbers.
%%% DEFN
\begin{defn}
Let $e :\Ss\times \Os \map \Ss$ be defined as follows:
\begin{align*}
e(\eepsilon,\omega) &= \eepsilon \\
%
e(x\bool,\omega) &= e(x, \omega)Q\big(\mathit{dy}(x,\omega),\omega)|_{x\bool}.
\end{align*}
\end{defn}
%
%
\noindent

\begin{defn}
  We define $\sim_{\mathit{dy}}$ as the smallest relation in
   $\Bool^\Ss\times \Bool^\Nat$ such that:
   $$
   \eta \sim_\mathit{dy} \omega \leftrightarrow \forall n \in \Nat.
    \eta(n)= \omega (\mathit{dy}(\ovverline n \Nat, \omega))
   $$
\end{defn}

This relation has some good properties:

\begin{lemma}
  \label{lemma:funbij}
  It holds that:
  \begin{itemize}
    \item $\forall \eta \in \Bool^\Nat. \exists ! \omega \in \Bool^\Ss. \eta \sim_{\mathit{dy}} \omega$
    \item $\forall \omega \in \Bool^\Ss. \exists ! \eta \in \Bool^\Nat. \eta \sim_{\mathit{dy}} \omega$
  \end{itemize}
\end{lemma}
\begin{proof}
  The proofs are very similar, for this we will take in exam only the first.
  By the fact that $\mathit{dy}$ is a bijectio  with respect to its
  first argument and is constant over its second argument,
  we obtain the existence of an $\omega$ which is in relation with $\eta$.
  Now suppose that there are $\omega_1, \omega_2$ both in relation with $\eta$
  being different. It holds then that $\exists \sigma \in\Ss$ such that
  $\omega_1(\sigma)\neq\omega_2(\sigma)$. Then, since $\mathit{dy}$ is in $\POR^-$,
  the value of its last argument does not affect th value of its output,
  moreover it is a bijection so we get that the claim holds.
\end{proof}
\begin{cor}
  The relation $\sim_{\mathit{dy}}$ is a bijection.
\end{cor}

\begin{proof}
  Consequence of Lemma \ref{lemma:funbij}.
\end{proof}









%\newpage


%%% COROLLARY
\begin{cor}
  For each $f \in \SFP$, there is a $g \in \POR$ such that:
  $$
  \mu\big(\{\eta \in \Bool^\Nat \ | \ f(x, \eta)=y \}\big)
  = \mu\big(\{\omega \in \Os \ | \ g(x, \omega) =y\}\big).
  $$
\end{cor}

%%% PROOF
\begin{proof}
  From Corollary \ref{cor:sfptopor-}, we know that there is a function $f'\in \POR^-$,
  and a $p \in \POLY$ such that
  $$
  \forall x, y \in \Ss.\forall \eta.\forall \omega. y = \eta_{p(x)} \to  f(x, \eta) = f'(x, y, \omega)\quad (*)
  $$
  So, by the fact that $\POR^- \subseteq \POR$, $f' \in \POR$, too.
  %
  % For this reason, the application of Lemma \ref{}, yields the following result:
  %
  % $$
  % \left(\forall 1 \le i \le p(|x|). \omega (\mathit{dyad}(i))={\eta}(i)\right) \to g(x, e(p(\mathit{size}(x, \omega), \omega), \omega), \omega) = f(x, \eta)
  % $$
  %
  Fixed an $\eta \in \{\eta \in \Bool^\Nat \ | \ f(x, \eta)=y \}$,
  its image with respect to $\sim_{\mathit{dy}}$
  is in
  $$
  \{\omega \in \Os \ | \ f'(x, e(p'(\mathit{size}(x, \omega), \omega), \omega), \omega) =y\}
  $$
  %
  Indeed, by Lemma \ref{lemma:auxsimdy}, it holds that
  $\eta_{p(x)} = e(p(\mathit{size}(x, \omega), \omega)$, where $p'$ is the $\POR^-$
  function computing the polynomial $p$ and $\mathit{size}$ is the $\POR^-$, to
  function computing the encoding of the natural number which corresponds to the
  size of its first input, so by $(*)$ we have the claim.
  It also holds that, fixed an $\omega \in \{\omega \in \Os \ | \ f'(x, e(p'(\mathit{size}(x, \omega), \omega), \omega), \omega) =y\}$,
  then its pre-image with respect to $\sim_{\mathit{dy}}$ is in $\{\eta \in \Bool^\Nat \ | \ f(x, \eta)=y \}$.
  The proof is analogous to the one we showed above.
  Now, since $\sim_{\mathit{dy}}$ is a bijection between the two sets, we
  conclude that
  $$
  \mu\big(\{\eta \in \Bool^\Nat \ | \ f(x, \eta)=y \}\big)
  = \mu\big(\{\omega \in \Os \ | \ g(x, e(p(\mathit{size}(x, \omega), \omega), \omega), \omega) =y\}\big).
  $$
  which concludes the proof.
\end{proof}



\newpage

\section{Appendix}\label{appendix}
% !TEX = short.tex
%!TeX spellcheck = en-US

\section{Auxiliary Results}
\label{sec:auxres}
%%%% SECTION SFP

\subsection{The Theory $\RS$}

\begin{remark}
  \label{rem:mufun}
  For each countable set $A$, the cylinder measure $\mu$ is a function.
\end{remark}
\begin{proof}
  Assume $X \subseteq \{\zero, \one\}^A$, and that:
  \begin{align*}
  X&=\{\omega \in \{\zero, \one\}^A\, |\, \omega|_{K_1} \in H_1\}\\
  X&=\{\omega \in \{\zero, \one\}^A\, |\, \omega|_{K_2} \in H_2\},
\end{align*}
we proceed by cases on the identity of the $K_i$ and $H_i$ for $i \in\{ 1,2\}$.
\begin{itemize}
  \item If the $K_i$s and the $H_i$s are identical, the conclusion is trivial.
  \item  Suppose $K_1=K_2$, while $H_1 \neq H_2$. This wound entail
  $X \neq X$, which is absurd.
  \item Suppose $K_1\neq K_2$, while $H_1 = H_2$. This is absurd because
  $H_i \subseteq \{\zero, \one\}^{K_i}$.
  \item Finally, we proceed by double induction on the elements
  of $K_1\setminus K_2$ and $ K_2\setminus K_1$. If such value is $0$,
  we get the claim, otherwise, suppose that there are $n+1$ one such elements.
  Without loss of generality, call that element $k$ and assume it in $K_1$.
  Let $K_2' = K_2 \cup \{k\}$ and
  $$
  H_2'=\{w \in \{\zero, \one\}^{K_2'}\, |\,  w|_{K_2} \in H\}.
  $$
  It holds that:
  $$
  \frac{|H_2'|}{2^{|K_2|}} = \frac{2\cdot |H_2|}{2^{|K_2+1|}}=
  \frac{2\cdot |H_2|}{2\cdot2^{|K_2|}}
  $$
  which proves the claim applying the IH.
\end{itemize}
\end{proof}



\begin{remark}[Cylinder measure]
  \label{rem:mumeas}
  For each countable set $A$, $\mu$ is a measure function.
\end{remark}
\begin{proof}
  First, we show that $\mu$ is defined on all the elements of the field on $\{\zero, \one\}^A$.
  \begin{itemize}
    \item If $X$ is a cylinder, then the conclusion is trivial.
    \item If such element is the union of two elements of the field $X_1$ and $X_2$,
    then $\mu$ is defined on them, so they can be expressed as cylinders.
    So they are two elements defined as
    \begin{align*}
      X_1=\{\omega \in \{\zero, \one\}^A\, |\, \omega|_K \in H\}\\
      X_2=\{\omega \in \{\zero, \one\}^A\, |\, \omega|_{K'} \in H'\}.
    \end{align*}
    Thus we get that:
    $$
    X = X_1 \cup X_2 = \{\omega \in \{\zero, \one\}^A\, |\, \omega|_{K \cup K'} \in H''\},
    $$
    with
    $$
    H'':=\{w \in \{\zero, \one\}^{K\cup K'}\, |\, w|_K \in H \lor w|_{K'} \in H'\}.
    $$
    So, $\mu$ is defined on $X$.
    \item Finally, if $X$ is obtained by complementation, we get that:
    $$
    X = \Bool^A \setminus X'= \{\omega \in \{\zero, \one\}^A\, |\, \omega|_{K}\not  \in H'\} = \{\omega \in \{\zero, \one\}^A\, |\, \omega|_{K} \in \Bool^K \setminus H'\},
    $$
    which is itself a cylinder, so $\mu$ is defined on it.
  \end{itemize}
  Now we show that $\mu$ is a probability measure over the field generated by
  $\{\zero, \one\}^A$.
  \begin{itemize}
    \item It is trivial to see that $0 \le \mu(X)\le 1$ for every cylinder $X$.
    \item Similarly, it is easily possible to verify that $\mu(\emptyset)=0$
    and that $\mu (\{\zero, \one\}^A)=1$.
    \item Finally, suppose that $A_1, A_2,\ldots$ is a sequence of pairwise
    disjoint sets such that
    $A=\bigcup_{i = 1}^\infty A_i$ is a cylinder, then $A$ can be expressed as follows:
    $$
    A = \{\omega \in \{\zero, \one\}^A | \omega|_K \in \bigcup_{i=1}^\infty H_i\},
    $$
    with $H= \bigcup_{i=1}^\infty H_i$ finite. Moreover,
    all the $H_i$s are finite and pairwise disjoint.
    So
    $$
    \mu(A) = \frac{|\bigcup_{i=1}^\infty H_i|}{2^{|K|}}=\frac{\sum_{i=1}^\infty|H_i|}{2^{|K|}}=\sum_{i=1}^\infty\frac{|H_i|}{2^{|K|}}=\sum_{i=1}^\infty\mu(H_i).
    $$
  \end{itemize}
  Finally, applying \cite[Theorem 3.1]{Billingsley}, we get the claim.
\end{proof}

\subsection{The Class $\SFP$}
\label{app:c1}

\begin{proof}[Proof of Proposition~\ref{prop}]
The proof is by induction on $n$.

$n=0$. Then, $\m{\triangleright^n_M}$ is the identity function.

$n+1$. By IH $\m{\triangleright^n_M}$ is a function.
Since $\vdash_{\m{\delta}}$ is a function,
$\m{\triangleright^{n+1}_M} =
\m{\triangleright^n_M} \circ \vdash_{\m{\delta}}$
is a function too.
\end{proof}


































%%%%% PROOF 2
%\subsection{Section~\ref{sec:SFPtoPOR}}

%\begin{proof}[Proof of Lemma~\ref{lemma:SFPchar}]
%$\Rightarrow$ Trivial.  Let us use the transition function of the  underlying machine $M$ as $\delta$ and $q$ as its initial state.
%
%Such objects exist due to  Definition~\ref{df:streamMachine},~\ref{df:PSTM},~\ref{df:SFP}.
%
%The polynomial $p$ is the time-bound of $M$, which exists according to Definition~\ref{df:PSTM}.

%$\Leftarrow$ The characterization ensures that the $\overline{\triangleright}^\delta_n$ is stationary after a polynomial number of steps.
%
%So, we can build a $\delta'$ function which is obtained  from $\delta$ pruning the self-loops between states. So, $\delta'$ and $q$ allows us to synthesize a STM $M'$, which is an $\SFP$-machine  by  Definition~\ref{df:streamMachine},~\ref{df:PSTM},~\ref{df:SFP}.
%\end{proof}







% \paragraph{\emph{Characterizing $\SFP$}}
%
% \begin{proof}[Proof of Lemma~\ref{lemma:transitionFunc}]
%   By induction on $n$, leveraging the fact that even $\vdash'_\delta$ is a function, which itself is a consequence of the fact thhat $\delta$ is a function.\end{proof}
%
%
%
%
% \begin{proof}[Proof of Lemma~\ref{lemma:char}]
% {If $f\in \fcob$, then
% there is a poytime TM $M$, which computes it.
% %
% It suffices to take the define $\delta$
% as the machine's transition function and
% $p$ as the its time-bound.
% %
% By definition, this if the $M$ halts at step
% $k<p(|x|)$, then $\overline{\stmreach}$ is constant
% for $k>k'$. This proves the claim.
% %
% Suppose that there are such $\delta$ and $p$,
% then the canonical TM which uses $\delta$
% as transition function computes $f$}
% \end{proof}









































































%\subsection{Section~\ref{sec:H}}







































%\newpage
\subsection{From $\polyF$ to $\POR^-$}
\label{app:secg}











\begin{lemma}
  \label{lemma:simcorr}
  If $\gamma$ is a function $\Ss \longrightarrow \Ss$ such that $|\gamma|=n$,
   define $x_\gamma = \listenc {\listenc {x_1, y_1} 2, \ldots,
  \listenc {x_n, y_n} 2} n$,  then
  $\simulate(x, x_\gamma, \omega)=\gamma(x)$,
  otherwise $\simulate(x, x_\gamma, \omega)=\eepsilon$.
\end{lemma}
\begin{proof}
  Let $\omega \in \Os$ any oracle.
  Suppose $x'$ is in $\gamma$'s domain. It means that there is a tuple
  $t=\listenc{x', \overline y} 2$ among the elements of $x_\gamma$, so there is
  $k \in \Nat$ such that $t=\pi_k(x_\gamma, \omega)$.
  Thus, we can prove that if the third argument of $\simulate'$
  is the number of elements in $t$, then
  $\forall k\le m < \pi_0 (t, \omega). \simulate'(x_\gamma, x,
  \ovverline m \Nat, \omega)=\overline y$; this result can be established by induction
  on $m$. Leveraging the fact that $\gamma$ is a function, so in its graph there
  cannot be two pairs with the same first projection.
  Then, the correctness of $\simulate$ comes as consequence because it is
  an instance of $\simulate'$ with a value which is greater or equal to any
  possible value of $k$, namely the number of pairs in $x_\gamma$.
  Now suppose that $x'$ is not in $\gamma$'s domain,
  then there is not a $k$ such that
  $\pi_k(x_\gamma)=\listenc{x', \overline y} 2$, so $\simulate'$ returns
  $\eepsilon$.
\end{proof}

\begin{defn}[Apply]
The function $\apply(\cdot,\cdot,\omega)\in \POR^-$
is defined as follows:
$$
\apply(x_\delta,x_c,\omega) := \mathtt{if}\big(x_c, g(x_c,x_\delta, \omega),
\lnot(eq(\xi(x_c,x_\delta,\omega),\eepsilon,\omega),\omega),
\omega\big),
$$
where:
\begin{align*}
g(x,y,\omega) &= \mathtt{if}\big(f_{1}(x,y, \omega), f_{2}(x,y, \omega),
eq(\xi(x,y,\omega),\zero,\omega),
\omega\big) \\
%
%
\\
f_{1}(x,y, \omega) &= \langle \chi(rrl(\pi_1(x,\omega),\omega),
\omega), \pi_3(\xi(x,y,\omega),\omega), \\
%
& \ \ \ lal(lal(lrl(\pi_3(x,\omega),\omega),
\pi_1(\xi(x,y,\omega)),\omega), rel(\pi_1(x,\omega),\omega),\omega) \\
%
& \ \ \ \lrl (\pi_4(x,\omega),\omega)\rangle_\Ll \\
%
f_{2}(x,y, \omega) &= \langle ral(rrl(\pi_1(x,\omega),\omega),
\pi_1(\xi(x,y,\omega),\omega),\omega),
\pi_3(\xi(x,y,\omega),\omega), \\
%
& \ \ \ \chi(lrl(\pi_3(x,\omega),\omega),\omega),
\lrl (\pi_4(x,\omega),\omega)\rangle_\Ll \\
%
\\
\chi(x,\omega) &= \mathtt{if}\big(\underline \circledast_\Tt,
x, eq(\pi_0(x,\omega), \underline{0}_\Nat,
\omega),\omega\big) \\
%
\xi(x,y,\omega) &= sim\big(y,\langle \pi_2(x,\omega),
\lel(\chi(\pi_3(x,\omega),\omega),\omega),
\lel(\chi(\pi_4(x, \omega), \omega), \omega)\big).
\end{align*}
\end{defn}


\applycorr*
\begin{proof}
  It should not be too much of a problem to see
  that the function $\chi$ return a tape which contains an instance of the
  blank character if and only if the tape passed as argument is empty,
  otherwise it returns its argument.
  %
  Thanks to this observation, we prove that $\xi$ is defined as the
  invoking of $\simulate$ on:
  \begin{enumerate}
    \item The machine's transition function $\delta$;
    \item The tuple containing:
    \begin{enumerate}
      \item The current state $\pi_2(x_c, \omega)$;
      \item The current character $\lel(\chi(\pi_3(x_c, \omega), \omega), \omega)$;
      \item The first character on the oracle tape, i.e.
      the \emph{random} bit $\lel(\chi(\pi_3(x_c, \omega), \omega), \omega)$.
    \end{enumerate}
  \end{enumerate}
  This call to the $\simulate$ function returns either the encoding of the image of this inputs through $\delta$ (if defined), or $\eepsilon$ otherwise, as a consequence of Lemma
  \ref{lemma:simcorr}.
  %
  Suppose that the result of $\xi$ is $\eepsilon$, then the claim is
  trivially true, since the function returns exactly its input, as required by the claim.
  %
  Otherwise, suppose that $\xi$ returns a value different from $\eepsilon$.
  It means that $\delta$ is defined on its input, so, the guard of the inner
  $\mathtt{if}$ expression checks whether the result describes a right or
  left movement of the head and, depending on the response, behaves
  differently. Observe that the constant $L$ is represented by $\zero$
  as stated in Definition \ref{def:canencs}.
  %
   We will show only the case in which the head moves right, for the left
   movement the proof is analogous. The configuration is made up as follows:
   \begin{enumerate}
     \item The left portion of the tape tape $\ral(\rrl(\pi_1(x_c,\omega),\omega),
    \pi_1(\chi(x_c, x_\delta, \omega),\omega))$
     is obtained removing the current character by means of
     $\rrl(\pi_1(x_c,\omega),\omega)$, appending
     the character which is written by the head
     $\pi_1(\chi(x_c, x_\delta, \omega),\omega)$ by means of $\ral$.
     \item The current state is obtained projecting
     the third element of $\chi$.
     \item The right portion of the work tape tape loses its leftmost element, becoming
     $\ovverline \circledast \Tt$ if empty, as described by
     $\chi(\lrl(\pi_3(x_c,\omega),\omega), \omega)$.
     \item The second tape loses its leftmost character.
   \end{enumerate}
   The function $\chi$ prevents to obtain empty list as encodings of a tape,
   as required by Definition \ref{def:canencs}.
  \end{proof}







\saPOR*

\begin{proof}
  \label{proof:saPOR}
Given $f\in \POR^-$ and $t\in \mathcal{L}_{\mathbb{PW}}$,
let $sa_{f,t}$ be defined as follows:
\begin{align*}
sa_{f,t}' (x,\eepsilon, \vec{z}, \omega) &:= x \\
sa_{f,t}' (x,y\bool, \vec{z}, \omega) &:= f\big(sa_{f,t}'(x,y,\omega),
\vec{z},\omega\big)|_{t} \\
\\
sa_{f,t}(x,y,\vec{z},\omega) &:= sa_{f,t}'(x, rrs(y,\omega),\vec{z},\omega).
\end{align*}
{The function $sa$
is correct as $\underline{n}_\Nat$
has size $n+1$.}
%
We prove that if $|y| = n$,
then $sa_{f,t}'(x,y,\vec{z},\omega)=
f(f(f(\vec{x},\vec{z},\omega), \vec{z},\omega)\dots)$,
nested $n$ times,
by induction on $n$.
\begin{itemize}
  \item If $n=0$, $sa_{f,t}$ reduces to $sa_{f,t}'$ with argument $\epsilon$,
  so the result is $x$.
\item $n+1.$ By applying IH and the definition of $sa_{f,t}'$. \\
\end{itemize}

The correctness of $\sa$ comes as a consequence of the correctness
of the function $\rrs$ (which removes the rightmost digit of a string)
and of the definition of $sa$.


\end{proof}

\applysize*
\begin{proof}
  \label{proof:applysize}
  At each step, the function $\apply$ manipulates three values:
  \begin{itemize}
    \item The portion of the tape on the left of the head;
    \item The portion of the tape on the right of the head;
    \item The current state.
  \end{itemize}
  %
  The manipulation of the tapes is done through shifting of characters and
  rewriting. An ordinary shifting operation does not change
  the overall size of the encoded configuration, while the rewriting can cause such phenomenon,
  due to the different sizes of the encoded characters, but fixed a machine,
  the maximum size of its characters is fixed. Say that value $k_1 \in \Nat$.
  Replacing a character on the encoded tape with it takes less than $2\cdot k_1$ bits.
  Even if a tape contains an infinite sequence of $\circledast$ and the machine
  moves in that direction, the overall result is the appending of a character on
  the other portion of the tape this takes exactly $2\cdot k_1$ bits, plus
  $2$ additional bits to store the new size of the portion of the tape which
  grows by one.
  %
  Then, we must take in account the difference in size due to the new state,
  but since the numbers of state in a machine is fixed, there is a constant $k_2$
  which bounds the size of any state, so rewriting a state takes less than
  $2\cdot k_2$ additional bits.
  %
  Finally, we must take in account the representation of the second tape:
  at each step, we strip off a cell from its representation since the head
  moves always on the right, so it causes no term growth.
  %
  The value of $k$ we are looking for is $2 \cdot(1+k_1+k_2)$.

\end{proof}


























































%\newpage
\begin{lemma}
  \label{lemma:dectape}
  The $\POR$ function $\dectape :
  \Ss \longrightarrow \Ss$ is
  such that if $\ovverline \cdot \Tt$ is the encoding for tapes
  proposed in Definition \ref{def:canencs}, then it holds that
  $\forall \sigma \in \{\zero, \one, \circledast\}^*, \omega.
   f(\ovverline \sigma \Tt, \omega)=
  \tau$ and $\tau$ is the longest suffix of $\sigma$ without $\circledast$.
\end{lemma}
% \begin{proof}
%   Take the function $\mathit{dectape}$ defined below:
%   \begin{align*}
%     \rho(t, x, \omega)&:= \pi(t, \mathit{diff}(\pi_0(t, \omega),
%     x, \omega), \omega)\\
%     \mathit{dectape}'(t, \eepsilon, \omega) &:= \eepsilon\\
%     \mathit{dectape}'(t, x\bbool, \omega) &:= \mathtt{if}(
%     \mathit{dectape}'(t, x, \omega)\rho(t, x\bbool, \omega),
%     \eepsilon, \lnot \eq(\rho(t, x\bbool, \omega),
%     \one\one\one, \omega), \omega)\\
%     \mathit{dectape}(t, \omega) &:= \mathit{dectape}'(t,
%     \rrs(\pi_0(t, \omega), \omega), \omega)
%   \end{align*}
%   Before proceeding with the correctness of $\mathit{dectape}$, we must prove
%   that, if $t$ is the encoding of a tape through $\ovverline \cdot \Tapes$, $x=\ovverline n\Nat$ and $n \le |t|$, $\rho(t, x, \omega)$ returns the $n$-th
%   projection of $t$ from the right. This is a consequence of the correctness
%   of the functions used in the definition of $\rho$, which has already been shown.
%   %
%   Before, we ought proof that $\mathit{dectape}'$ is a generalization of dectape,
%   for which $x=\ovverline n\Nat \to \ovverline \sigma \Tapes = t \to
%   \mathit{dectape}'(t, x, \omega)$ is the longest substring whithout
%   $\circledast$ of $\sigma$ which ends in
%   the $n$-th character of $\sigma$ from the left.
%   The proof of the main claim is by induction on $n$:
%   \begin{itemize}
%     \item[$0$] According to what we proved above about $\rho$, and the fact the
%     function checks the rightmost character of the $\sigma$'s encoding. Moreover,
%     the encoding of any encoded $\sigma \in \Sigmab^*$ contains at least one
%     charater. if this characteris $\circledast$ its encoding
%     is $\one\one\one$, so the condition on the $\mathit{if}$'s guard
%     is verified and the whole expression reduces to $\eepsilon$, otherwise
%     the function returns the decoding of the remaining part (which reduces
%     to $\eepsilon$) and the decoded value of the character in exam.
%     \item[$n+1$] The claim follows from the IH on the
%     recursive call, ad the condideration above regarding the value of the
%     projected character.
%   \end{itemize}
% The main claim can be proved by joining the consideration above and the fact
% that the definition of $\mathit{decode}$ is defined as an instance of
% $\mathit{decode}$ with $x$ equal to the size of $\sigma$, so the substring
% returned by $\mathit{decode}'$ is actually a suffix.
% \end{proof}
\begin{proof}[Proof of Lemma~\ref{lemma:dectape}]
Let us consider the function $\dectape(x,\omega)$, described
in Definition \ref{def:dectape}.
If $x= \underline{\sigma}_\Tt$ is the encoding of a tape $\sigma \in \Sigmab^*$,
$y=\underline{n}_\Nat$ and $n \leq |\sigma|$,
so $\rho(x,y,\omega)$ returns the $n$-th projection
of $\sigma$ from the right.
%
This is a consequence of the correctness of the function
used in the definition of $\rho$,
which we have already discussed.
%
In order to show the correctness of $\dectape$,
we prove that $\dectape'$
is a generalization of $\dectape$,
for which if $y=\underline{n}_\Nat$ with $n>0$,
and  $\underline{\sigma}_\Tt = x$, then
$dectape'(x,y,\omega)$ is the longest subfix
without $\blank$, of $\sigma$'s $n$-th prefix.
%
The proof is by induction on $n$.
\begin{itemize}
  \item Case $0$. In this claim we have with the condition $n>0$.
  \item Case $n+1$. According to the IH, the recursive call returns the longest
  suffix of the $n$-th prefix of the string. Form the consideration above, thus
  In this case, intuitively, $\rho$ checks the $n+1$-th character from the left
  of the encoding for $\sigma=\sigma'c$.
  %
  If the character $c$ is $\blank$, its encoding will be $\one\one\one$,
  so the condition in the guard of $\mathtt{if}$ is verified
  and the whole expression reduces to $\eepsilon$, which is indeed
  the longest suffix of $\sigma' \blank$ without $\blank$.
  %
  Otherwise, the function returns the decoding of the
  $\mathit{\dectape'}(x,y,\omega) = \tau$, which is the longest suffix of $\sigma'$
  without $\blank$, thus the function returns $\tau c$.
  This string is free from $\blank$ as a consequence of the IH and is also the longest
  suffix of $\sigma$ with this property: otherwise, $\tau$ would not be the
  longest suffix of $\sigma'$ free from $\blank$, contradicting the IH.
  %
\end{itemize}
\noindent
We conclude the proof by summing the consideration above
and the fact that the definition of $\dectape$
is defined as an instance of $\dectape'$
with $x$ equal to the size of $\sigma$.
So, the substring returned by $\dectape$
is actually a suffix.
\end{proof}












































\subsection{Extractor Function in $\POR$}



\begin{lemma}
  \label{lemma:dyadbij}
  The function $\mathit{dy}(\ovverline n \Nat, \omega)$ is bijective with respect to its first argument.
\end{lemma}
\begin{proof}
  By the definition of the function, we know that it is in $\POR^-$, so
  the function is constant with respect to its second parameter. Moreover,
  it is clearly an injection because different numbers have different binary encodings
  and $\forall n>0 \in \Nat.\forall \omega \in \Os. \mathit{bin}(n,\omega)$ has $\one$
  as leftmost bit
  (it can be shown by induction on $n$, leveraging the definition of $\mathit{bin}$).


  So if we take two distinct binary encodings of natural numbers $n, m$ and call them $\one\sigma$ and $\one\tau$.
  It must hold that $\sigma \neq \tau$, otherwise we would have $n=m$.

  So we just need to show that it is surjective. We know that $\mathit{dy}$
  is computed removing a bit which is always $\one$.
  This entails that taken a string $\sigma \in \Ss$,
  it is the image of the natural number $n$ such that
  the binary encoding of $n+1$ is $\one \sigma$. This number always exist.
\end{proof}

\begin{lemma}[Correctness of $e$]
  \label{lemma:corrofe}
  $\forall i \in \Nat.\forall j \le i. \forall \omega \in \Bool^\Ss. e(\ovverline i \Nat, \omega)(j) = \omega(\mathit{dy}(\ovverline j\Nat, \omega))$
  and the length of $e(\ovverline i \Nat, \omega)$ is exactly $i+1$.
\end{lemma}
\begin{proof}
  By induction on $i$:
  \begin{itemize}
    \item[0] For the first claim we have that:
    $$
    e(\one, \omega)(0)=\eepsilon Q\big(\dy(\one,\omega),\omega\big)= \omega(\dy(\one,\omega)).
    $$
    While, the second claim is trivial.
    \item[i+1] By the IH on the second claim, we get that for $j=i+1$, the
    $j$-th element of $e(\ovverline {i+1} \Nat, \omega)$ is exactly
    $Q\big(\dy(\ovverline {i+1}\Nat,\omega), \omega \big)$, which is equal to
    $\omega(\dy(\ovverline {i+1}\Nat,\omega))$. For smaller values of $j$,
    the first claim is a consequence of the definition of $e$ and the IH.
    The second claim is trivial.
  \end{itemize}\end{proof}


\begin{lemma}
  \label{lemma:auxsimdy}
  $$
  \eta \sim_{\mathit{dy}} \omega \to \forall n \in \Nat. \eta_n = e(\ovverline n \Nat, \omega).
  $$
\end{lemma}
\begin{proof}
  By contraposition: suppose that the consequence does not hold:
  $$
  \eta_n \neq e(\ovverline n \Nat, \omega).
  $$
  As a consequence of the correctness of $e$ (Lemma \ref{lemma:corrofe}),
  this means that there is an $i \in \Nat$ such that
  $\eta(i)\neq\omega(\mathit{dy}(\ovverline i\Nat, \omega))$,
  which is a contradiction.
\end{proof}



\end{document}
