%!TeX spellcheck = en-US
In the Chapter \ref{chap:RBA}, we have shown that
there is a strong correspondence between the
class of functions $\POR$ and the $\Sigma^b_1$
formul\ae{} of $\Lpw$, precisely that the
set of the $\Sigma^b_1$ representable formul\ae{} of $\Lpw$
are exactly the $\POR$ functions, (Theorem \ref{thm:TaskAB}).
% For this reason,
% if we prove that there is a similar correspondence between $\POR$
% and some standard class of functions we could be able to arithmetize
% probabilistic complexity classes such as $\BPP$.

For this reason, we are now interested in how $\POR$ relates to some other probabilistic classes of functions. Indeed,
if we prove that there is correspondence between $\POR$
and some standard class of functions, such as $\PPT$ (Definition \ref{def:pptinformal}),
we could be able to define
an arithmetical characterization of
probabilistic complexity classes such as $\BPP$ in the style of
Theorem \ref{thm:TaskAB}.
%
To this aim, we show that as it happens for Cobham's algebra \cite{Cobham1965} and $\FP$ functions,
the $\POR$ function algebra corresponds to the $\PPT$ class.
%
Moreover, since $\BPP$ is defined on top of $\PPT$,
this paves the way for determining a logical characterization of this specific complexity class,
as done in Chapter \ref{chap:characterization}.

Informal definitions of PTMs and $\PPT$ functions have already been given in Chapter
\ref{chap:preliminaries}, but to healp the reader in
walking through this section,
we will briefly recall those concepts.

%%% CLASS PPT
\defptm*
\defppt*
%
%
\noindent
However, formal definitions of PTMs and $\PPT$ will
be given in Section \ref{sub:SFPtoPPT}, respectively in
Definitions
\ref{def:ptm}
and
\ref{def:ppt}.
\noindent
Referring to Definitions \ref{def:ptminformal} and \ref{def:pptinformal},
we can state more precisely the goal of this chapter:
we show that for each $f\in \PPT$, there is
a corresponding function $g \in\POR$, such that
the probability that $f(x)=y$ is equal to the measure
of the set of $\omega \in \Os$
with $g(x, \omega)= y$.
We show that also the converse claim holds,
formally:
%
% Conjecture
\begin{conj}\label{conj1}~
\begin{enumerate}
\itemsep0em
\item[i)] For each $f\in \PPT$, there is $g \in \POR$
such that:
$$
\forall x, y. \emph{Pr}\big[f(x)=y\big] = \mu\big(\{
\omega \in \Os \ | \ g(x,\omega)=y\}\big).
$$

\item[ii)] For each $g\in \POR$, there is $f\in \PPT$
such that:
$$
\mu\big(\omega \in \Os \ | \ g(x,\omega) = y\}\big)
= \emph{Pr}\big[ f(x) =y \big].
$$
\end{enumerate}
\end{conj}
\noindent
In order to prove Conjecture \ref{conj1}, we proceed as follows:

\begin{enumerate}
 \item We define an intermediate formalism between $\POR$ and $\PPT$
 whose random choices are made \emph{explicit} by means of a stream $\eta: \Nat \longrightarrow \{\zero,\one\}$ containing an infinite sequence of bit determining the outcome of all the possible random choices. Such formalism is $\SFP$.
 It will be introduced in Section \ref{sec:SFP}, but we  prove the
 correspondence between $\PPT$ and $\SFP$ only at the end, namely
 in Section \ref{sub:SFPtoPPT}. This delay should not be too much of a problem
 because the definition of $\SFP$ is almost the same of $\PPT$ so,
 in our opinion, their equivalence is almost trivial.
 \item On top of the definition of $\SFP$, we define its encoding in $\POR$
 and prove it correct. This is carried out in Section~\ref{sec:SFPtoPOR}.
 \item Finally, we prove that each function in $\POR$ can be encoded in $\SFP$,
 too. This reduction is presented in Section~\ref{sec:PORtoSFP} and
 steps through three intermediate formalisms aimed to separate the different technical
 concerns of this proof.
\end{enumerate}

\noindent
The overall picture of the reduction is described by Figure \ref{fig:redschema}.

\begin{figure}
  \begin{resizebox}{\textwidth}{!}{%
    \begin{tikzpicture}
      \def \radius{7cm}
    %  \node (POR) {$\POR$};
    %  \node[right = 10cm of POR] (SFP) {$\SFP$};
    %  \node[below right = 2cm and 1cm of POR] (SIFPRA) {$\SIFPRA$};
    %  \node[below right = 1cm and 1cm of SIFPRA] (SIFPLA) {$\SIFPLA$};
    %  \node[above right = 1cm and 2cm  of SIFPLA] (SFPOD) {$\SFPOD$};
      \node at ({-30}:\radius) (SFP) {$\SFP$};
      \node at ({-50}:\radius) (SFPOD) {$\SFPOD$};
      \node at ({-90}:\radius) (SIFPLA) {$\SIFPLA$};
      \node at ({-130}:\radius) (SIFPRA) {$\SIFPRA$};
      \node at ({-150}:\radius) (POR) {$\POR$};

      \draw[->] (SFP) edge node[fill=white] {Section \ref{sec:SFPtoPOR}}(POR);

      \draw[->] (POR) edge[bend right=15] node[fill=white] {\ref{sub:portosifpra}}(SIFPRA);
      \draw[->] (SIFPRA) edge[bend right=15] node[fill=white] {\ref{sub:sifpratosifpla}}(SIFPLA);
      \draw[->] (SIFPLA) edge[bend right=15]node[fill=white] {\ref{sub:sifplatosfpod}} (SFPOD);
      \draw[->] (SFPOD) edge[bend right=15] node[fill=white] {\ref{sub:sfpodtosfp}}(SFP);

      \node[right = 5cm of SFP] (PPT) {$\PPT$};

      \draw[<->] (PPT) edge node[fill=white] {Section \ref{sub:SFPtoPPT}}(SFP);
      %\draw[->] (SFP) edge (PPT);
    \end{tikzpicture}
    }%
  \end{resizebox}
  \caption{Schema of the Reductions from $\POR$ from $\PPT$ and vice-versa}
  \label{fig:redschema}

\end{figure}
