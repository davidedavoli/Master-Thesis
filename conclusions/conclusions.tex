%!TeX spellcheck = en-US
This research was aimed to address the study of probabilistic
complexity classes by means of a bounded arithmetic.
To this end, we defined the function algebra $\POR$ (Section \ref{sec:POR})
and we introduced the $\Lpw$ first-order language and the $\RS$
bounded arithmetic (Section \ref{sec:S13}),
together with a notion of $\Sigma^b_1$-representability within $\RS$,
establishing that the $\POR$ functions are exactly the
$\Sigma^b_1$-representable functions of $\RS$, Theorem \ref{thm:TaskAB}.
%
In particular, this result relies on the extension of the standard \emph{qualitative}
semantics of first-order formul\ae{} with the introduction of a
non-standard \emph{quantitative} semantics (Definition \ref{def:quantsem})
associating to each formula
a measurable set. This novel semantics is employed in the notion of
$\Sigma^b_1$-representability to describe the probability distributions
computed by the represented functions.


Probabilistic complexity classes are usually defined basing on
the class of Probabilistic Turing Machines. For this reason,
we showed that the $\POR$ functions --- and consequently the class of
$\Sigma^b_1$-representable functions --- are exactly the
PTM poly-time computable functions ($\PPT$).

The proof of this result required some effort. First, to bridge the gap
between the definitions of the classes $\POR$ and $\PPT$,
we defined an intermediate class of functions $\SFP$,
which is grounded on the TM-like paradigm of the Stream Turing Machines
and is meant to be equivalent to both $\POR$ and $\PPT$. Then, we proposed a series of reductions connecting $\POR$ and $\SFP$.

The reduction from $\SFP$ to $\POR$ (Section \ref{sec:SFPtoPOR}) is
quite straightforward: it relies on the fact that
in $\POR$ it is possible to define a function which,
taking in input the initial configuration
of a polynomial Stream Turing Machine,
outputs the corresponding final configuration.

On the other hand,
we encountered some obstacles attempting a
direct reduction from $\POR$ to $\SFP$  (Section \ref{sec:PORtoSFP}).
The biggest has to do with the different ways adopted by
$\POR$ and $\SFP$ to access randomness.
In particular, the reduction of each $\POR$ function
relies to at most to exponentially many
random bits in the size of their input,
while $\SFP$ functions can access only
polynomially many random bits during their reduction.
%
Our proof relies on three intermediate formalisms
which were introduce to bridge the separate the paradigm
related concerns of the reduction --- indeed we were
reducing a functional paradigms towards a TM-like one ---
from the probabilistic aspects of the reduction process.

At the end of Chapter \ref{chap:sfptopor},
we managed to show that a form of equivalence holds
between $\POR$ and $\SFP$. Finally, we also managed to show that
the class of functions $\SFP$ is equivalent to the $\PPT$ class (Proposition \ref{prop:ptm=sfp}).
Thanks to this last result, we manage to extend the $\Sigma^b_1$ representability
result (Theorem \ref{thm:TaskAB}) from the class of $\POR$ function to the
class of $\PPT$ functions (Theorem \ref{thm:TaskABC}).

After Chapter \ref{chap:sfptopor}, we employed the
large amount of technical results given in those pages to
prove, as a corollary, the equivalence of the
Cobham style function algebra $\fcob$ and the class $\FP$.
This was done because similar results are widely agreed among the literature,
although being almost completely folklore: as far as we know,
up to now, there were not exhaustive and self-contained
proofs of this result.
%
In this case, the proof was quite simple: the first inclusion
relies on the fact that
$\POR$ is a generalization of $\fcob$, but that
all the non-random $\POR$ functions have an equivalent $\FP$ function.
Similarly, the second inclusion relies on the observation that
$\SFP$ is a generalization of $\FP$ and that under the hypothesis
that an $\SFP$ function does not depend on random choices, it
is in $\fcob$ as well.


After we have found a characterization of all the $\PPT$
functions within the language $\Lpw$,
in Chapter \ref{chap:characterization}, we addressed
the problem of defining characterization of thinner
probability complexity classes, such as $\BPP$, $\RP$,
co-$\RP$ and $\ZPP$.
To this aim, we extended the $\Lpw$'s language and semantics with
a non-standard measure quantifiers $\cq{s/t}$,
whose quantitative semantics has measure equal to $1$ if and only if the
semantics of the formula they are quantifying
has measure greater or equal to $\frac{|s|}{|t|}$.
The language thus obtained has been named $\Lmq$ and
was proved being a conservative extension of $\Lpw$, Remark \ref{rem:lpwlmq}.

Due to this extension, we were capable to reformulate the definitions of $\BPP$,
$\RP$, co-$\RP$ within $\Lmq$. Then, these characterizations have
been employed together with a standard result of probabilistic complexity theory ---
$\ZPP = \RP \cap$ co-$\RP$ (Theorem \ref{thm:zpprpcorp1})  --- to develop a characterization of  $\ZPP$.
% In our opinion, this proof is particularly interesting not due to the result
% itself, which is quite easy by itself, but due to the high level of abstraction
% of the proof, which relies only on the characterizations we defined and on
% some simple logical observations.

\section{Future work}

In the next months, we plan to investigate the characterizations of probabilistic
complexity classes given in Chapter \ref{chap:characterization} in order to determine whether
it is possible to develop equivalent characterizations within
a standard first-order arithmetical language. This would pass through
reduction of the measure quantifier $\cq{s/t}$ to a standard existential quantifier
and the reduction of the predicate $\Flip$ to a first-order formula.

A similar result would pave the way to the identification of a \emph{recursively enumerable} subset
of those non-recursively enumerable complexity classes. Indeed, none of $\BPP$, $\RP$,
co-$\RP$ and $\ZPP$ are known to be such and thus are considered
\emph{semantical} complexity classes.

To do so, we plan to reduce our non-standard logic $\Lmq$
to a standard first-order logic.
This would also cause the reduction of the \emph{quantitative} semantics
to a \emph{qualitative} semantics.

%For example, take in exam $\BPP$:
Concerning the reduction of the non-standard quantifiers, we conjecture that,
due to the syntactical constraints on
the formul\ae{} within our characterizations, they
can be reduced to a threshold-existential quantifier, \cite{Gradel}.
Intuitively, this quantifier, denoted by $\exists^\ge{k}x. F$
holds if and only if there are at least $k$ interpretations of the variable $x$
which satisfy the formula $F$. We believe that this reduction is possible:
intuitively, stating
an assertion similar to
$\mu(\llbracket F \rrbracket)\ge \frac s t$ is equivalent to stating that
``over $N$ classes of $\Flip$ possible interpretations, at least $\frac s t$
of them satisfy the quantified formula''. Thus, the quantitative semantics of the
$\Lmq$ would be encoded inside the language of the logic enriched with threshold quantifiers.
This would come with the substitution of the $\Flip(t)$ predicate with a new formula
$\phi(t, s)$ which is equivalent to $\Flip(t)$
under the hypothesis that $s$ encodes an interpretation of such
predicate.
%
% On the other hand, we are planning to leverage the variable introduced by
% the counting quantifier $\exists^{\ge k}z.$
% To replace the predicate $\Flip(x)$ with a formula $\phi(x, z)$.
Intuitively, we want this formula $\phi$ to mean
``if the interpretation of flip is $s$, then $\Flip(t)$ would hold''.
%
Precisely, we plan to define $\phi$ leveraging the encoding of
lists and functions of Section \ref{subsub:encodings}, thus we could define
the predicate as a $\POR^-$ function and leverage Theorem \ref{theorem1}
to conclude that the $\phi$ formula we are aiming to exists.

Subsequently, we plan to adapt some works for counting quantifiers elimination
--- such as for example \cite{Schweikardt, Chistikov2021PresburgerAW}.
Those works describes procedures for eliminating counting quantifiers from
Presburger's Arithmetic. Even if Presburger's Arithmetic is a small
and even decidable fragment of Peano Arithmetic, we think that
it is possible to develop similar techniques
aimed to \emph{reduce} counting quantifiers to standard existentials
instead of eliminating them. Moreover, we are confident about the possible
outcomes of this process because we are facilitated by working in
the peculiar context of our characterizations
rather than in a general one.

Once the characterizations proposed in Chapter \ref{chap:characterization}
will be expressed in the language of first-order logic, it will be possible
to identify recursively enumerable subsets of probabilistic semantic classes by means of
the provability relation $\vdash$. For instance, assume that $\BPP$ can be described
by means of a class of first-order formul\ae{} $\{\psi_i\}_{i \in \Nat}$,
in the sense that each of these formul\ae{} corresponds to a problem in $\BPP$,
then it would be possible to choose a proof system --- for example $\PA$'s axioms ---
and to recursively enumerate all the derivations proving one
of those formul\ae{}, namely all the $\psi_i$ such that $\PA \vdash \psi_i$.
All the formul\ae{} enumerated this way would correspond
to a language in $\BPP$. Let $\PBPP$ be this class of languages:
it would certainly hold that $\PBPP\subseteq \BPP$. However,
we believe that $\BPP \not \subseteq \PBPP$:
contrarily, we would be surprised discovering a similar result, due to the
incompleteness of $\PA$ and to the widely agreed
conjecture that $\BPP$ is indeed a non recursively enumerable complexity class \cite{AroraBarak}.

Finally, it would be worth to investigate the set $\PBPP$ under an
extensional perspective: which $\BPP$
problems are in $\PBPP$? Which problems are in $\PBPP \setminus \BPP$?
