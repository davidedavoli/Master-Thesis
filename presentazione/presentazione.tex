\documentclass[xcolor={x11names}]{beamer}
\usepackage[utf8]{inputenc}
\usepackage{mathtools}
\usepackage{amsmath}
\usepackage{stmaryrd}
\usepackage{bm}
\usepackage{verbatim}
\usepackage[super]{nth}
\usepackage{ebproof}
\usepackage{tikz}
\usepackage[style=authoryear, maxcitenames=2]{biblatex}
\usepackage{multicol}
\usepackage[italian]{babel}
\usepackage{graphicx}
\usepackage{xcolor}


\usetikzlibrary{arrows}
\usetikzlibrary{calc}
\usetikzlibrary{positioning}
\usetikzlibrary{fit}
\usetikzlibrary{chains}
\usetikzlibrary{decorations.pathmorphing}

\newcommand{\query}{\emph{Q}}
\newcommand{\Sf}{\emph{S}}
\newcommand{\Cf}{\emph{C}}

\newcommand{\POR}{\mathcal{POR}}
\newcommand{\FOL}{\mathsf{FOL}}
\newcommand{\MQPA}{\mathsf{MQPA}}
\newcommand{\SIFP}{\mathsf{SIFP}}
\newcommand{\RA}{{\mathsf{RA}}}
\newcommand{\LA}{{\mathsf{LA}}}
\newcommand{\SIFPRA}{\mathsf{SIFP_{\RA}}}
\newcommand{\SFPOD}{\mathbf{SFP_{\mathbf{OD}}}}
\newcommand{\SIFPLA}{\mathsf{SIFP_{\LA}}}
\newcommand{\lang}[1]{{\mathcal L ({#1})}}


\newcommand{\Bool}{\mathbb{B}}
\newcommand{\Ss}{\mathbb{S}}
\newcommand{\Os}{\mathbb{O}}
\newcommand{\Nat}{\mathbb{N}}

%\newcommand{\zzero}{\mathbf{0}}
%\newcommand{\oone}{\mathbf{1}}
\newcommand{\zzero}{\zero}
\newcommand{\oone}{\one}
\newcommand{\eepsilon}{\bm{\epsilon}}
\newcommand{\cconc}{\bm{\conc}}
%\newcommand{\ttimes}{\bm{\times}}

\newcommand{\ssubseteq}{\bm{\subseteq}}

\newcommand{\zero}{\mathtt{0}}
\newcommand{\one}{\mathtt{1}}
\newcommand{\bool}{\mathtt{b}}
\newcommand{\bbool}{\mathtt{b}}


\newcommand{\tTerm}{{\mathbf t}}
\newcommand{\fTerm}{{\mathbf f}}
\newcommand{\vTerm}{{\mathbf v}}
\newcommand{\bbT}{{\mathbf b}}
\newcommand{\uT}{{\mathsf u}}
\newcommand{\vT}{{\mathsf v}}
\newcommand{\arrowT}{\Rightarrow}
\newcommand{\RS}{{\mathit{RS}^1_2}}


\newcommand{\Lpw}{\mathcal{L}}
\newcommand{\Lmq}{\mathcal{L}^{\mathsf{MQ}}}
\newcommand{\Lbuss}{\mathcal{L}_{\mathsf{Buss}}}

\newcommand{\Flip}{\mathtt{Flip}}

\newcommand{\midd}{\; \; \mbox{\Large{$\mid$}}\;\;}
\newcommand{\conc}{\frown}

\newcommand{\suc}{\mathtt{S}}
\newcommand{\PA}{\mathsf{PA}}

\newcommand{\DIA}{\mathbf{D}}
\newcommand{\BOX}{\mathbf{C}}

\newcommand{\longv}[1]{{}}

\newcommand{\rt}[1]{\(#1\)-Reduction-Tree}
\newcommand{\RT}[3]{\mathit{RT}_{#1}(#2, #3)}



\newcommand{\id}{\mathsf{Id}}
\newcommand{\stm}{\mathsf{Stm}}
\newcommand{\xp}{\mathsf{Exp}}
\newcommand{\fl}[1]{\mathsf{Flip(}#1\mathsf{)}}
\newcommand{\rb}{\mathsf{RandBit()}}
\newcommand{\while}[2]{\mathbf{while}(#1)\{#2\}}
\newcommand{\If}[2]{\mathbf{if}(#1)\{#2\}}
\newcommand{\sk}{\mathbf{skip};}
\newcommand{\halt}{\mathbf{halt};}
\newcommand{\takes}{\leftarrow}
\newcommand{\store}{\Sigma}
\newcommand{\as}[2]{[#1 \leftarrow #2]}
\newcommand{\ssos}{\triangleright}
\newcommand{\sred}{\rightharpoonup}
\newcommand{\ccrt}{\mathit{ccrt}}

\newcommand{\leadstola}{{\leadsto_\LA}}
\newcommand{\leadstora}{{\leadsto_\RA}}
\newcommand{\leadstolan}[1]{{\leadsto_\LA^{#1}}}
\newcommand{\leadstoran}[1]{{\leadsto_\RA^{#1}}}

%\newcommand{\LL}{\mathfrak L}
\newcommand{\LL}[1]{{\mathfrak L_{#1}}}

\newcommand{\MM}{\mathfrak M}
\newcommand{\copyb}{\mathit{copyb}}
\newcommand{\trunc}{\mathit{trunc}}

%\newtheorem{theorem}{Theorem}
\newtheorem{prop}{Proposition}
\newtheorem{remark}{Remark}
%\newtheorem{lemma}{Lemma}
\newtheorem{cor}{Corollary}
%\newtheorem{example}{Example}


\theoremstyle{definition}
\newtheorem{defn}{Definition}
\newtheorem{notation}{Notation}
\newtheorem{ex}{Example}













%%%%%%%%%%%%
%%%% CONJECTURE


\newcommand{\df}{\mathsf{d}}

\renewcommand{\C}{\mathbf{C}}
\newcommand{\cq}[1]{\mathbf{C}^{#1}}
\newcommand{\dq}[1]{\mathbf{D}^{#1}}

\newcommand{\BPP}{\mathbf{BPP}}
\newcommand{\PBPP}{\mathbf{PBPP}}
\newcommand{\RP}{\mathbf{RP}}
\newcommand{\ZPP}{\mathbf{ZPP}}
\newcommand{\PP}{\mathbf{PP}}

%\newcommand{\PTM}{\mathbf{PTM}}
\newcommand{\PPT}{\mathbf{PPT}}
\newcommand{\FBPP}{\mathbf{FBPP}}
\newcommand{\SIMP}{\mathbf{SIMP}}
\newcommand{\SFIP}{\mathbf{SFIP}}
\newcommand{\BRS}{\mathbf{BRS}}

\newcommand{\SFP}{\mathbf{SFP}}
\newcommand{\FP}{\mathbf{FP}}
\newcommand{\Lstm}{L(\mathsf{Stm})}


\theoremstyle{theorem}
\newtheorem{conj}{Conjecture}
\newtheorem{characterization}{Characterization}







%% TASK B
\newcommand{\PORo}{\POR^\lambda}

\newcommand{\sT}{\mathsf{s}}

\newcommand{\zT}{\mathsf{0}}
\newcommand{\oT}{\mathsf{1}}
\newcommand{\eT}{\mathsf{\epsilon}}
\newcommand{\bT}{\mathsf{b}}
%% ??
\newcommand{\cdotT}{\cdot}
\newcommand{\Tail}{\mathsf{Tail}}
\newcommand{\Trunc}{\mathsf{Trunc}}
\newcommand{\Cond}{\mathsf{Cond}}
\newcommand{\flipcoin}{\mathsf{Flipcoin}}
\newcommand{\Rec}{\mathsf{Rec}}

\newcommand{\TT}{\mathtt{T}}
\newcommand{\FF}{\mathtt{F}}
\newcommand{\GG}{\mathtt{G}}
\newcommand{\HH}{\mathtt{H}}
\newcommand{\KK}{\mathtt{K}}
\newcommand{\CC}{\mathtt{C}}

\newcommand{\PV}{\mathsf{PV}^\omega}


\newcommand{\BT}{\mathsf{B}}
\newcommand{\BNeg}{\mathsf{BNeg}}
\newcommand{\BOr}{\mathsf{BOr}}
\newcommand{\BAnd}{\mathsf{BAnd}}
\newcommand{\Eps}{\mathsf{Eps}}
\newcommand{\boolT}{\mathsf{Bool}}
\newcommand{\Zero}{\mathsf{Zero}}
\newcommand{\Eq}{\mathsf{Eq}}
\newcommand{\Times}{\mathsf{Times}}
\newcommand{\Sub}{\mathsf{Sub}}
\newcommand{\Conc}{\mathsf{Conc}}

\newcommand{\subseteqT}{\subseteq}


\newcommand{\ooverline}[1]{\overline{\overline{#1}}}
\newcommand{\func}[1]{f_{\mathrm{#1}}}

\newcommand{\NP}{\mathbf{NP}}

\newcommand{\varO}{\mathbf{x}}
\newcommand{\varT}{\mathbf{y}}
\newcommand{\termO}{\mathbf{t}}
\newcommand{\termT}{\mathbf{u}}
\newcommand{\termF}{\mathbf{w}}


\newcommand{\realize}{\ \circledR\ }

\newcommand{\EM}{EM}
\newcommand{\MP}{MP}

\newcommand{\enc}{\sharp}
\newcommand{\app}{\mathsf{app}}


\newcommand{\topO}{\mathbbm{1}}

\newcommand{\PIND}{PIND}
\newcommand{\uTerm}{\mathtt{U}}


%%% TASK C

\newcommand{\fcob}{\mathcal{F}_{\mathsf{Cob}}}
\newcommand{\cob}[1]{{#1}_{\mathcal{D}}}


\newcommand{\MS}{M_S}
\newcommand{\Qs}{\mathcal{Q}}
\newcommand{\sstar}{\circledast}
\newcommand{\reaches}[2]{\triangleright^{#1}_{#2}}
\newcommand{\Sigmab}{\hat{\Sigma}}

\newcommand{\POLY}{\mathsf{POLY}}
\newcommand{\MSFP}{M_\SFP}
\newcommand{\Dd}{\mathbb{D}}

\newcommand{\Oone}{\overlineN{\oone}}
\newcommand{\Zzero}{\overlineN{\zzero}}

\newcommand{\overlineN}[1]{\widetilde{#1}}

\newcommand{\Star}{*}

\newcommand{\Tuples}{\mathbb{T}}

\newcommand{\ovverline}[2]{{\underline{#1}}_{#2}}
\newcommand{\rmr}{\mathit{rmr}}
\newcommand{\rml}{\mathit{rml}}
\newcommand{\addr}{\mathit{addr}}
\newcommand{\addl}{\mathit{addl}}
\newcommand{\enct}{\mathit{enct}}
\newcommand{\tenc}{\mathit{tenc}}
\newcommand{\matcht}{\mathit{matcht}}
\newcommand{\apply}{\mathit{apply}}
\newcommand{\step}{\mathit{step}}
\newcommand{\eval}{\mathit{eval}}

\newcommand{\Const}{\Sf}
\newcommand{\concat}{\mathit{concat}}
\newcommand{\rv}{\mathit{rv}}
\newcommand{\reverse}{\mathit{reverse}}
\newcommand{\ras}{\mathit{ras}}
\newcommand{\las}{\mathit{las}}

\newcommand{\rel}{\mathit{rel}}
\newcommand{\lel}{\mathit{lel}}
\newcommand{\ral}{\mathit{ral}}
\newcommand{\lal}{\mathit{lal}}
\newcommand{\rrl}{\mathit{rrl}}
\newcommand{\lrl}{\mathit{lrl}}
\newcommand{\srrl}{\mathit{srrl}}
\newcommand{\simulate}{\mathit{sim}}

\newcommand{\Succ}{\Sf}
\newcommand{\Succb}{\Sf_2}
\newcommand{\pd}{\mathit{pd}}
\newcommand{\op}{\mathit{op}}
\newcommand{\mult}{\mathit{mult}}

\newcommand{\odd}{\mathit{odd}}
\newcommand{\even}{\mathit{even}}
\newcommand{\eq}{\mathit{eq}}

\newcommand{\lst}{\mathit{lst}}
\newcommand{\fst}{\mathit{fst}}

\newcommand{\rmsep}{\mathit{rmsep}}

%\newcommand{\rt}[2]{\(#1\)-Reduction-Tree}
%\newcommand{\RT}[3]{\mathit{RT}_{#1}(#2, #3)}

\newcommand{\rrs}{\mathit{rrs}}
\newcommand{\res}{\mathit{res}}
\newcommand{\lrs}{\mathit{lrs}}
\newcommand{\les}{\mathit{les}}

\newcommand{\sa}{\mathit{sa}}

\newcommand{\dectape}{\mathit{dectape}}

\newcommand{\Ll}{\mathbb{L}}

\newcommand{\Tt}{\mathbb{T}}

\newcommand{\blank}{\circledast}

\newcommand{\stmreach}{\lvert\triangleright}
\newcommand{\stmtrans}{\Vdash}

\newcommand{\tmreach}{\triangleright}
\newcommand{\tmtrans}{\vdash}

%\newcommand{\reaches}[2]{\triangleright^{#1}_{#2}}
\newcommand{\reachesf}[2]{{\overline{\triangleright}}^{#1}_{#2}}



\newcommand{\m}[1]{\textcolor{black}{#1}}

\newcommand{\g}[1]{\textcolor{black}{#1}}

\newcommand{\polyF}{\mathcal{PTF}}
\newcommand{\Id}{\mathit{Id}}

\newcommand{\tmstep}{\vdash}


\newcommand{\listenc}[2]{\langle{#1}\rangle^{#2}_{\Ll}}




%%% conditional printing

\newcommand{\Df}{\mathcal{D}}
\newcommand{\Hf}{\mathcal{H}}


\newcommand{\PL}{\mathbf{PL}}

\newcounter{short}
\setcounter{short}{0}
\newcounter{appendix}
\setcounter{appendix}{1}
\newcounter{extended}
\setcounter{extended}{2}


\newcounter{num}
\setcounter{num}{10}
\newcounter{commenting}
\setcounter{commenting}{0}

\newcommand{\always}{0}
\newcommand{\everything}{1}
\newcommand{\appendixorsup}{2}
\newcommand{\shortonly}{3}
\newcommand{\shortorsup}{4}
\newcommand{\extendedorsup}{5}
\newcommand{\onlyappendix}{6}
\newcommand{\notappendix}{7}

\newcommand{\lshort}{0}
\newcommand{\lappendix}{1}
\newcommand{\lextended}{2}
\newcommand{\leverything}{x}

\newenvironment{conditional}[1]%
        {%
            \ifthenelse{\NOT \(%
              \(\equal{#1}{\always} \AND 1 > 0\) \OR%
              \(\equal{#1}{\appendixorsup} \AND \(\equal{\level}{\lappendix} \OR \equal{\level}{\lextended} \OR \equal{\level}{\leverything}\)\) \OR%
              \(\equal{#1}{\shortorsup} \AND \(\equal{\level}{\lshort} \OR \equal{\level}{\lappendix} \equal{\level}{\lextended} \OR \equal{\level}{\leverything}\)\) \OR%
              \(\equal{#1}{\extendedorsup} \AND \(\equal{\level}{\lextended} \OR \equal{\level}{\leverything}\)\) \OR%
              \(\equal{#1}{\shortonly} \AND \(\equal{\level}{\lshort} \OR \equal{\level}{\leverything}\)\) \OR%
              \(\equal{#1}{\onlyappendix} \AND \(\equal{\level}{\lappendix} \OR \equal{\level}{\leverything}\)\) \OR%
              \(\equal{#1}{\notappendix} \AND \(\equal{\level}{\lshort} \OR \equal{\level}{\lextended} \OR \equal{\level}{\leverything}\)\)%
            \)}%
                    {\setcounter{commenting}{1}\expandafter\comment}%
                    {}%
                    }%
         {%
            \ifthenelse{\value{commenting} = 1}%
                    {\setcounter{commenting}{0}\expandafter\endcomment}%
                    {}%
          }


\AtBeginSection[]{
  \begin{frame}
  \vfill
  \centering
  \begin{beamercolorbox}[sep=8pt,center,shadow=true,rounded=true]{title}
    \usebeamerfont{title}\insertsectionhead\par%
  \end{beamercolorbox}
  \vfill
  \end{frame}
}

\AtBeginSubsection[]{
  \begin{frame}
  \vfill
  \centering
  \begin{beamercolorbox}[sep=8pt,center,shadow=true,rounded=true]{title}
    \usebeamerfont{title}\insertsectionhead\\\vspace{3ex}\large \insertsubsectionhead\par%
  \end{beamercolorbox}
  \vfill
  \end{frame}
}

\definecolor{lightgray}{gray}{0.75}

\addbibresource{bib.bib}

\deftranslation[to=italian]{Example}{Esempio}

\title{Bounded Arithmetic and Randomized Computation}
\author{Davide Davoli}
\institute{Alma Mater Studiorum - Università di Bologna}
\date{\nth{13} July 2022 - Alma Mater Studiorum - Università di Bologna}

\begin{document}
\selectlanguage{english}

\frame{\titlepage}

\section{Introduction}

\begin{frame}
  \frametitle{Implicit Computational Complexity}
  \begin{tikzpicture}[thick]
    \onslide<1->{
    \node[minimum width = 5cm, minimum height = 4cm, draw, rounded corners] (Prog) {};
    \node[fill=white, align=center]  at ([yshift=1em]Prog.90) {Programs
    %\\{\footnotesize $\{0, 1\}^*$}
    };
    \node[right = of Prog, minimum width = 5cm, minimum height = 4cm, draw, rounded corners] (Prob) {};
    \node[fill=white, align=center]  at ([yshift=1em]Prob.90) {Problems
    %\\{\footnotesize$2^{\{0, 1\}^*}$}
    };
      }
\onslide<1>{
      \node[fill = Khaki1] (P) at (Prog) {$P$};
      \node[fill = LightBlue1] (semP) at (Prob) {$\llbracket P\rrbracket$};
      \draw[->] (P) edge[bend left] node[fill=white] {$\llbracket \cdot \rrbracket$} (semP);
      }
\onslide<2->{
      \node[minimum width = 4cm, minimum height = 3cm, draw, align= center,fill=Khaki1] (EffProg) at (Prog) {Class of programs};
      \node[minimum width = 3cm, minimum height = 2cm, draw, align= center, fill=LightBlue1] (EffProb) at (Prob) {Class of \\problems};
      \draw (EffProg.north east) edge[dashed]  (EffProb.north west);
      \draw (EffProg.south east) edge[dashed]  (EffProb.south west);
      %  \draw[->] (P) edge[bend left] node[fill=white] {$\llbracket \cdot \rrbracket$} (semP);
      }
    % \node[label = {Clique}] (clique) at ($(P)+(0.5cm, 1cm)$) {\textbullet};
    % \node[label = {SAT}] (sat) at ($(P)+(-1.7cm, -0.5cm)$) {\textbullet};
    % \node[label = {Parity}] (parity) at ($(P)+(-0.2cm, -1.5cm)$) {\textbullet};
    % \node[label = {Reachability}] (reachability) at ($(P)+(1.5cm, -1.3cm)$) {\textbullet};
    % \node[label = {$f$}] at ($(P)$) (f) {\textbullet};
    %
    % \node[label = {$U_\mathit{Clique}$}] (uclique) at ($(CD)+(0.5cm, 1cm)$) {\textbullet};
    % \node[label = {$U_\mathit{SAT}$}] (usat) at ($(CD)+(-1.7cm, -0.5cm)$) {\textbullet};
    % \node[label = {$U_\mathit{Parity}$}] (uparity) at ($(CD)+(-0.2cm, -1.5cm)$) {\textbullet};
    % \node[label = {$U_\mathit{Rechability}$}] (ureachability) at ($(CD)+(1.5cm, -1.3cm)$) {\textbullet};
    % \node[label = {$U_f$}] at ($(CD)$) (uf) {\textbullet};
    %
    % \node[label = {$U_1$}] (u1) at ($(CD)+(-1.5cm, 1.1cm)$) {$\bullet$};
    % \node[label = {$U_2$}] (u2) at ($(CD)+(-1.1cm, 0.6cm)$) {$\bullet$};
    % \node[label = {$U_3$}] (u3) at ($(CD)+(1.5cm, 1cm)$) {$\bullet$};
    %
    % \draw[->] (clique) edge[dashed] (f);
    % \draw[->] (f) edge[dashed] (sat);
    %
    %
    % \draw[->] (uclique) edge[dotted, bend left=10] node[fill=white]{$\Gamma$} (clique);
    % \draw[->] (usat) edge[dotted, bend left=10] node[fill=white]{$\Gamma$} (sat);
    % \draw[->] (uf) edge[dotted, bend left=10] node[fill=white]{$\Gamma$} (f);
    %
    % \draw[->] (uclique) edge[dashed] (uf);
    % \draw[->] (uf) edge[dashed] (usat);

  \end{tikzpicture}

  \medskip

  \onslide<3->
{  {\bf Idea:} Study a complexity class describing the programs belonging to its preimage with respect to $\llbracket \cdot \rrbracket$.
}
\end{frame}

\begin{frame}
\frametitle{Purpose of a Randomized Bounded Arithmetic}

Bounded arithmetics are logical theories which allow us to:

\medskip
\begin{itemize}
\item Represent programs by means of proofs.
\item Capture complexity classes by means of classes of foruml\ae{}.
\end{itemize}
\pause
\bigskip

A bounded arithmetic with randomness would enable us to:

\medskip
\begin{itemize}
\item Capture the $\PPT$ functions.
\item Characterize some probabilistic complexity classes, e.g. $\BPP$.
\item Find a \emph{recursively enumerable} non-trivial subset of $\BPP$.
\end{itemize}


\end{frame}

\begin{frame}

\note{
Buss in his PhD thesis developed a bounded arithmetic --- called $S^1_2$ --- for the class $\FP$.
%
Buss' result an be proven, for example, passing through a function algebra, as did
by Ferreira
}
\frametitle{S. Buss' Bounded Arithmetic}
\onslide<1->
\cite{Buss} introduces a Bounded Arithmetic for the class $\FP$.
\medskip
\vspace {-0.25cm}
\onslide<2->
\begin{theorem}[Buss' Theorem \cite{Buss}]
  \label{theorem:buss}
  $f \in \FP$ if and only if there is a $\Sigma^b_{1}$ formula $G_f$ such that:
  $$
  S^1_2\vdash \forall x. \exists !y. G_f(x, y)
  $$
\end{theorem}
\begin{itemize}
  \item<3-> The proof system $S^1_2$ captures the complexity of $f$.
\end{itemize}
\bigskip
\onslide<4->

\resizebox{\textwidth}{!}{%
\begin{tikzpicture}[->, node distance=2cm, semithick, scale = 0.50]
 \node (Lpw) {$S^1_2$};
 \node[right = 2cm of Lpw] (POR) {PTCA};
 \node[right = 2cm of POR] (PPT) {$\FP$};
 \draw[->] (Lpw) edge[bend left] (POR);
 \draw[->] (POR) edge[bend left, dashed] (PPT);
 \draw[->] (PPT) edge[bend left, dashed] (POR);
 \draw[->] (POR) edge[bend left] (Lpw);
\end{tikzpicture}
}

Schema of the proof given by \cite{Ferreira88}.

\medskip
% \onslide<3->{\begin{block}{Examples}
% \vspace {-0.5cm}
% \[
% \llbracket \Flip(t)\rrbracket_{\xi} := \{\omega \in \{\zzero, \oone\}^\Ss | \
% \omega(\llbracket t\,\rrbracket_{\xi}) = \oone\}
% \]
% \end{block}
% }\onslide<4->{
% \vspace{-0.5cm}
% \[
% \begin{gathered}
% \mu\left(\llbracket \Flip(\one) \land \Flip(\zero)\rrbracket_\xi\right) = \\
% %\{\omega \in \{\zzero, \oone\}^\Ss | \ \omega(\one) = \oone\} \cap \{\omega \in \{\zzero, \oone\}^\Ss | \ \omega(\zero) = \oone\}\\
% %\mu \left(\{\omega \in \{\zzero, \oone\}^\Ss | \ \omega(\one) = \oone \land \omega(\zero) = \oone\}\right)= \frac 1 4
% \mu \left(\{\omega \in \{\zzero, \oone\}^\Ss | \ \omega(\one) = \oone\} \cap \{\omega \in \{\zzero, \oone\}^\Ss | \ \omega(\zero) = \oone\}\right)= \frac 1 4
% %\mu(\llbracket \Flip(\one) \land \Flip(\zero)\rrbracket_\xi)= \frac 1 4
% \end{gathered}
% \]
% }

\end{frame}

% \begin{frame}
% \frametitle{Definition of $\POR$}
%
% $\POR$ is obtained extending Cobham's function algebra with an argument $\omega$ and a query function $Q$. It is the smallest set of functions $\Ss^k \times \{\zzero, \oone\}^\Ss \longrightarrow \Ss$ containing:
%
% \begin{itemize}
% \item<2-> the empty function, $E(x,\omega)=\eepsilon$
% \item<3-> the Boolean successor $S_b(x,\omega)=xb$, for every $b\in\Bool$
% \item<4-> the projection functions, $P^n_i(x_1,\dots, x_n,\omega)=x_i$, for $n\in \Nat$ and $1\leq i\leq n$
% \item<5-> the conditional function
% $$
% C(x,y,\omega) = \begin{cases}
% \oone \ \ \ &\text{if } x \bm{\subseteq} y \\
% \zzero \ \ \ &\text{otherwise}
% \end{cases}
% $$
% \item<6-> the query function $Q(x,\omega)=\omega(x)$
% \end{itemize}
% \pause\pause\pause\pause\pause\pause
% and closed under composition and bounded recursion on notation.
%
%
% \end{frame}

% \begin{frame}
% \frametitle{RBA's semantics}
%
% \begin{itemize}
% \item Each possible value of the $\Flip$ predicate is represented by its characteristic function $\omega: \Ss \longrightarrow \{\zzero, \oone\}$.
% \item The semantics of each formula is the \emph{measurable} set of functions $\omega$ which satisfy it.
% \end{itemize}
%
% \pause
%
%
% \begin{block}{Examples}
% \[
% \llbracket \Flip(t)\rrbracket_{\xi} := \{\omega \in \{\zzero, \oone\}^\Ss | \
% \omega(\llbracket t\,\rrbracket_{\xi}) = \oone\}
% \]
% \end{block}
% \pause
% \[
% \begin{gathered}
% \llbracket \Flip(\one) \land \Flip(\zero)\rrbracket_\xi = \\
% \{\omega \in \{\zzero, \oone\}^\Ss | \ \omega(\one) = \oone\} \cap \{\omega \in \{\zzero, \oone\}^\Ss | \ \omega(\zero) = \oone\}\\
% \{\omega \in \{\zzero, \oone\}^\Ss | \ \omega(\one) = \oone \land \omega(\zero) = \oone\}\\[2ex]
% \mu(\llbracket \Flip(\one) \land \Flip(\zero)\rrbracket_\xi)= \frac 1 4
% \end{gathered}
% \]
%
% \end{frame}



% Definire PPT a voce

\begin{frame}
\frametitle{Main Theorem}
\note{
Ugo dal Lago and I, extended Buss' work to probabilistic computation,
this required us to modify Buss' logic adding a predicate called $\Flip$
to capture probabilistic behaviors


$RS^1_2$ is the transposition into our framework of Buss' $S^1_2$ axioms, which capture the class $P$.

...

}
\onslide<1->{
$\Lpw$ is a first-order word language including the unary predicate $\Flip$ and associated with a \emph{quantitative} semantic inspired by~\cite{ADLP}.

\pause

\begin{theorem}
The $\PPT$ functions are exactly the functions $f$ for which there exists a $\Sigma^b_1$ formula of $\Lpw$, $G_f$, such that:
% \[
% \forall x, y. \mu(\llbracket G_f\,(x, y)\rrbracket_\xi) = \mathit{Pr}[f\,(x) = y] \land \big(RS^1_2 \vdash \forall x.\exists ! y . G_f\,(x, y)\big)
% \]
\begin{itemize}
  \item $RS^1_2\vdash \forall x \exists ! y. G_f(x, y)$.
  \item The interpretation of $G_f(x, y)$ has measure equal to $\mathit{Pr}[f\,(x) = y]$.
\end{itemize}
\end{theorem}
}
\pause
\smallskip
% The proof consists in a series of reductions from $\Lpw$ to poly-time PTMs and viceversa.\\
% \pause
\bigskip
% \resizebox{\textwidth}{!}{%
% \begin{tikzpicture}[->, node distance=2cm, semithick, scale = 0.50]
%  \node (Lpw) {$\Lpw$};
%  \node[right = 2cm of Lpw] (POR) {$\POR$};
%  \node[right = 2cm of POR] (SFP) {$\SFP$};
%  \node[right = 2cm of SFP] (PPT) {$\PPT$};
%  \onslide<3>{\draw[->] (Lpw) edge[bend left] (POR);}
%  \onslide<4>{\draw[->] (Lpw) edge[bend left, very thick] (POR);}
%  \draw[->] (POR) edge[bend left] (SFP);
%  \draw[->] (SFP) edge[bend left] (PPT);
%  \draw[->] (PPT) edge[bend left] (SFP);
%  \draw[->] (SFP) edge[bend left] (POR);
% \onslide<3>{\draw[->] (POR) edge[bend left] (Lpw);}
% \onslide<4>{\draw[->] (POR) edge[bend left, very thick] (Lpw);}
% \end{tikzpicture}
% }
% \onslide<4>{Proved following \cite{CookUrquhart} and \cite{Ferreira88}.}

  \begin{resizebox}{\textwidth}{!}{%
    \begin{tikzpicture}
      \def \radius{4cm}
    %  \node (POR) {$\POR$};
    %  \node[right = 10cm of POR] (SFP) {$\SFP$};
    %  \node[below right = 2cm and 1cm of POR] (SIFPRA) {$\SIFPRA$};
    %  \node[below right = 1cm and 1cm of SIFPRA] (SIFPLA) {$\SIFPLA$};
    %  \node[above right = 1cm and 2cm  of SIFPLA] (SFPOD) {$\SFPOD$};
      \node at ({-30}:\radius) (SFP) {$\SFP$};
      \node at ({-60}:\radius) (SFPOD) {$\SFPOD$};
      \node at ({-90}:\radius) (SIFPLA) {$\SIFPLA$};
      \node at ({-120}:\radius) (SIFPRA) {$\SIFPRA$};
      \node at ({-150}:\radius) (POR) {$\POR$};
      \node[left = 2cm of POR] (Lpw) {$\Lpw$};

      \draw[->] (SFP) edge node {}(POR);

      \draw[->] (POR) edge[bend right=15] node {}(SIFPRA);
      \draw[->] (SIFPRA) edge[bend right=15] node {}(SIFPLA);
      \draw[->] (SIFPLA) edge[bend right=15]node {} (SFPOD);
      \draw[->] (SFPOD) edge[bend right=15] node {}(SFP);

      \node[right = 2cm of SFP] (PPT) {$\PPT$};

      \draw[<->] (PPT) edge (SFP);
      \draw[<->] (Lpw) edge (POR);
\onslide<4->{      \draw[<->, very thick] (Lpw) edge (POR);
}    \end{tikzpicture}
    }%


    \onslide<4->{Proved following \cite{CookUrquhart} and \cite{Ferreira88}.}
  \end{resizebox}

\end{frame}

\begin{comment}
\begin{frame}
\frametitle{Our goals}

We were interested in:

\begin{itemize}
\item Discovering which class of formul\ae{} of $\Lpw$ corresponds to the PPT functions.
\item Developing a characterization of $\BPP$ throughout some $\Lpw$ formul\ae{}.
\item Identifying a proof system for such class.
\end{itemize}
\medskip
\pause
\begin{block}{Conjecture on $\POR$}
For capturing all the PPT functions it's enough to extend Ferreira's $P$-Functions with a functional argument $\omega: \Ss \longrightarrow \{0, 1\}$ and a function $Q(x, \omega)=\omega(x)$.
\end{block}\bigskip


\end{frame}

\begin{frame}
\frametitle{The correspondance between $\POR$ and $\Lpw$}
\framesubtitle{and between RBA and PTCA}

\begin{block}{RBA}
\vspace*{-1.5em}
\begin{align*}
&\begin{multlined}
\forall f \in \POR. \exists G_f \in \Sigma^b_1. \\
\big(\RS \vdash \forall x.\exists ! y . G_f(x, y)\big)\land \forall x, y. \llbracket G_f(x, y)\rrbracket_\xi =\{\omega \in \{\zzero, \oone\}^\Ss | f(x, \omega) = y\} \\
\end{multlined}\\[-1em]
&\begin{multlined}
\forall G \in \Sigma^b_1. \exists f_G \in PF. \\
\big(\RS \vdash \forall x.\exists ! y . G(x, y)\big) \to \forall x, y.\llbracket G(x, y)\rrbracket_\xi=\{\omega \in \{\zzero, \oone\}^\Ss | f_G(x, \omega) = y\}\\
\end{multlined}
\end{align*}
\end{block}
\pause
\begin{block}{PTCA}
\vspace*{-1.5em}
\begin{gather*}
\forall f \in PF. \exists G_f \in \Sigma^b_1. \big(S^1_2 \vdash \forall x.\exists ! y . G(x, y)\big)\land  \models \forall x. G_f(x, f(x)) \\
\forall G \in \Sigma^b_1. \exists f_G \in PF. \big(S^1_2 \vdash \forall x.\exists ! y . G(x, y)\big) \to \models \forall x. G_f(x, f(x))
\end{gather*}
\end{block}


\end{frame}

\end{comment}

\section{Some reductions}
%\subsection{Form $\POR$ to $\PPT$ (and back)}

\begin{frame}
\frametitle<1>{Expressivity of $\POR$, Part I}
%\frametitle<2>{Expressivity of $\POR$, Part I}
\frametitle<2>{Expressivity of $\POR$, Part II}
\frametitle<3>{Expressivity of $\POR$, Part III}
\frametitle<4>{Expressivity of $\POR$, part IV}
\frametitle<5>{Expressivity of $\POR$, part V}
%\frametitle<6->{Expressivity of $\POR$, part V}

\begin{overprint}
\resizebox{\textwidth}{!}{%
\begin{tikzpicture}[->, node distance=2cm, semithick, scale = 0.50]
 \node (POR) {$\POR$};
\onslide<1-2> {\node[right = 2.9cm of POR] (SFP) {$\SFP$};}
\onslide<1-2> {\node[right = 2.9cm of SFP] (PPT) {$\PPT$};}
\onslide<1> {\draw[->] (POR) edge[bend left, dashed] (SFP);}
\onslide<1> {\draw[->] (SFP) edge[bend left] (PPT);}
\onslide<1> {\draw[->] (PPT) edge[bend left] (SFP);}
\onslide<2> {\draw[->] (SFP) edge[bend left] (PPT);}
\onslide<2> {\draw[->] (PPT) edge[bend left] (SFP);}
\onslide<1> {\draw[->] (SFP) edge[bend left, very thick] (POR);}
%\onslide<1> {\draw[->] (SFP) edge[bend left, dashed] (POR);}
\onslide<2> {\draw[->] (SFP) edge[bend left] (POR);}
\onslide<2> {\draw[->] (POR) edge[bend left, very thick] (SFP);}
 \onslide<3-> {\node[right of= POR] (SIFPRA) {\footnotesize $\SIFPRA$};}
 \onslide<3-> {\node[right of = SIFPRA] (SIFPLA) {\footnotesize$\SIFPLA$};}
 \onslide<3-> {\node[right of = SIFPLA] (SFPOD) {\footnotesize $\SFPOD$};}
\onslide<3-> {\node[right of =  SFPOD] (SFP) {$\SFP$};}
\onslide<3> {\draw[->] (POR) edge[very thick] (SIFPRA);}
\onslide<3> {\draw[->] (SIFPRA) edge[dashed] (SIFPLA);}
\onslide<3> {\draw[->] (SIFPLA) edge[dashed] (SFPOD);}
\onslide<3> {\draw[->] (SFPOD) edge[dashed] (SFP);}

\onslide<4-> {\draw[->] (POR) edge (SIFPRA);}
\onslide<4> {\draw[->] (SIFPRA) edge[very thick] (SIFPLA);}
\onslide<4> {\draw[->] (SIFPLA) edge[dashed] (SFPOD);}
\onslide<4> {\draw[->] (SFPOD) edge[dashed] (SFP);}

\onslide<5-> {\draw[->] (SIFPRA) edge (SIFPLA);}
\onslide<5> {\draw[->] (SIFPLA) edge[very thick] (SFPOD);}
\onslide<5> {\draw[->] (SFPOD) edge[very thick] (SFP);}

%\onslide<6-> {\draw[->] (SIFPLA) edge (SFPOD);}
%\onslide<6> {\draw[->] (SFPOD) edge[very thick] (SFP);}

%\onslide<9-> {\draw[->] (SFPOD) edge (SFP);}

\end{tikzpicture}
}
\vspace*{-5mm}
\end{overprint}

\begin{overprint}
% \onslide<1>
%   % \begin{defn}[PTM, \cite{AroraBarak}]%
%   %   \label{def:PTMbarak}%
%   %   A probabilistic Turing machine (PTM) is a Turing machine with two
%   %   transition functions $\delta_0, \delta_1$. To execute a PTM $M$
%   %   on an input $x$, we choose in each step with probability $\frac 1 2$
%   %   to apply the transition function $\delta_0$ and with probability $\frac 1 2$
%   %   to apply $\delta_1$. This choice is made independently of all previous choices.
%   % \end{defn}%
%
%   \bigskip
%
%   \begin{tikzpicture}
%    \tikzset{tape/.style={minimum size=.7cm, draw}}
%    \begin{scope}[start chain=0 going right, node distance=0mm]
%     \foreach \x [count=\i] in {$\one$,$\zero$,$\one$,$\one$,$\zero$,$\one$,$\zero$,$\zero$} {
%      \ifnum\i=8 % if last node reset outer sep to 0pt
%        \node [on chain=0, tape, outer sep=0pt] (n\i) {\x};
%        \draw (n\i.north east) -- ++(.1,0) decorate [decoration={zigzag, segment length=.12cm, amplitude=.02cm}] {-- ($(n\i.south east)+(+.1,0)$)} -- (n\i.south east) -- cycle;
%       \else
%        \node [on chain=0, tape] (n\i) {\x};
%       \fi
%       \ifnum\i=1 % if first node draw a thick line at the left
% %       \draw [line width=.1cm] (n\i.north west) -- (n\i.south west);
%       \fi
%     }
%     \node [left=2cm of n1] (PTM) {PTM ($\PPT$)};
%     \node [right=.25cm of n8] {$\cdots$};
%     \node [tape, above left=.25cm and 1cm of n1] (d0) {$\delta_0$};
%     \node [tape, below left=.25cm and 1cm of n1] (d1) {$\delta_1$};
%     \draw [>=latex, ->] (d0) -| (n5);
%     \draw [>=latex, ->] (d1) -| (n5);
%    \end{scope}
%   \end{tikzpicture}
%
%   \medskip
%
%   \begin{tikzpicture}
%    \tikzset{tape/.style={minimum size=.7cm, draw}}
%    \begin{scope}[start chain=0 going right, node distance=0mm]
%     \foreach \x [count=\i] in {$\one$,$\zero$,$\one$,$\one$,$\zero$,$\one$,$\zero$,$\zero$} {
%      \ifnum\i=8 % if last node reset outer sep to 0pt
%        \node [on chain=0, tape, outer sep=0pt] (n\i) {\x};
%        \draw (n\i.north east) -- ++(.1,0) decorate [decoration={zigzag, segment length=.12cm, amplitude=.02cm}] {-- ($(n\i.south east)+(+.1,0)$)} -- (n\i.south east) -- cycle;
%       \else
%        \node [on chain=0, tape] (n\i) {\x};
%       \fi
%       \ifnum\i=1 % if first node draw a thick line at the left
% %       \draw [line width=.1cm] (n\i.north west) -- (n\i.south west);
%       \fi
%     }
%     \foreach \x [count=\i] in {$\zero$,$\one$,$\one$,$\one$,$\one$,$\zero$,$\zero$,$\zero$} {
%      \ifnum\i=8 % if last node reset outer sep to 0pt
%        \node [right = 0cm of m7, tape, outer sep=0pt] (m\i) {\x};
%        \draw (m\i.north east) -- ++(.1,0) decorate [decoration={zigzag, segment length=.12cm, amplitude=.02cm}] {-- ($(m\i.south east)+(+.1,0)$)} -- (m\i.south east) -- cycle;
%       \else
%        \node [below = 0.5cm of n\i, tape] (m\i) {\x};
%       \fi
%       \ifnum\i=1 % if first node draw a thick line at the left
% %       \draw [line width=.1cm] (n\i.north west) -- (n\i.south west);
%       \fi
%     }
%     \node [tape, above left= -0.125cm and 1cm of m1] (d) {$\delta$};
%     \node [left=0.30cm of d] (STM) {STM ($\SFP$)};
%     \node [right=.25cm of n8] {$\cdots$};
%     \node [right=.25cm of m8] {$\cdots$};
%     \draw [>=latex, ->] ($(m4)+(0,-0.5cm)$) -- ($(m6)+(0,-0.5cm)$);
% %    \node [tape, below left=.25cm and 1cm of n1] (d1) {$\delta_1$};
%     \draw [>=latex, ->] (d) -| (n5);
%     \draw [>=latex, ->] (d) -| (m1);
%    \end{scope}
%   \end{tikzpicture}

  % \begin{defn}[Stream Machines, informally]%
  % Stream machines are ordinary $(k+1)$-taped Turing Machines which use one of
  % their tapes to store an infinite stream of characters.
  % This tape is read from left to right, one character for transition.
  % An $\SFP$ machine is a PT Stream machine.
  % \end{defn}%
% \onslide<2>
% 	\begin{lemma}%
% 	Stream Machines $\simeq$ PTMs%
% 	\end{lemma}%
% 	\begin{proof}[Proof]%
%     Take:
%     \Large
%     $$
%       \delta := \delta_0 \sqcup \delta_1
%     $$
% 	% It suffices to show that having two transition functions $\delta_0$ and $\delta_1$ is equivalent to having an uniformly distributed read-only left to right binary tape, and a single transition function $\delta$ which behaves exactly like $\delta_b$ if the character on the tape is $b$.   %
% 	\end{proof}%
% 	\begin{corollary}%
% 	$\SFP\simeq\PPT$%
% 	\end{corollary}%
\onslide<1>
	The Stream Machines' source of randomness are tapes: functions $\eta \in \{\zzero, \oone\}^\Nat$, while $\POR$ functions use functions $\omega \in \{\zzero, \oone\}^\Ss$.\\

	The two classes correspond, but  \emph{modulo measure}:

  \begin{lemma}
    For every $M \in \SFP$, there is a $f_M$ in $\POR$ such that:
    %
	  $$\mu(\{\eta\in \{\zzero, \oone\}^\Nat\,|\, M(x, \eta)=y\})=\mu(\{\omega\in \{\zzero, \oone\}^\Ss\,|\, f_M(x, \omega)=y\})
    $$
	\end{lemma}
	\medskip
%	\begin{proof}[Proof]
%	If $p(x)$ is the bound to the complexity of a $M \in \SFP$, we can show that there exists a $\POR$ function $\mathit{eval}_{M}(x, \omega)$ which
%	\begin{itemize}
%	\item Works on an encoding of $M$ on tuples.
%	\item Computes $1^{p(x)}$ and by induction on it simulates the machine's transitions
%	\item Before each transition, queries $\omega$ for simulating the random tape $\eta$.
%	\end{itemize}
%	\end{proof}
	% Indeed, a $\SFP$ function can be simulated by a $\POR$ function which computes the transitive closure of the $\delta$ function, and at each step
	% queries a different coordinate of $\omega$ instead of $\eta$.

\onslide<2>

	\begin{itemize}
	\item The \emph{poly-time} complexity of a function $M \in \SFP$ bounds the amount of the random bits of $\eta$ influencing the output.
	\item A $\POR$ function $f(x, \omega): \Ss \times \{\zero, \one\}^\Ss \longrightarrow \Ss$ can query up to $2^{p(|x|)}$ different values to $\omega$.
	\end{itemize}

  \bigskip

  Fixed $x$ and $\omega$, each $f \in \POR$ queries at most a polynomial number of values to $\omega$.\\

  \bigskip

  \textbf{Idea:} reduce $\POR$ to an imperative formalism, and then to $\SFP$.

\onslide<3>
	%\begin{lemma}
	%$\forall f \in \POR.\exists M_f \in \SFP. \forall x, y.\linebreak
	%\mu(\{\omega\in \{\zzero, \oone\}^\Ss| f(x, \omega)=y\})= \mu(\{\eta\in \{\zzero, \oone\}^\Nat| M_f(x, \eta)=y\})$
	%\end{lemma}
	%\bigskip
	% \vspace*{-1cm}
	% \begin{align*}
	% \mathsf{Exp} &::= \epsilon \midd \mathsf{Exp.\zzero}
	% \midd \mathsf{Exp.\oone} \midd
	% \mathsf{Id} \midd \mathsf{Id} \sqsubseteq
	% \mathsf{Exp} \midd
	% \mathsf{Id} \wedge \mathsf{Exp} \midd
	% \neg \mathsf{Exp} \\
	% %
	% \mathsf{Stm} &::= \mathsf{Id} \leftarrow \mathsf{Exp}
	% \midd \mathsf{Stm};\mathsf{Stm} \midd \mathsf{Flip}(\mathsf{Exp}) \midd
	% \mathbf{while}(\mathsf{Exp})\{\mathsf{Stm}\}
	% \end{align*}
	% \begin{lemma}
	% $\forall f \in \POR.\exists P_f \in \SFIP. \forall x, y.
	% \forall x, \omega. f(x, \omega) = P_f(x, \omega)$
	% $\POR\subseteq \SFIP_{\mathbf{RA}}$.
  %	\end{lemma}
  \medskip

  \textbf{Goal: } Showing that for each $f \in \POR$ there is a \emph{poly-time} imperative program $P_f$ which can query $\omega$ and such that $\llbracket P_f\rrbracket = f$.

  \medskip

  % \begin{proof}[Proof]
	% By induction on $f$, it suffices to show a $\SIFPRA$ program for each basic function, then to show how to compose them and how to implement bounded recursion.
	% \end{proof}

\onslide<4>
%	\vspace*{-10mm}
%	\begin{lemma}
%	$\forall P \in \SFIP.\exists M_P \in \SFP. \forall x, y.\linebreak
%	\mu(\{\omega\in \{\zzero, \oone\}^\Ss| P(x, \omega)=y\})= \mu(\{\eta\in \{\zzero, \oone\}^\Nat| M_P(x, \eta)=y\})$
%	\end{lemma}

%	\centering
%	\resizebox{!}{0.55\textheight}{%
%	\begin{tikzpicture}[->, node distance=2cm, semithick, scale = 0.50]
%
%	 \node[minimum size=3cm, draw=black] (P) {$P$};
%	 \node[right = of P, minimum size=3cm, draw=black] (MP) {$M_P$};
%
%	 \node[minimum size=1cm, draw=black] (x) at ($(P)!0.5!(MP)+(0,8cm)$) {$x$};
%
%	 \node[minimum size=2cm, right = 10mm of x, draw=black] (PO) {$\mathcal P(\{\zzero, \oone\}^\Nat)$};
%	 \node[minimum size=2cm, left = 10mm of x, draw=black] (PBN) {$\mathcal P(\{\zzero, \oone\}^\Ss)$};
%
%	 \node[minimum size=1cm, draw=black] at ($(P)-(2cm,8cm)$) (y1) {$y_1$};
%	 \node[minimum width=15mm, minimum height=10mm, draw=black] at ($(y1)+(35mm,0)$) (y2) {$y_2$};
%	 \node[minimum width=25mm, minimum height=10mm, draw=black] at ($(y2)+(50mm,0)$) (y3) {$y_3$};
%	 \node[minimum width=15mm, minimum height=10mm, draw=black] at ($(y3)+(50mm,0)$) (y4) {$y_4$};
%
%	 \draw ($(x.south)-(2mm,0)$) -- ($(x.south)-(2mm,2cm)$) -- ($(P.north)+(5mm,2cm)$) -- ($(P.north)+(5mm, 0)$);
%	 \draw (PBN) -- (P);
%	 \draw ($(x.south)+(2mm,0)$) -- ($(x.south)-(-2mm,2cm)$) -- ($(MP.north)+(-5mm,2cm)$) -- ($(MP.north)-(5mm, 0)$);
%	 \draw (PO) -- (MP);
%	 \draw let \p1 = (P.south) in let  \p2 = (y1.north) in ($(\x2-2mm, \y1)$) -- ($(\x2-2mm, \y2)$);
%	 \draw let \p1 = (P.south) in let  \p2 = (y2.north) in ($(\x2-2mm, \y1)$) -- ($(\x2-2mm, \y2)$);
%	 \draw let \p1 = (P.south) in let  \p2 = (y3.north) in ($(\x1+20mm, \y1)$) -- ($(\x1+20mm, \y1-3cm)$) -- ($(\x2-2mm, \y1-3cm)$) -- ($(\x2-2mm, \y2)$);
%	 \draw let \p1 = (P.south) in let  \p2 = (y4.north) in ($(\x1+23mm, \y1)$) -- ($(\x1+23mm, \y1-26mm)$) -- ($(\x2-2mm, \y1-26mm)$) -- ($(\x2-2mm, \y2)$);
%	 \draw let \p1 = (MP.south) in let  \p2 = (y4.north) in ($(\x2+2mm, \y1)$) -- ($(\x2+2mm, \y2)$);
%	 \draw let \p1 = (MP.south) in let  \p2 = (y3.north) in ($(\x2+16.5mm, \y1)$) -- ($(\x2+16.5mm, \y2)$);
%	 \draw let \p1 = (MP.south) in let  \p2 = (y2.north) in ($(\x1-22mm, \y1)$) -- ($(\x1-22mm, \y1-13mm)$) -- ($(\x2+2mm, \y1-13mm)$) -- ($(\x2+2mm, \y2)$);
%	 \draw let \p1 = (MP.south) in let  \p2 = (y1.north) in ($(\x1-25mm, \y1)$) -- ($(\x1-25mm, \y1-1cm)$) -- ($(\x2+2mm, \y1-1cm)$) -- ($(\x2+2mm, \y2)$);
%	\draw let \p1 = (MP.south) in let  \p2 = (y1.north) in ($(\x1+19mm, \y1)$) -- ($(\x1+19mm, \y1-26mm)$) -- ($(\x2-2mm, \y1-26mm)$) -- ($(\x2-2mm, \y2)$);

	% \draw (MP.south) -- (y1.north);
	% \draw (MP.south) -- (y2.north);
	% \draw (MP.south) -- (y3.north);
	% \draw (MP.south) -- (y4.north);
%	\end{tikzpicture}

	% \begin{align*}
	% \mathsf{Exp} &::= \epsilon \midd \mathsf{Exp.\zzero}
	% \midd \mathsf{Exp.\oone} \midd
	% \mathsf{Id} \midd \mathsf{Id} \sqsubseteq
	% \mathsf{Exp} \midd
	% \mathsf{Id} \wedge \mathsf{Exp} \midd
	% \neg \mathsf{Exp} \\
	% %
	% \mathsf{Stm} &::= \mathsf{Id} \leftarrow \mathsf{Exp}
	% \midd \mathsf{Stm};\mathsf{Stm} \midd \mathsf{RB()} \midd
	% \mathbf{while}(\mathsf{Exp})\{\mathsf{Stm}\}
	% \end{align*}
	% \begin{lemma}
	% $\SFIP_\mathbf{RA} \approxeq \SFIP_\mathbf{LA}$
	% \end{lemma}
  \medskip

  \textbf{Goal: } Showing that for each \emph{poly-time} $P \in \SIFPRA$ there is a \emph{poly-time} imperative program $Q\in \SIFPLA$ such that:
  %
  $$
  \mu(\{\omega\in \{\zzero, \oone\}^\Ss\,|\, \llbracket P\rrbracket (x, \omega)=y\})= \mu(\{\eta\in \{\zzero, \oone\}^\Nat\,|\, \llbracket Q\rrbracket(x, \eta)=y\})
  $$
  %
  and $Q$ reads random bits sequentially from $\eta$.

  \medskip

	% \begin{proof}[Proof]
	% A register stores a table containing the queries and their value. The $\mathsf{Flip}(\mathsf{Exp})$ primitive is simulated by means of a procedure which accesses the table before and after each query.
	% \end{proof}
%\onslide<6>
%	\vspace*{-7mm}
%	\begin{lemma}
%	$\forall P \in \SFIP.\exists M_P \in \SFP. \forall x, y.\linebreak
%	\mu(\{\omega\in \{\zzero, \oone\}^\Ss| P(x, \omega)=y\})= \mu(\{\eta\in \{\zzero, \oone\}^\Nat| M_P(x, \eta)=y\})$
%	\end{lemma}

\onslide<5>
\medskip

\textbf{Goal: } Showing that for each \emph{poly-time} $P \in \SIFPLA$ there is a \emph{poly-time} STM $N\in \SFP$ such that:
  $$
  \mu(\{\eta\in \{\zzero, \oone\}^\Ss| \llbracket P\rrbracket (x, \eta)=y\})= \mu(\{\eta\in \{\zzero, \oone\}^\Nat| \llbracket N\rrbracket(x, \eta)=y\})
  $$

  \begin{theorem}
    $\POR \simeq \PPT$.
  \end{theorem}
\end{overprint}

\end{frame}

\section{Characterizing $\BPP$}


\begin{frame}
\frametitle{A semantic Characterization of $\BPP$}

We are investigating what happens if we extend $\Lpw$ with a measure quantifier $\cq{n/m}(F\,)$, defined by the following semantics:

\[
\llbracket\cq{n/m}(F\,)\rrbracket_\xi := \begin{cases}
\{\zzero, \oone\}^\Ss & \text{ if } \mu (\llbracket F\, \rrbracket_\xi)\ge n/m\\
\emptyset & \text{ otherwise }
\end{cases}
\]

\pause

\begin{corollary}\label{cor:bpp}
$\BPP$ is exactly the set of languages $L$ with characteristic function $f_L$
for which there is a $\Sigma^b_1$ formula $G_L$ such that:
\begin{gather*}
\RS \vdash \forall x \exists ! y. G_L(x, y)\\
\forall \sigma \in \Ss.\models \cq{2/3} \left( G_L(\sigma,f_L(\sigma)) \right)
\end{gather*}
\end{corollary}
\end{frame}



\section{Further work}

\begin{frame}
\frametitle{Towards a syntactic characterization of $\BPP$}

It is well known that $\BPP \subseteq \Sigma^p_2$, so according to \cite{Buss}, for each $L \in \BPP$ there exists a $\Sigma^b_3$ formula $H_L$ which decides $L$.

Putting together this result with the semantic characterization of $\BPP$, it must hold that:
\vspace{-2mm}
\begin{align*}
\forall x, y. \cq {2/3} (G_L(x, y)&\leftrightarrow H_L(x, y))
\end{align*}
\medskip
Since all the terms in $\Lpw$ are bounded, $\cq {n/m}$ can be reduced to a counting existential, and then to a bonded existential.
\begin{align*}
\forall x, y. \exists^{\ge k(x)}t_f\preceq t(x). (G_L(x, y)&\leftrightarrow H_L(x, y))[\Flip(t)\leftarrow \pi_{t}(t_f\,)=\oone]
\end{align*}

\begin{block}{Conjecture}
There is a non-trivial subset of $\BPP$, $\PBPP$ such that:
\[
\forall L\in \mathit{PBPP}.\exists F_L. PA \vdash \forall x. \forall y.F_L(x, y)
\]
\end{block}

\end{frame}


\begin{frame}
\frametitle{Contributions}

%\resizebox{\textwidth}{!}{%
%\begin{tikzpicture}[->, node distance=2cm, semithick, scale = 0.50]
% \node (S13) {$\RS$};
% \node[right = 2cm of S13] (POR) {$\POR$};
% \node[right = 2cm of POR] (PPt) {$\PPT$};
% \draw[-] (S13) edge (POR);
% \draw[-] (POR) edge (PPT);
% \end{tikzpicture}
% }

\medskip
\begin{itemize}
\item $\POR\simeq \PPT$.
\item As a corollary: Cobham-style Algebras $= \FP$.
\item Characterizations of $\BPP$, $\ZPP$, and other probabailistic complexity classes.
\end{itemize}

\end{frame}
\begin{comment}
\begin{frame}

\centering
\resizebox{!}{0.9\textheight}{%
\begin{tikzpicture}[->, node distance=2cm, semithick, scale = 0.50]

 \node[minimum size=3cm, draw=black] (P) {$P$};
 \node[right = of P, minimum size=3cm, draw=black] (MP) {$M_P$};

 \node[minimum size=1cm, draw=black] (x) at ($(P)!0.5!(MP)+(0,8cm)$) {$x$};

 \node[minimum size=2cm, right = 10mm of x, draw=black] (PO) {$\mathcal P(\{\zzero, \oone\}^\Ss)$};
 \node[minimum size=2cm, left = 10mm of x, draw=black] (PBN) {$\mathcal P(\{\zzero, \oone\}^\Nat)$};

 \node[minimum size=1cm, draw=black] at ($(P)-(2cm,8cm)$) (y1) {$y_1$};
 \node[minimum width=15mm, minimum height=10mm, draw=black] at ($(y1)+(35mm,0)$) (y2) {$y_2$};
 \node[minimum width=25mm, minimum height=10mm, draw=black] at ($(y2)+(50mm,0)$) (y3) {$y_3$};
 \node[minimum width=15mm, minimum height=10mm, draw=black] at ($(y3)+(50mm,0)$) (y4) {$y_4$};

 \draw ($(x.south)-(2mm,0)$) -- ($(x.south)-(2mm,2cm)$) -- ($(P.north)+(5mm,2cm)$) -- ($(P.north)+(5mm, 0)$);
 \draw (PBN) -- (P);
 \draw ($(x.south)+(2mm,0)$) -- ($(x.south)-(-2mm,2cm)$) -- ($(MP.north)+(-5mm,2cm)$) -- ($(MP.north)-(5mm, 0)$);
 \draw (PO) -- (MP);

 \draw (P.south) -- (y1.north);
 \draw (P.south) -- (y2.north);
 \draw (P.south) -- (y3.north);
 \draw (P.south) -- (y4.north);

 \draw (MP.south) -- (y1.north);
 \draw (MP.south) -- (y2.north);
 \draw (MP.south) -- (y3.north);
 \draw (MP.south) -- (y4.north);
\end{tikzpicture}
}


\end{frame}

\begin{frame}

\centering
\resizebox{!}{0.9\textheight}{%
\begin{tikzpicture}[->, node distance=2cm, semithick, scale = 0.50]

 \node[minimum size=3cm, draw=black] (P) {$P$};
 \node[right = of P, minimum size=3cm, draw=black] (MP) {$M_P$};

 \node[minimum size=1cm, draw=black] (x) at ($(P)!0.5!(MP)+(0,8cm)$) {$x$};

 \node[minimum size=2cm, right = 10mm of x, draw=black] (PO) {$\mathcal P(\{\zzero, \oone\}^\Ss)$};
 \node[minimum size=2cm, left = 10mm of x, draw=black] (PBN) {$\mathcal P(\{\zzero, \oone\}^\Nat)$};

 \node[minimum size=1cm, draw=black] at ($(P)-(2cm,8cm)$) (y1) {$y_1$};
 \node[minimum width=15mm, minimum height=10mm, draw=black] at ($(y1)+(35mm,0)$) (y2) {$y_2$};
 \node[minimum width=25mm, minimum height=10mm, draw=black] at ($(y2)+(50mm,0)$) (y3) {$y_3$};
 \node[minimum width=15mm, minimum height=10mm, draw=black] at ($(y3)+(50mm,0)$) (y4) {$y_4$};

 \draw ($(x.south)-(2mm,0)$) -- ($(x.south)-(2mm,2cm)$) -- ($(P.north)+(5mm,2cm)$) -- ($(P.north)+(5mm, 0)$);
 \draw (PBN) -- (P);
 \draw ($(x.south)+(2mm,0)$) -- ($(x.south)-(-2mm,2cm)$) -- ($(MP.north)+(-5mm,2cm)$) -- ($(MP.north)-(5mm, 0)$);
 \draw (PO) -- (MP);



 \draw let \p1 = (P.south) in let  \p2 = (y1.north) in ($(\x2-2mm, \y1)$) -- ($(\x2-2mm, \y2)$);
 \draw let \p1 = (P.south) in let  \p2 = (y2.north) in ($(\x2+2mm, \y1)$) -- ($(\x2+2mm, \y2)$);
 \draw let \p1 = (P.south) in let  \p2 = (y3.north) in ($(\x1+20mm, \y1)$) -- ($(\x1+20mm, \y1-3cm)$) -- ($(\x2-2mm, \y1-3cm)$) -- ($(\x2-2mm, \y2)$);
 \draw let \p1 = (P.south) in let  \p2 = (y4.north) in ($(\x1+23mm, \y1)$) -- ($(\x1+23mm, \y1-26mm)$) -- ($(\x2-2mm, \y1-26mm)$) -- ($(\x2-2mm, \y2)$);

 \draw let \p1 = (MP.south) in let  \p2 = (y4.north) in ($(\x2+2mm, \y1)$) -- ($(\x2+2mm, \y2)$);
 \draw let \p1 = (MP.south) in let  \p2 = (y3.north) in ($(\x2+16.5mm, \y1)$) -- ($(\x2+16.5mm, \y2)$);
 \draw let \p1 = (MP.south) in let  \p2 = (y2.north) in ($(\x1-22mm, \y1)$) -- ($(\x1-22mm, \y1-13mm)$) -- ($(\x2-2mm, \y1-13mm)$) -- ($(\x2-2mm, \y2)$);
 \draw let \p1 = (MP.south) in let  \p2 = (y1.north) in ($(\x1-25mm, \y1)$) -- ($(\x1-25mm, \y1-1cm)$) -- ($(\x2+2mm, \y1-1cm)$) -- ($(\x2+2mm, \y2)$);
 %\draw let \p1 = (MP.south) in let  \p2 = (y1.north) in ($(\x1+19mm, \y1)$) -- ($(\x1+19mm, \y1-26mm)$) -- ($(\x2-2mm, \y1-26mm)$) -- ($(\x2-2mm, \y2)$);

% \draw (MP.south) -- (y1.north);
% \draw (MP.south) -- (y2.north);
% \draw (MP.south) -- (y3.north);
% \draw (MP.south) -- (y4.north);
\end{tikzpicture}
}
\end{frame}
\end{comment}


\begin{frame}{}
  \centering \huge
  \vspace{1cm}
  {Thanks for your attention}
\end{frame}

\begin{frame}[t]
\frametitle{References}
\printbibliography
\end{frame}


\end{document}
