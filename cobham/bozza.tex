%!TeX spellcheck = en-US
In the literature, it is widely agreed  that Cobham's function algebra \emph{et simila}
--- e.g. Ferreira's PTCF --- are sound and complete with respect to the polynomial
time computable functions $\FP$. Unfortunately,
as far as we know, there is no proof of such result which is clear,
detailed, self-contained and easily available.
%
For this reason, we decided to give, as a corollary of our results,
a proof of that fact.
%
Even in this case, we show the two inclusions separately. We will first address
the inclusion of $\FP$ in a Cobham style function algebra in Section \ref{sec:polyFcompleteness},
then, in Section \ref{sec:polyFsoundness}, we show that even the opposite direction holds.
%
To do so, we employ the definitions of Turing Machines given in Chapter \ref{chap:preliminaries}
to show that they can be reduced FSTMs,
then we give a definition of a Cobham-Style function algebra $\fcob$ and prove it
is equivalent to $\POR^-$, then leveraging
some results we gave in Chapter \ref{chap:sfptopor}, we conclude that the
the class of $\FP$ functions is equivalent to $\fcob$.



\section{The class $\fcob$}

Following~\cite{Ferreira88} and~\cite{Cobham1965}, we introduce the class $\fcob$ which will be proved to precisely characterize string-functions computable in polynomial time.

\begin{defn}[The Class $\fcob$]
The class $\fcob: \Ss^n \longrightarrow \Ss$ is
the smallest class containing initial functions:
\begin{itemize}
\itemsep0em

\item $\cob{E}(x)=\emptyset$;

\item $\cob{P}^i(x_1,\dots, x_n)=x_i$, for $1\leq i\leq n$;

\item $\cob{C}^\zero(x) = x\zero$ and
$\cob{C}^\one(x) = x\one$;

\item $\cob{Q}(x,y)=\one$ if and only if $x\subseteq y$,
$\cob{Q}(x,y)=\zero$ otherwise;

\end{itemize}
and closed under composition and
bounded recursion,
where a function $f$ is defined by bounded
recursion from $g, h_0, h_1$ as:
\begin{align*}
f(x_1,\dots, x_n,\eepsilon) &= g(x_1,\dots, x_n); \\
%
f(x_1,\dots, x_n, y\zero) &= h_0(x_1,\dots, x_n, y,
f(x_1,\dots, x_n, y)) |_{t(x_1,\dots, x_n,y)}; \\
%
f(x_1,\dots, x_n,y\one) &= h_1(x_1,\dots, x_n,
y, f(x_1,\dots, x_n,y))|_{t(x_1,\dots, x_n,y)};
\end{align*}
with $t$ term of $\Lpw$.
\end{defn}

The class $\fcob$ is equi-expressive with respect to the class $\POR^-$: indeed,
even if the former class takes in input an additional functional argument,
its functions cannot access it by means of the $\POR$'s querying function $Q$.

Indeed, the $\POR^-$ has been defined only to decouple the combinatorial
concerns of the reduction from $\SFP$ to $\POR$ from the measure-theoretic ones.
The $\POR^-$ class by itself, has not got any concrete reason to be defined in that peculiar way.
Thus, the definition of the $\fcob$ class is meant to remove from the definition of $\POR^-$
the superfluous right-most argument which does not affect at all the
result of the computation. As a consequence, it is easy to show
equivalence of $\POR^-$ with the
class $\fcob$ we have defined above.

First, we show that each function in $\fcob$ can be represented by a function
in $\POR^-$ which behaves exactly like it,
independently form the value of its oracle $\omega$.

\begin{remark}[$\fcob$ Soundness]
  \label{rem:cobtopor}
  $\forall f \in \fcob.\exists g_f \in\POR^-.
  \forall \omega, \vec x. f(\vec x)=g_f(\vec x,\omega)$.
\end{remark}
\begin{proof}
  By induction on the proof that a function is in Ferreira's $\fcob$.

  \begin{itemize}
    \item If the function is $\cob E$ , $E$ respects the property.
    \item If the function is $\cob {P^n_j}$ , the correspondent $\POR$
    projector respects the claim.
    \item If the function is $\cob {C^\bool}$ , $S_\bool$ verifies the claim.
    \item $\cob Q$ can be encoded by means of bounded recursion on notation, as :
    \begin{align*}
      \mathit{pref}(x, \epsilon, \omega) &:= \mathit{eq}(x, \zero, \omega); \\
      \mathit{pref}(x, y\bool, \omega) &:= \mathtt{if}(\one, \mathit{pref}(x, y, \omega), \mathit{eq}(x, y\bool, \omega))|_{\one}
    \end{align*}
    where the functions $\mathtt{if}, \mathit{eq}$ are in $\POR^-$. The definitions
    can found in Section \ref{sec:dynenc}.
    The proof of the correctness can be done by induction of $y$.
    \begin{itemize}
      \item[$\eepsilon$] In this case $\mathit{pref}$ returns $\one$ if and only if $x=y=\epsilon$.
      \item[$\tau\bool$] Suppose that $x$ is a prefix of $\tau \bool$, then they are equal or
      $x$ is a strong prefix of $\tau\bool$. In the first case, the functions returns $\one$
      as a consequence of the correctness of $\mathtt{if}$ and $\mathit{eq}$, proven
      in Section \ref{sec:dynenc}. Otherwise $x$ is a prefix of $y$, thus the result is a consequence oh the IH.
      Suppose that $x$ is not a prefix of $y$, they cannot be equal, so the result comes from the IH.
    \end{itemize}
    \item The composition and recursion on notation cases
    follow by IH.
  \end{itemize}
\end{proof}

Finally we state that each function $f \in \POR^-$ can be represented
by means of a function $\fcob$ which, without taking any functional argument,
behaves exactly like $f$ does.

\begin{remark}[$\fcob$ Completeness]
  \label{rem:portocob}
  $\forall g \in\POR^-.\exists f_g\in \fcob.
  \forall \omega, \vec x. g(\vec x, \omega)=f_g(\vec x)$.
\end{remark}
\begin{proof}
  By induction on the proof that $g\in \POR$

  \begin{itemize}
    \item If the function is $E$, $\cob E$ is a witness for the existential.
    \item If the function is ${P^n_j}$ , the correspondent $\fcob$
    projector has the claimed property.
    \item If the function is $S_\bool$, $\cob {C^\bool}$ verifies the claim.
    \item $\POR^-$'s $C$ can be encoded by means of bounded recursion on notation, taking:
    \begin{align*}
    C(\eepsilon, z_\epsilon, z_0, z_1, \omega) &:= z_\epsilon;\\
    C(x\bool, z_\epsilon, z_0, z_1, \omega) &:= z_b|_{z_0z_1}.
  \end{align*}
    \item The composition and recursion on notation cases
    follow by IH.
  \end{itemize}
\end{proof}


\section{From $\FP$ to a Cobham style Function Algebra}
\label{sec:polyFcompleteness}

In this section, we show that each $f \in \FP$ function is in $\fcob$ too.
To do so, we start by reducing TMs to FSTM.
The reduction is purely combinatorial:
intuitively, we extend the transition function
of the starting TM to match all the possible characters
of the secondary tape. Thus we get a FSTM which behaves
exactly like
the starting TM. Then Lemma \ref{lemma:polyFtoCob} shows that
there is a function $g$ in $\POR^-$ such that
$\forall x, y \in \Ss. \forall \omega \in \Os. f(x, y)= g(x, y, \omega)$.
Finally by Remark \ref{rem:portocob}, we have that there is a function $g'$ in $\fcob$
such that $\forall \omega. g(x, y, \omega)= g(x, y)$. This entails that $f=g$ and,
thus, that $\FP \subseteq \fcob$.

\begin{lemma}
  \label{lemma:polyFcompleteness}
  Each ordinary poly-time computable function $f$ there is
  a function in $\polyF$ $g$ such that:
  $$
  \forall x_1, x_2. f(x_1) = g(x_1, x_2).
  $$
\end{lemma}
\begin{proof}
  Given the (canonical) Turing Machines which computes $f$,
  $M_f=\langle  \m{\Qs}, \m{q_0}, \m{\Sigma}, \m{\delta}\rangle$, we can define
  the Finite Stream Turing Machine which computes $g$ as
  $N=\langle \m{\Qs}, \m{q_0}, \m{\Sigma}, \m{\Delta(\delta)}\rangle$ where
  the functional $\Delta$ extends all the transitions of the original machine
  in order to match each of the characters on the secondary tape. Formally:
%
  $$
  \Delta(\delta) := \bigcup \{\{\langle p, c_r, \zero, q, c_w, d \rangle,
                                      \langle p, c_r, \one, q, c_w, d \rangle,
                                      \langle p, c_r, \circledast, q, c_w, d \rangle| \langle p, c_r, q, c_w, d \rangle \in \delta \}.
  $$
  \noindent
  Proceeding by induction on the steps of the ordinary Turing Machine that
  the Finite Stream Turing Machine $N$ (restricted to its work tape), behaves
  exactly as the Turing Machine $M$, it si possible to show that $N$ and $M$
  produce the same output configuration with respect to their work tapes.
\end{proof}

\begin{cor}
  $\FP \subseteq \fcob$.
\end{cor}
\begin{proof}
  Suppose that $f \in\FP$. We must show that $f \in \fcob$, too.
  Lemma \ref{lemma:polyFcompleteness} shows that there is a $f' \in \polyF$ such that
  $$
  \forall x_1, x_2 \in \Ss. f(x_1)=f'(x_1, x_2).
  $$
  % Thus, we can define $f''$ as follows:
  % $$
  % f''(x)=f'(x, \epsilon)
  % $$
  % obtaining
  % $$
  % \forall x\in \Ss. f(x_1)=f''(x_1)
  % $$
  From Lemma \ref{lemma:polyFtoCob}, we know that there is a function
  $g \in \POR^-$ such that:
  $$
  \forall x_1, x_2 \in \Ss. \forall \omega \in \Os. f'(x_1, x_2)=g(x_1, x_2, \omega).
  $$
  Consequently, from Remark \ref{rem:portocob}, we observe that $\exists g' \in \fcob$
  such that:
  $$
  \forall x_1, x_2\in \Ss \forall \omega \in \Os. g(x_1,  x_2, \omega)=g'(x_1, x_2).
  $$
  Joining together these results, and choosing some values for $\omega$, e.g. $\omega = \lambda x.\zero$, we obtain that:
  $$
  \forall x_1, x_2 \in \Ss. f(x)=g'(x_1, x_2).
  $$
  Take in exam the following function:
  $$
  g''(x):=g'(x, \eepsilon),
  $$
  it is trivial that $g''\in \fcob$ because such class is closed under composition. This concludes the proof, indeed we have that $f=g''$ and $g \in \fcob$.
\end{proof}


\section{From a Cobham style Function Algebra to $\FP$}
\label{sec:polyFsoundness}

In this section, we prove that the Cobham style function algebra
$\fcob$ is sound with respect to ordinary Turing Machines.
The is a mere consequence of some results
we already established: Remark \ref{rem:cobtopor} proofs
the soundness of this class with respect to $\POR^-$, then
Theorem \ref{thmTaskC} proves a quantitative soundness of $\POR^-$ with respect to $\SFP$, thus the main concern of the proof is showing
that in the functions in $\SFP$ we obtain there in this specific case
are in $\FP$, too. This is done showing a reduction from STMs to TMs.
%
Before diving the claim and the detailed proof, we would like to point out that
we are not meaning to show that, \emph{in general}, a STM can be reduced to
an equivalent TM. But that this can be done under the strong assumption
that the value on the oracle tape does not affect the result of the computation.
%
This particular result is part of the proof of Lemma \ref{lemma:polyFsoundness},
in which we show that $\polyF$ is sound with respect to $\FP$.

\begin{lemma}
  \label{lemma:polyFsoundness}
  Each function $f \in \fcob$ is in $\FP$, too.
\end{lemma}
\begin{proof}
  Suppose that $f \in \fcob$. According to Remark \ref{rem:cobtopor}, there is a function $g \in \POR^-$ such that:
  $$
  \forall x\in \Ss.\forall \omega \in \Os. f(x)=g(x, \omega)
  $$
  so, said $f(x)=y$, we have that:
  $$
  \mu(\{\omega \in \Os| g(x, \omega)= y\})=1.
  $$
  Furthermore, from Theorem \ref{thmTaskC},
  we know that there is a $h \in \SFP$ such that:
  $$
  \mu(\{\eta \in \{\zero, \one\}^\Nat| h(x, \eta)= y\})=1,
  $$
  which entails that:
  $$
  \forall \eta \in \{\zero, \one\}^\Nat. h(x, \eta)= y
  $$
  due to the polynomial time bound of $\SFP$.
  Moreover, as a consequence of Lemmas
    \ref{lemma:portosifp}, \ref{lemma:sifpratosifpla}, Proposition \ref{prop:SFPODimplSIFPLA}
    and Lemma \ref{lemma:SFPODtoSFP}, we can suppose
  without lack of generality, that
  said $M_h=\langle \Qs, \Sigma, \delta, q_0\rangle$ the STM computing $h$,
  $\delta$ is such that:
  \begin{equation}
    \forall c \in\Sigma.\forall  q \in \Qs. \delta(q, c, \zero)=\delta(c, q, \one).\tag{$*$}
  \end{equation}
  This because the program $P_g \in \SIFPRA$ whose semantics is equal to $g$
  does not employ $\fl e$ expressions. Moreover, even the $\SIFPLA$ implementation
  of $P_g$ does not use to $\rb$ expression, thus it is identical to $P_g$.
  Consequently, its $\SFPOD$ correspondent function can be defined upon a machine
  which does not consume the oracle tape, thus, $M_h$ is a
  machine whose transition function $\delta$ has the property described above.
  %
  Finally, we can define a standard TM which behaves \emph{exactly} as $M_h$ does:
  $M:= \langle \Qs, \Sigma, \Delta(\delta), q_0\rangle$, with
  $$
    \Delta(\delta) :=\{ \langle p, c_r, q, c_w, d\rangle | \exists \bool \in \Bool. \langle p, c_r, \bool, q, c_w, d\rangle \in \delta \}.
  $$
\noindent
  Since $\delta$ satisfies Equation $(*)$, $\Delta(\delta)$ is a function, so
  $M$ is a TM.
  %
 We show that for every $k \in \Nat$, said $\overline \eta= \lambda x.\zero$,
 it holds that:
  $$
  \langle \sigma, q, \tau, \overline \eta\rangle \stmreach_\delta^k \langle \sigma', q', \tau', \overline \eta\rangle
  \to
  \langle \sigma, q, \tau\rangle \tmreach_{\Delta(\delta)}^k \langle \sigma', q', \tau'\rangle.
  $$
  The proof is by induction on $k$:
  \begin{itemize}
    \item [$0$] The claim is trivial.
    \item [$n+1$] Suppose that:
    $$
    \langle \sigma, q, \tau, \overline \eta\rangle \stmreach_\delta^n \langle \sigma'', q'', \tau'', \overline \eta\rangle \stmtrans_\delta \langle \sigma', q, \tau', \overline \eta\rangle.
    $$
    The claim comes from the IH
    observing that $\delta(q, c, \zero) =\Delta(\delta)(q, c)$ and thus that:
    $$
    \langle \sigma'', q'', \tau''\rangle \tmtrans_{\Delta(\delta)} \langle \sigma', q, \tau'\rangle
    $$
    as a consequence of the definition of $\stmtrans_\delta$ and $\tmtrans_\delta$
    (Definitions \ref{df:STMTransition} and \ref{df:TMTransition}).
  \end{itemize}

  \noindent
  As a consequence of this result, we know that for every $x \in \Ss$
  the function computed by $M$
  and called $h'$ is such that:
  $$
  \forall x \in \Ss. h'(x) = h (x, \overline \eta).
  $$
  Moreover, if the first machine
  is poly-time, then even the second is so.
  %
  \noindent
  To summarize, supposing $f \in \fcob$, we know that there is a function $g \in \POR^-$ such that:
  $$
  \forall x\in \Ss.\forall \omega \in \Os. f(x)=g(x, \omega).
  $$
  then there is also a function $h$ in $\SFP$ such that:
  $$
  \forall x\in \Ss.\forall \eta \in \Bool^\Nat. f(x)=h(x, \eta),
  $$
  which entails that:
  $$
  \forall x\in \Ss. f(x)=h(x,\overline \eta).
  $$
  So, since $\forall x \in \Ss. h'(x) = h (x, \overline \eta)$, we conclude that:
  $$
  \forall x\in \Ss. f(x)=h'(x).
  $$
  Finally, since $h' \in \FP$, we obtain $f \in \FP$.


\end{proof}

\section{Concluding the proof}

In Section \ref{sec:polyFcompleteness}, we have shown that the $\fcob$ function
algebra is complete with respect to the class $\FP$. Subsequently, in Section
\ref{sec:polyFsoundness}, we have shown that  $\fcob$ is even sound with respect to
$\FP$. These two results, if taken together, show that the class of functions computed by the
Cobham style function algebra $\fcob$ coincides with the class of functions computed
by polynomial Turing machines, namely $\FP$.
%
This result have been for long part of the folklore surrounding Cobham's function
algebra, but as far as we know, among the literature there was not any detailed,
self-contained and exhaustive proof of that fact.

\begin{theorem}
  $\FP=\polyF$.
\end{theorem}
\begin{proof}
  Consequence of Lemmas
  \ref{lemma:polyFcompleteness}
  and
  \ref{lemma:polyFsoundness}.

\end{proof}
