\documentclass{article}
\usepackage[english, italian]{babel}

\usepackage{geometry,xcolor,amsthm,amsmath,amssymb,stmaryrd, mathtools, hyperref, verbatim, bussproofs, bm, bbm, xifthen, tikz, mathrsfs}
\usepackage{thmtools, thm-restate}
\usepackage[backend=biber]{biblatex}

\newcommand{\query}{\emph{Q}}
\newcommand{\Sf}{\emph{S}}
\newcommand{\Cf}{\emph{C}}

\newcommand{\POR}{\mathcal{POR}}
\newcommand{\FOL}{\mathsf{FOL}}
\newcommand{\MQPA}{\mathsf{MQPA}}
\newcommand{\SIFP}{\mathsf{SIFP}}
\newcommand{\RA}{{\mathsf{RA}}}
\newcommand{\LA}{{\mathsf{LA}}}
\newcommand{\SIFPRA}{\mathsf{SIFP_{\RA}}}
\newcommand{\SFPOD}{\mathbf{SFP_{\mathbf{OD}}}}
\newcommand{\SIFPLA}{\mathsf{SIFP_{\LA}}}
\newcommand{\lang}[1]{{\mathcal L ({#1})}}


\newcommand{\Bool}{\mathbb{B}}
\newcommand{\Ss}{\mathbb{S}}
\newcommand{\Os}{\mathbb{O}}
\newcommand{\Nat}{\mathbb{N}}

%\newcommand{\zzero}{\mathbf{0}}
%\newcommand{\oone}{\mathbf{1}}
\newcommand{\zzero}{\zero}
\newcommand{\oone}{\one}
\newcommand{\eepsilon}{\bm{\epsilon}}
\newcommand{\cconc}{\bm{\conc}}
%\newcommand{\ttimes}{\bm{\times}}

\newcommand{\ssubseteq}{\bm{\subseteq}}

\newcommand{\zero}{\mathtt{0}}
\newcommand{\one}{\mathtt{1}}
\newcommand{\bool}{\mathtt{b}}
\newcommand{\bbool}{\mathtt{b}}


\newcommand{\tTerm}{{\mathbf t}}
\newcommand{\fTerm}{{\mathbf f}}
\newcommand{\vTerm}{{\mathbf v}}
\newcommand{\bbT}{{\mathbf b}}
\newcommand{\uT}{{\mathsf u}}
\newcommand{\vT}{{\mathsf v}}
\newcommand{\arrowT}{\Rightarrow}
\newcommand{\RS}{{\mathit{RS}^1_2}}


\newcommand{\Lpw}{\mathcal{L}}
\newcommand{\Lmq}{\mathcal{L}^{\mathsf{MQ}}}
\newcommand{\Lbuss}{\mathcal{L}_{\mathsf{Buss}}}

\newcommand{\Flip}{\mathtt{Flip}}

\newcommand{\midd}{\; \; \mbox{\Large{$\mid$}}\;\;}
\newcommand{\conc}{\frown}

\newcommand{\suc}{\mathtt{S}}
\newcommand{\PA}{\mathsf{PA}}

\newcommand{\DIA}{\mathbf{D}}
\newcommand{\BOX}{\mathbf{C}}

\newcommand{\longv}[1]{{}}

\newcommand{\rt}[1]{\(#1\)-Reduction-Tree}
\newcommand{\RT}[3]{\mathit{RT}_{#1}(#2, #3)}



\newcommand{\id}{\mathsf{Id}}
\newcommand{\stm}{\mathsf{Stm}}
\newcommand{\xp}{\mathsf{Exp}}
\newcommand{\fl}[1]{\mathsf{Flip(}#1\mathsf{)}}
\newcommand{\rb}{\mathsf{RandBit()}}
\newcommand{\while}[2]{\mathbf{while}(#1)\{#2\}}
\newcommand{\If}[2]{\mathbf{if}(#1)\{#2\}}
\newcommand{\sk}{\mathbf{skip};}
\newcommand{\halt}{\mathbf{halt};}
\newcommand{\takes}{\leftarrow}
\newcommand{\store}{\Sigma}
\newcommand{\as}[2]{[#1 \leftarrow #2]}
\newcommand{\ssos}{\triangleright}
\newcommand{\sred}{\rightharpoonup}
\newcommand{\ccrt}{\mathit{ccrt}}

\newcommand{\leadstola}{{\leadsto_\LA}}
\newcommand{\leadstora}{{\leadsto_\RA}}
\newcommand{\leadstolan}[1]{{\leadsto_\LA^{#1}}}
\newcommand{\leadstoran}[1]{{\leadsto_\RA^{#1}}}

%\newcommand{\LL}{\mathfrak L}
\newcommand{\LL}[1]{{\mathfrak L_{#1}}}

\newcommand{\MM}{\mathfrak M}
\newcommand{\copyb}{\mathit{copyb}}
\newcommand{\trunc}{\mathit{trunc}}

%\newtheorem{theorem}{Theorem}
\newtheorem{prop}{Proposition}
\newtheorem{remark}{Remark}
%\newtheorem{lemma}{Lemma}
\newtheorem{cor}{Corollary}
%\newtheorem{example}{Example}


\theoremstyle{definition}
\newtheorem{defn}{Definition}
\newtheorem{notation}{Notation}
\newtheorem{ex}{Example}













%%%%%%%%%%%%
%%%% CONJECTURE


\newcommand{\df}{\mathsf{d}}

\renewcommand{\C}{\mathbf{C}}
\newcommand{\cq}[1]{\mathbf{C}^{#1}}
\newcommand{\dq}[1]{\mathbf{D}^{#1}}

\newcommand{\BPP}{\mathbf{BPP}}
\newcommand{\PBPP}{\mathbf{PBPP}}
\newcommand{\RP}{\mathbf{RP}}
\newcommand{\ZPP}{\mathbf{ZPP}}
\newcommand{\PP}{\mathbf{PP}}

%\newcommand{\PTM}{\mathbf{PTM}}
\newcommand{\PPT}{\mathbf{PPT}}
\newcommand{\FBPP}{\mathbf{FBPP}}
\newcommand{\SIMP}{\mathbf{SIMP}}
\newcommand{\SFIP}{\mathbf{SFIP}}
\newcommand{\BRS}{\mathbf{BRS}}

\newcommand{\SFP}{\mathbf{SFP}}
\newcommand{\FP}{\mathbf{FP}}
\newcommand{\Lstm}{L(\mathsf{Stm})}


\theoremstyle{theorem}
\newtheorem{conj}{Conjecture}
\newtheorem{characterization}{Characterization}







%% TASK B
\newcommand{\PORo}{\POR^\lambda}

\newcommand{\sT}{\mathsf{s}}

\newcommand{\zT}{\mathsf{0}}
\newcommand{\oT}{\mathsf{1}}
\newcommand{\eT}{\mathsf{\epsilon}}
\newcommand{\bT}{\mathsf{b}}
%% ??
\newcommand{\cdotT}{\cdot}
\newcommand{\Tail}{\mathsf{Tail}}
\newcommand{\Trunc}{\mathsf{Trunc}}
\newcommand{\Cond}{\mathsf{Cond}}
\newcommand{\flipcoin}{\mathsf{Flipcoin}}
\newcommand{\Rec}{\mathsf{Rec}}

\newcommand{\TT}{\mathtt{T}}
\newcommand{\FF}{\mathtt{F}}
\newcommand{\GG}{\mathtt{G}}
\newcommand{\HH}{\mathtt{H}}
\newcommand{\KK}{\mathtt{K}}
\newcommand{\CC}{\mathtt{C}}

\newcommand{\PV}{\mathsf{PV}^\omega}


\newcommand{\BT}{\mathsf{B}}
\newcommand{\BNeg}{\mathsf{BNeg}}
\newcommand{\BOr}{\mathsf{BOr}}
\newcommand{\BAnd}{\mathsf{BAnd}}
\newcommand{\Eps}{\mathsf{Eps}}
\newcommand{\boolT}{\mathsf{Bool}}
\newcommand{\Zero}{\mathsf{Zero}}
\newcommand{\Eq}{\mathsf{Eq}}
\newcommand{\Times}{\mathsf{Times}}
\newcommand{\Sub}{\mathsf{Sub}}
\newcommand{\Conc}{\mathsf{Conc}}

\newcommand{\subseteqT}{\subseteq}


\newcommand{\ooverline}[1]{\overline{\overline{#1}}}
\newcommand{\func}[1]{f_{\mathrm{#1}}}

\newcommand{\NP}{\mathbf{NP}}

\newcommand{\varO}{\mathbf{x}}
\newcommand{\varT}{\mathbf{y}}
\newcommand{\termO}{\mathbf{t}}
\newcommand{\termT}{\mathbf{u}}
\newcommand{\termF}{\mathbf{w}}


\newcommand{\realize}{\ \circledR\ }

\newcommand{\EM}{EM}
\newcommand{\MP}{MP}

\newcommand{\enc}{\sharp}
\newcommand{\app}{\mathsf{app}}


\newcommand{\topO}{\mathbbm{1}}

\newcommand{\PIND}{PIND}
\newcommand{\uTerm}{\mathtt{U}}


%%% TASK C

\newcommand{\fcob}{\mathcal{F}_{\mathsf{Cob}}}
\newcommand{\cob}[1]{{#1}_{\mathcal{D}}}


\newcommand{\MS}{M_S}
\newcommand{\Qs}{\mathcal{Q}}
\newcommand{\sstar}{\circledast}
\newcommand{\reaches}[2]{\triangleright^{#1}_{#2}}
\newcommand{\Sigmab}{\hat{\Sigma}}

\newcommand{\POLY}{\mathsf{POLY}}
\newcommand{\MSFP}{M_\SFP}
\newcommand{\Dd}{\mathbb{D}}

\newcommand{\Oone}{\overlineN{\oone}}
\newcommand{\Zzero}{\overlineN{\zzero}}

\newcommand{\overlineN}[1]{\widetilde{#1}}

\newcommand{\Star}{*}

\newcommand{\Tuples}{\mathbb{T}}

\newcommand{\ovverline}[2]{{\underline{#1}}_{#2}}
\newcommand{\rmr}{\mathit{rmr}}
\newcommand{\rml}{\mathit{rml}}
\newcommand{\addr}{\mathit{addr}}
\newcommand{\addl}{\mathit{addl}}
\newcommand{\enct}{\mathit{enct}}
\newcommand{\tenc}{\mathit{tenc}}
\newcommand{\matcht}{\mathit{matcht}}
\newcommand{\apply}{\mathit{apply}}
\newcommand{\step}{\mathit{step}}
\newcommand{\eval}{\mathit{eval}}

\newcommand{\Const}{\Sf}
\newcommand{\concat}{\mathit{concat}}
\newcommand{\rv}{\mathit{rv}}
\newcommand{\reverse}{\mathit{reverse}}
\newcommand{\ras}{\mathit{ras}}
\newcommand{\las}{\mathit{las}}

\newcommand{\rel}{\mathit{rel}}
\newcommand{\lel}{\mathit{lel}}
\newcommand{\ral}{\mathit{ral}}
\newcommand{\lal}{\mathit{lal}}
\newcommand{\rrl}{\mathit{rrl}}
\newcommand{\lrl}{\mathit{lrl}}
\newcommand{\srrl}{\mathit{srrl}}
\newcommand{\simulate}{\mathit{sim}}

\newcommand{\Succ}{\Sf}
\newcommand{\Succb}{\Sf_2}
\newcommand{\pd}{\mathit{pd}}
\newcommand{\op}{\mathit{op}}
\newcommand{\mult}{\mathit{mult}}

\newcommand{\odd}{\mathit{odd}}
\newcommand{\even}{\mathit{even}}
\newcommand{\eq}{\mathit{eq}}

\newcommand{\lst}{\mathit{lst}}
\newcommand{\fst}{\mathit{fst}}

\newcommand{\rmsep}{\mathit{rmsep}}

%\newcommand{\rt}[2]{\(#1\)-Reduction-Tree}
%\newcommand{\RT}[3]{\mathit{RT}_{#1}(#2, #3)}

\newcommand{\rrs}{\mathit{rrs}}
\newcommand{\res}{\mathit{res}}
\newcommand{\lrs}{\mathit{lrs}}
\newcommand{\les}{\mathit{les}}

\newcommand{\sa}{\mathit{sa}}

\newcommand{\dectape}{\mathit{dectape}}

\newcommand{\Ll}{\mathbb{L}}

\newcommand{\Tt}{\mathbb{T}}

\newcommand{\blank}{\circledast}

\newcommand{\stmreach}{\lvert\triangleright}
\newcommand{\stmtrans}{\Vdash}

\newcommand{\tmreach}{\triangleright}
\newcommand{\tmtrans}{\vdash}

%\newcommand{\reaches}[2]{\triangleright^{#1}_{#2}}
\newcommand{\reachesf}[2]{{\overline{\triangleright}}^{#1}_{#2}}



\newcommand{\m}[1]{\textcolor{black}{#1}}

\newcommand{\g}[1]{\textcolor{black}{#1}}

\newcommand{\polyF}{\mathcal{PTF}}
\newcommand{\Id}{\mathit{Id}}

\newcommand{\tmstep}{\vdash}


\newcommand{\listenc}[2]{\langle{#1}\rangle^{#2}_{\Ll}}




%%% conditional printing

\newcommand{\Df}{\mathcal{D}}
\newcommand{\Hf}{\mathcal{H}}


\newcommand{\PL}{\mathbf{PL}}

\newcounter{short}
\setcounter{short}{0}
\newcounter{appendix}
\setcounter{appendix}{1}
\newcounter{extended}
\setcounter{extended}{2}


\newcounter{num}
\setcounter{num}{10}
\newcounter{commenting}
\setcounter{commenting}{0}

\newcommand{\always}{0}
\newcommand{\everything}{1}
\newcommand{\appendixorsup}{2}
\newcommand{\shortonly}{3}
\newcommand{\shortorsup}{4}
\newcommand{\extendedorsup}{5}
\newcommand{\onlyappendix}{6}
\newcommand{\notappendix}{7}

\newcommand{\lshort}{0}
\newcommand{\lappendix}{1}
\newcommand{\lextended}{2}
\newcommand{\leverything}{x}

\newenvironment{conditional}[1]%
        {%
            \ifthenelse{\NOT \(%
              \(\equal{#1}{\always} \AND 1 > 0\) \OR%
              \(\equal{#1}{\appendixorsup} \AND \(\equal{\level}{\lappendix} \OR \equal{\level}{\lextended} \OR \equal{\level}{\leverything}\)\) \OR%
              \(\equal{#1}{\shortorsup} \AND \(\equal{\level}{\lshort} \OR \equal{\level}{\lappendix} \equal{\level}{\lextended} \OR \equal{\level}{\leverything}\)\) \OR%
              \(\equal{#1}{\extendedorsup} \AND \(\equal{\level}{\lextended} \OR \equal{\level}{\leverything}\)\) \OR%
              \(\equal{#1}{\shortonly} \AND \(\equal{\level}{\lshort} \OR \equal{\level}{\leverything}\)\) \OR%
              \(\equal{#1}{\onlyappendix} \AND \(\equal{\level}{\lappendix} \OR \equal{\level}{\leverything}\)\) \OR%
              \(\equal{#1}{\notappendix} \AND \(\equal{\level}{\lshort} \OR \equal{\level}{\lextended} \OR \equal{\level}{\leverything}\)\)%
            \)}%
                    {\setcounter{commenting}{1}\expandafter\comment}%
                    {}%
                    }%
         {%
            \ifthenelse{\value{commenting} = 1}%
                    {\setcounter{commenting}{0}\expandafter\endcomment}%
                    {}%
          }

\addbibresource{../bibliography/bibliography.bib}

\title{Bounded Arithmetic and Randomized Computation \\[1.5ex] \Large Riassunto}
\author{Davide Davoli}


\begin{document}
\selectlanguage{italian}
\maketitle
Da novembre 2021 faccio parte di un progetto di ricerca che coinvolge assieme a me il Prof. Ugo Dal Lago, il Dott. Paolo Pistone, la Dott.ssa Melissa Antonelli dell'Università di Bologna e la Prof.ssa Isabel Oitavem dell'Università di Lisbona.

Il nostro lavoro di ricerca verte sull'estensione della teoria $S^1_2$ introdotta da S. Buss \cite{Buss86} per caratterizzare la classe $\FP$ \cite{Cobham1965} verso una nuova teoria, detta $\RS$, in grado di catturare classi di complessità probabilistiche mediante lo sviluppo di una randomized bounded arithmetic.
%
Un tale risultato permetterebbe, per esempio, di sviluppare caratterizzazioni
puramente logiche di classi di complessità probabilistiche e, dunque,
di studiare tali oggetti ed i problemi aperti ad essi legati senza impiegare metodi
solamente combinatorici, ma anche quelli logici.

\paragraph*{Preliminaries.} In questo capitalo vengono introdotte alcune nozioni
fondamentali per lo sviluppo dei capitoti successivi. In particolare,
viene definita in modo formale la nozione di spazio di probabilità, assieme a quella
di $\sigma$-algebra, e di misura. Nella sezione successiva, viene brevemente descritto
il paradigma di computazione probabilistico e vengono definite le principali classi di
complessità probabilistiche che catturano il concetto di \emph{feasibility} in tale ambito. Infine, vengono introdotti alcuni risultati preliminari del lavoro di S. Buss per lo sviluppo di una bounded arithmetic in grado di catturare la classe di complessità $\FP$.

\paragraph*{A Randomized Bounded Arithmetic.}
Questo capitolo descrive come, partendo dal lavoro di Ferreira \cite{Ferreira88}, sia possibile sviluppare una bounded arithmetic per catturare un'algebra di funzioni ivi introdotta e detta $\POR$. In primo luogo, viene definita tale algebra di funzioni e, assieme ad essa, il linguaggio al prim'ordine $\Lpw$ assieme ad una semantica quantitativa che associa ad ogni formula un insieme misurabile anziché un valore binario di verità. Successivamente, viene definita una nozione di $\Sigma^b_1$-rappresentabilità sul modello di quella definita da Buss in \cite{}. Infine, viene descritto come è possibile provare che tutte le funzioni $\Sigma^b_1$-rappresentabili all'interno della teoria $\RS$ appartengono a $\POR$. Nella sezione successiva, viene delineata la prova dell'inclusione inversa, seguendo \cite{}.

\paragraph*{On the equivalence between $\PPT$ and $\POR$.} Questo capitolo si occupa
di studiare l'espressività delle funzioni che si trovano all'interno di $\POR$. In particolare, all'inizio del capitolo si
congettura che tale classe di funzioni sia equivalente alla classe $\PPT$, ossia
l'insieme delle distribuzioni di stringhe che sono calcolabili da macchine di Turing
probabilistiche in tempo al più polinomiale. Per provare tale congettura, vengono
introdotti numerose classi di funzione, fondate su diversi paradigmi di computazione,
le quali le quali si dimostrano essere equivalenti l'una alle altre.
La più importante di queste
classi di funzioni è $\SFP$. Essa contiene tutte le funzioni che sono calcolabili in
tempo polinomiale da macchine di Turing dotate di un nastro aggiuntivo da cui leggono
una sequenza infinita di bit casuali. In una prima parte del capitolo, viene provata
un'equivalenza fra $\POR$ e $\SFP$, passivamente viene provata l'equivalenza fra
$\SFP$ e $\POR$, infine, l'equivalenza fra $\POR$ e $\PPT$.

\paragraph*{On the equivalence between $\FP$ and Cobham's Algebra.} In questo capitolo
vengono sfruttati i risultati del corollario precedente per derivare, come corollario,
l'equivalenza fra la classe di funzioni computabili da una macchina di Turing in tempo
polinomiale, detta $\FP$, e le funzioni descritte da un'algebra di funzioni $\fcob$
definita sul modello
di quella proposta da Cobham, \cite{Cobham1965}. Infatti, questo risultato, per quanto ne siamo a conoscenza,
non possiede alcuna prova sufficientemente esaustiva ed auto-contenuta, nonostante sia
vastamente condiviso in letteratura.
Dunque, dal momento che il modelli di calcolo su cui abbiamo definito $\SFP$
è molto simile a quello delle macchine di Turing così come l'algebra di
funzioni $\POR$ è molto simile a $\fcob$, abbiamo deciso di provare tale risultato.
L'equivalenza fra $\FP$ e $\fcob$ si ottiene dall'equivalenza fra $\POR$ e $\SFP$
mediante alcune semplici trasformazioni fra macchine di Turing.

\paragraph*{Characterizing complexity classes.} All'interno di quest'ultimo capitolo,
studiamo come è possibile estendere ulteriormente il linguaggio $\Lpw$ in modo da fare
uso della nozione di $\Sigma^b_1$-rappresentabilità per caratterizzare note classi di
complessità probabilistiche. A tal fine, prendendo ispirazione dall'aritmetica $\MQPA$
descritta in \cite{ADLP, ADLP21}, definiamo un'estensione di $\Lpw$ mediante un
quantificatore $\cq{s/t}F$ il cui valore di verità dipende direttamente dalla misura
dell'insieme descritto dalla semantica $F$, ottenendo linguaggio $\Lmq$.
Successivamente, sulla base delle definizioni delle classi di complessità
probabilistiche date in precedenza, definiamo le caratterizzazioni di $\BPP$,
$\RP$, co-$\RP$ e $\ZPP$ all'interno di $\Lmq$, provandone la correttezza.

\paragraph*{Future Work.} In questa sezione, congetturiamo la possibilità di ridurre
il linguaggio $\Lmq$ al linguaggio della logica al prim'ordine. Questo processo
richiederebbe la riduzione del quantificatore $\cq{s/t}$ ad un quantificatore a
soglia su modello di \cite{Gradel}, contestualmente a questa riduzione, tutto il linguaggio delle formule di $\Lmq$ sarebbe ridotto a quello standard della logica del
prim'ordine, ad eccezione del quantificatore a soglia. Successivamente,
sulla base di alcuni lavori simili \cite{Schweikardt, Chistikov2021PresburgerAW} ma su aritmetiche meno espressive,
congetturiamo che, quantomeno nel contesto delle caratterizzazioni
di classi di complessità probabilistiche, anche quest'ultimo quantificatore
possa a sua volta essere ridotto ad un quantificatore al prim'ordine standard.
Si potrebbe dunque identificare un sottoinsieme di $\BPP$ ricorsivamente
enumerabile contenente tutti i problemi di quest'ultima classe che sono
descritti da formule al prim'ordine provabili all'interno di un sistema come $\PA$.
Sarebbe dunque interessante studiare quali problemi che si congetturano essere
in $\BPP \setminus \mathbf P$ si trovino in questa classe.

\newpage
\printbibliography


\end{document}
