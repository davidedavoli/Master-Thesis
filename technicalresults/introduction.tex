%!TeX spellcheck = en-US
\label{app}
In this chapter we collect some of the technical results which are
employed in other parts of this work, hoping to foster the understanding
of the main thread of the proof, without lacking of completeness.
%
In the first section (Section \ref{sub:encoding}), we show how it is possible to define a co-articulated
collection of encodings of data structures on the set of binary strings $\Ss$.
Then we define $\POR$ functions --- Actually {$\POR^-$} functions ---
which are the embedding of canonical operations on these encodings.
%
The existence of these functions entails that, within $\POR$, it is possible
to represent and manipulate natural numbers, lists, sets and other
more complex data structures.
%
All these results, taken together be the basis of the proof that
$\SFP$ can be reduced to $\POR$.

In the following section (Section \ref{sec:auxres}),
we collect the auxiliary Lemmas employed in the previous chapters,
divided per section.
%
This is aimed to separate low-level results from the main
structure of the proof.

% Finally, this chapter is intended to collect all the non-novel work
% of this thesis, containing mainly technical results, so
% the reader who is not willing to get aquainted with the the technical
% details of the proof shall
